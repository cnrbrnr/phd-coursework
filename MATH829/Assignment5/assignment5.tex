\documentclass[10pt]{article}
\usepackage[margin=1.3cm]{geometry}

% Packages
\usepackage{amsmath, amsfonts, amssymb, amsthm}
\usepackage{bbm} 
\usepackage{dutchcal} % [dutchcal, calrsfs, pzzcal] calligraphic fonts
\usepackage{graphicx}
\usepackage[T1]{fontenc}
\usepackage[tracking]{microtype}

% Palatino for text goes well with Euler
\usepackage[sc,osf]{mathpazo}   % With old-style figures and real smallcaps.
\linespread{1.025}              % Palatino leads a little more leading

% Euler for math and numbers
\usepackage[euler-digits,small]{eulervm}

% Command initialization
\DeclareMathAlphabet{\pazocal}{OMS}{zplm}{m}{n} \graphicspath{{./images/}}

% Custom Commands
\newcommand{\bs}[1]{\boldsymbol{#1}} \newcommand{\E}{\mathbb{E}}
\newcommand{\var}[1]{\text{Var}\left(#1\right)}
\newcommand{\bp}[1]{\left({#1}\right)} \newcommand{\mbb}[1]{\mathbb{#1}}
\newcommand{\1}[1]{\mathbbm{1}_{#1}} \newcommand{\mc}[1]{\mathcal{#1}}
\newcommand{\nck}[2]{{#1\choose#2}} \newcommand{\pc}[1]{\pazocal{#1}}
\newcommand{\ra}[1]{\renewcommand{\arraystretch}{#1}}
\newcommand*{\floor}[1]{\left\lfloor#1\right\rfloor}
\newcommand*{\ceil}[1]{\left\lceil#1\right\rceil}
\newcommand{\ip}[2]{\left\langle#1,#2\right\rangle }

\DeclareMathOperator{\Var}{Var} \DeclareMathOperator{\Cov}{Cov}
\DeclareMathOperator{\diag}{diag}

\makeatletter
\def\Ddots{\mathinner{\mkern1mu\raise\p@
\vbox{\kern7\p@\hbox{.}}\mkern2mu
\raise4\p@\hbox{.}\mkern2mu\raise7\p@\hbox{.}\mkern1mu}}
\makeatother

\def\powertower#1#2{#1\ifnum#2>1 ^{\powertower{#1}{\numexpr#2-1\relax}}\fi}

\newtheorem{theorem}{Theorem}
\newtheorem{lemma}{Lemma}

\begin{document}

    \begin{center}
        {\bf\large{MATH 829: FUNCTIONAL ANALYSIS AND QUANTUM MECHANICS}}
        \smallskip
        \hrule
        \smallskip
        {\bf Assignment} 5\hfill {\bf Connor Braun} \hfill {\bf 2024-10-16}
    \end{center}
    \vspace{5pt}
    \begin{center}
        \begin{minipage}{\dimexpr\paperwidth-10cm}
            I did not work with anybody in completing this assignment.
        \end{minipage}
    \end{center}
    \noindent{\bf Problem 1} Let $(X,\ip{\cdot}{\cdot})$ be a Hilbert space over $\mbb{K}\in\{\mbb{R},\mbb{C}\}$. Let $(e_j)_{j\geq 1}$ be an infinite orthonormal
    system in $(X,\ip{\cdot}{\cdot})$ Prove that for every $x\in X$, we have the so-called {\it Bessel's inequality}
    $\sum_{j=1}^\infty|\ip{x}{e_j}|^2\leq\|x\|^2$. Follow the steps below:\\[5pt]
    {\bf a)} Let $y_n=\sum_{j=1}^n\ip{x}{e_j}e_j$. Show that $\ip{y_n}{x}=\sum_{j=1}^n|\ip{x}{e_j}|^2$.\\[5pt]
    {\bf Solution}. For $n\geq 1$, we directly compute: 
    \begin{align*}
        \ip{y_n}{x}=\ip{\sum_{j=1}^n\ip{x}{e_j}e_j}{x}=\sum_{j=1}^n\ip{x}{e_j}\ip{e_j}{x}=\sum_{j=1}^n\ip{x}{e_j}\overline{\ip{x}{e_j}}=\sum_{j=1}^n|\ip{x}{e_j}|^2.\tag*{$\qed$}
    \end{align*}
    {\bf b)} Show that $\|y_n\|^2=\sum_{j=1}^n|\ip{x}{e_j}|^2$ for $n\geq 1$.\\[5pt]
    {\bf Solution}. Fixing $n\geq 1$, we can again directly compute
    \begin{align*}
        \|y_n\|^2=\ip{y_n}{y_n}=\ip{\sum_{j=1}^n\ip{x}{e_j}e_j}{\sum_{j=1}^n\ip{x}{e_j}e_j}&=\sum_{j=1}^n\ip{x}{e_j}\ip{e_j}{\sum_{j=1}^n\ip{x}{e_j}e_j}\\
        &=\sum_{j=1}^n\sum_{k=1}^n\ip{x}{e_j}\overline{\ip{x}{e_k}}\ip{e_j}{e_k}\\
        &=\sum_{j=1}^n\sum_{k=1}^n\ip{x}{e_j}\overline{\ip{x}{e_k}}\delta_{j,k}\\
        &=\sum_{j=1}^n\ip{x}{e_j}\overline{\ip{x}{e_j}}\\
        &=\sum_{j=1}^n|\ip{x}{e_j}|^2
    \end{align*}
    where $\delta_{j,k}$ is the Kronecker delta evaluated at $j,k\geq 1$.\hfill{$\qed$}\\[5pt]
    {\bf c)} Use Cauchy-Schwarz to show that $\sum_{j=1}^n|\ip{x}{e_j}|^2\leq \|x\|^2$.\\[5pt]
    {\bf Solution}. For any $n\geq 1$, part (a) says that $\ip{y_n}{x}\geq 0$ (as a sum of nonnegative terms). Thus, $|\ip{y_n}{x}=\ip{y_n}{x}|$ so by applying Cauchy-Schwarz,
    \begin{align*}
        |\ip{y_n}{x}|\leq \|y_n\|\|x\|=\|x\|\left(\sum_{j=1}^n|\ip{x}{e_j}|^2\right)^{1/2}
    \end{align*} 
    where the equality follows (b). Provided $x\neq 0$, and $n$ is sufficiently large to have $\ip{x}{e_j}\neq 0$ for some $1\leq j\leq n$ (for otherwise $0=\sum_{j=1}^n|\ip{x}{e_j}|^2\leq\|x\|^2$ trivially) this implies
    \begin{align*}
        \sum_{j=1}^n|\ip{x}{e_j}|^2\leq \|x\|\left(\sum_{j=1}^n|\ip{x}{e_j}|^2\right)^{1/2}\quad\Rightarrow\quad \left(\sum_{j=1}^n|\ip{x}{e_j}|^2\right)^{1/2}\leq \|x\|\quad\Rightarrow\quad \sum_{j=1}^n|\ip{x}{e_j}|^2\leq \|x\|^2.\tag*{$\qed$}
    \end{align*}
    {\bf d)} Explain why the inequality from (c) implies Bessel's inequality.\hfill{$\qed$}\\[5pt]
    {\bf Solution}. Since the above holds for all $n\geq1$ sufficiently large, we can take the limit to find
    \begin{align*}
        \lim_{n\rightarrow\infty}\sum_{j=1}^n|\ip{x}{e_j}|^2\leq\lim_{n\rightarrow\infty}\|x\|^2\quad\Rightarrow\quad\sum_{j=1}^\infty|\ip{x}{e_j}|^2\leq\|x\|^2
    \end{align*}
    which is Bessel's inequality precisely.\\[5pt]
    {\bf e)} Let $(e_j)_{j\geq 1}$ be an infinite orthonormal system in $(X,\ip{\cdot}{\cdot})$. Show that for every $x\in X$, $\exists y\in X$ such that $y=\sum_{j=1}^\infty\ip{x}{e_j}e_j$.\\[5pt]
    {\bf Proof.} Fix $x\in X$, and define $y_n:=\sum_{j=1}^n\ip{x}{e_j}e_j$ as in (a). Then, defining $y:=\sum_{j=1}^\infty\ip{x}{e_j}e_j=\lim_{n\rightarrow\infty}\sum_{j=1}^n\ip{x}{e_j}e_j=\lim_{n\rightarrow\infty}y_n$, we can use
    (b) to show
    \begin{align*}
        \|y\|^2=\|\lim_{n\rightarrow\infty}y_n\|^2=\lim_{n\rightarrow\infty}\|y_n\|^2=\lim_{n\rightarrow\infty}\sum_{j=1}^n|\ip{x}{e_j}|^2=\sum_{j=1}^\infty|\ip{x}{e_j}|^2\leq \|x\|^2<\infty
    \end{align*}
    where the second equality follows from the continuity of the norm, the third from part (b), and the inequality used is Bessel's. From this, we conclude that $\|y\|<\infty$, so indeed $y\in X$.\\[5pt]
    \noindent{\bf Problem 2}. For $n\in\mbb{Z}$, let $e_n(t)=\frac{e^{2int}}{\sqrt{\pi}}$, where $i=\sqrt{-1}$.\\[5pt]
    {\bf a)} Assuming that $(e_n(t))_{n\in\mbb{Z}}$ is an orthonormal basis for the Hilbert space $L_2\left[-\frac{\pi}{2},\frac{\pi}{2}\right]$ with the usual inner product, prove that
    \begin{align*}
        \left\{\frac{1}{\sqrt{\pi}},\frac{\cos(2nt)}{\sqrt{\pi/2}},\frac{\sin(2nt)}{\sqrt{\pi/2}}\right\}_{n\geq 1}
    \end{align*}
    is also an orthonormal basis.\\[5pt]
    {\bf Proof}. We shall denote $X:=L_2\left[-\frac{\pi}{2},\frac{\pi}{2}\right]$, and $D:=\left[-\frac{\pi}{2},\frac{\pi}{2}\right]$ for notational ease. Fixing some $x\in X$, $\exists!(\alpha_n)_{n\in\mbb{Z}}$ so that $x=\sum_{n=-\infty}^{\infty}\alpha_ne_n$ since $(e_n)_{n\in\mbb{Z}}$ is a basis. 
    Recalling Euler's identity $e^{iy}=\cos(y)+i\sin(y)$ for $y\in\mbb{R}$, we can manipulate the basis expansion for $x$ to find for $t\in D$,
    \begin{align*}
        x(t)=\sum_{n=-\infty}^\infty\alpha_ne_n(t)=\sum_{n=-\infty}^\infty\alpha_n\frac{e^{2int}}{\sqrt{\pi}}&=\sum_{n=-\infty}^\infty\frac{\alpha_n}{\sqrt{\pi}}(\cos(2nt)+i\sin(2nt))\\
        &=\frac{\alpha_0}{\sqrt{\pi}}+\sum_{n=1}^\infty\left[\frac{\alpha_n}{\sqrt{\pi}}(\cos(2nt)+i\sin(2nt))+\frac{\alpha_{-n}}{\sqrt{\pi}}(\cos(-2nt)+i\sin(-2nt))\right]\\
        &=\frac{\alpha_0}{\sqrt{\pi}}+\sum_{n=1}^\infty\left[\frac{\alpha_n}{\sqrt{\pi}}(\cos(2nt)+i\sin(2nt))+\frac{\alpha_{-n}}{\sqrt{\pi}}(\cos(2nt)-i\sin(2nt))\right]\\
        &=\frac{\alpha_0}{\sqrt{\pi}}+\sum_{n=1}^\infty\left[\frac{\alpha_n+\alpha_{-n}}{\sqrt{2}}\frac{\cos(2nt)}{\sqrt{\pi/2}}+i\frac{\alpha_n-\alpha_{-n}}{\sqrt{2}}\frac{\sin(2nt)}{\sqrt{\pi/2}}\right].
    \end{align*}
    From here, we define $\beta_0:=\alpha_0$, $g_0:=\frac{1}{\sqrt{\pi}}$, and for $n\geq 1$,
    \begin{align*}
        \beta_n:=\begin{cases}
            \frac{\alpha_{\frac{n+1}{2}}+\alpha_{-\frac{n+1}{2}}}{\sqrt{2}},\quad&\text{if $2\nmid n$}\\
            i\frac{\alpha_{\frac{n}{2}}-\alpha_{-\frac{n}{2}}}{\sqrt{2}},\quad&\text{if $2\mid n$}
        \end{cases},\quad\text{and}\quad g_n(t):=\begin{cases}
            \frac{\cos(2(\frac{n+1}{2})t)}{\sqrt{\pi/2}},\quad&\text{if $2\nmid n$}\\
            \frac{\sin(2(\frac{n}{2})t)}{\sqrt{\pi/2}},\quad&\text{if $2\mid n$}
        \end{cases},\quad t\in D.
    \end{align*}
    With this, our previous calculation yields $x(t)=\frac{\beta_0}{\sqrt{\pi}}+\sum_{n=1}^\infty\beta_ng_n(t)$, and the uniqueness of this representation in $(\beta_n)_{n\geq 0}$ follows the uniqueness of the $(\alpha_n)_{n\geq \mbb{Z}}$.\\[5pt]
    \indent Lastly, we note that $(g_n)_{n\geq 0}$ is indeed an orthonormal system. To show this,
    notice that since $D$ is symmetric about zero, $\int_Df(t)dt=0$ whenever $f\in X$ is an odd function (that is, whenever $f(-t)=-f(t)$ for $t\in D$). In particular, $\sin(kt)$ is odd for any $k\in\mbb{R}$. Then notice that via some well-known trigonometric identities:
    \begin{align*}
        \cos(k_1t)\sin(k_2t)=\frac{\sin((k_1+k_2)t)-\sin((k_1-k_2)t)}{2},\quad\text{and}\quad \sin(k_1t)\cos(k_2t)=\frac{\sin((k_1+k_2)t)+\sin((k_1-k_2)t)}{2}
    \end{align*}
    for $k_1,k_2\in\mbb{R}$. Therefore, we can immediately conclude that $\ip{g_n}{g_m}=0$ whenever $n,m>0$ are not of the same parity. Then, when $n,m\geq 1$:
    \begin{align*}
        \ip{g_{2n-1}}{g_{2m-1}}=\frac{2}{\pi}\int_D\cos(2nt)\cos(2mt)dt&=\frac{2}{2\pi}\left(\int_D\cos(2(n-m)t)dt+\int_D\cos(2(n+m)t)dt\right)\\
        &=\begin{cases}
            \frac{1}{\pi}\left(\pi+\frac{1}{2(n+m)}\sin(2(n+m)t)\bigg|^{\frac{\pi}{2}}_{-\frac{\pi}{2}}\right)=1,\quad&\text{if $n=m$}\\
            \frac{1}{\pi}\left(\frac{1}{2(n-m)}\sin(2(n-m)t)\bigg|^{\frac{\pi}{2}}_{-\frac{\pi}{2}}+\frac{1}{2(n+m)}\sin(2(n+m)t)\bigg|^{\frac{\pi}{2}}_{-\frac{\pi}{2}}\right)=0,\quad&\text{if $n\neq m$}\\
        \end{cases}\\
        &=\delta_{n,m}\\
        &=\delta_{2n-1,2m-1},\\
        \ip{g_{2n}}{g_{2m}}=\frac{2}{\pi}\int_D\sin(2nt)\sin(2mt)dt&=\frac{2}{2\pi}\left(\int_D\cos(2(n-m)t)dt-\int_D\cos(2(n+m)t)dt\right)\\
        &=\begin{cases}
            \frac{1}{\pi}\left(\pi-\frac{1}{2(n+m)}\sin(2(n+m)t)\bigg|^{\frac{\pi}{2}}_{-\frac{\pi}{2}}\right)=1,\quad&\text{if $n=m$}\\
            \frac{1}{\pi}\left(\frac{1}{2(n-m)}\sin(2(n-m)t)\bigg|^{\frac{\pi}{2}}_{-\frac{\pi}{2}}-\frac{1}{2(n+m)}\sin(2(n+m)t)\bigg|^{\frac{\pi}{2}}_{-\frac{\pi}{2}}\right)=0,\quad&\text{if $n\neq m$}\\
        \end{cases}\\
        &=\delta_{n,m}\\
        &=\delta_{2n,2m}
    \end{align*}
    covering the cases where both indices are odd (the former) or even (the latter). Lastly, for any $n\geq 1$ we have one of three scenarios:
    \begin{align*}
        \left[2\nmid n\quad\Rightarrow\quad\ip{g_0}{g_n}=\frac{\sqrt{2}}{\pi}\int_{D}\cos(2\frac{n+1}{2}t)dt=0\right],\quad \left[2\mid n\quad\Rightarrow\quad\ip{g_0}{g_n}=\frac{\sqrt{2}}{\pi}\int_D\sin(2\frac{n}{2}t)dt=0\right]
    \end{align*}
    and $\ip{g_0}{g_0}=\frac{1}{\pi}\int_Ddt=\frac{\pi}{\pi}=1$, so indeed, for any $n,m\geq 0$, $\ip{g_n}{g_m}=\delta_{n,m}$. We conclude that the proposed set is indeed an orthornormal basis.\hfill{$\qed$}\\[5pt]
    {\bf b)} Write $f_1(t)=t$ and $f_2(t)=t^2$ (both in $L_2(D)$) with respect to the basis found in (a).\\[5pt]
    {\bf Solution}. Dealing with $f_1(t)$ first, for $n\geq 1$ we get
    \begin{align*}
        \ip{f_1}{g_0}&=\frac{1}{\sqrt{\pi}}\int_Dtdt=0,\quad\ip{f_1}{g_{2n-1}}=\frac{\sqrt{2}}{\sqrt{\pi}}\int_Dt\cos(2nt)dt=0
    \end{align*}
    since in both cases the integrand is an odd function and $D$ is symmetric about $0\in\mbb{R}$. Alternatively,
    \begin{align*}
        \ip{f_1}{g_{2n}}=\frac{\sqrt{2}}{\sqrt{\pi}}\int_Dt\sin(2nt)dt&=\frac{\sqrt{2}}{\sqrt{\pi}}\left[-\frac{t}{2n}\cos(2nt)\bigg|^{\frac{\pi}{2}}_{-\frac{\pi}{2}}+\int_D\frac{1}{2n}\cos(2nt)dt\right]\\
        &=\frac{\sqrt{2}}{\sqrt{\pi}}\left[-\frac{\pi}{2n}\cos(n\pi)+\frac{1}{4n^2}\left(\sin(n\pi)-\sin(-n\pi)\right)\right]\\
        &=\frac{\sqrt{\pi}}{\sqrt{2}n}(-1)\cos(n\pi)\\
        &=\frac{\sqrt{\pi}}{\sqrt{2}n}(-1)^{n+1}.
    \end{align*}
    With this, we may express
    \[f_1(t)=\sum_{n\geq 0}\ip{f_1}{g_n}g_n(t)=\sum_{n\geq 1}\ip{f_1}{g_{2n}}g_{2n}(t)=\sum_{n\geq 1}\frac{1}{n}(-1)^{n+1}\sin(2nt).\]
    We may repeat a similar procedure for $f_2$:
    \begin{align*}
        \ip{f_2}{g_0}&=\frac{1}{\sqrt{\pi}}\int_Dt^2dt=\frac{1}{3\sqrt{\pi}}t^3\bigg|^{\frac{\pi}{2}}_{-\frac{\pi}{2}}=\frac{\pi^{5/2}}{12},\\
        \ip{f_2}{g_{2n}}&=\frac{\sqrt{2}}{\sqrt{\pi}}\int_Dt^2\sin(2nt)dt=0
    \end{align*}
    where the latter holds since (once again) the integrand is an odd function. For the case of odd indices:
    \begin{align*}
        \ip{f_2}{g_{2n-1}}=\frac{\sqrt{2}}{\sqrt{\pi}}\int_Dt^2\cos{2nt}dt&=\frac{\sqrt{2}}{\sqrt{\pi}}\left[\frac{t^2}{2n}\sin(2nt)\bigg|^{\frac{\pi}{2}}_{-\frac{\pi}{2}}-\frac{1}{n}\int_Dt\sin(2nt)dt\right]\\
        &=-\frac{\sqrt{2}}{\sqrt{\pi}n}\int_Dt\sin(2nt)dt\\
        &=\frac{\sqrt{\pi}}{\sqrt{2}n^2}(-1)^n.
    \end{align*}
    The last equality holds since the second line is just $-\frac{1}{n}\ip{f_1}{g_{2n}}$ when $n\geq 1$. Now we have a representation of $f_2$ in the basis as well:
    \begin{align*}
        f_2(t)=\sum_{n\geq 0}\ip{f_2}{g_n}g_n(t)=\frac{\pi^{2}}{12}+\sum_{n\geq 1}\frac{1}{n^2}(-1)^n\cos(2nt).\tag*{$\qed$}
    \end{align*}
    {\bf c)} Use Parseval's identity to prove that
    \begin{align*}
        \sum_{n=1}^\infty\frac{1}{n^2}=\frac{\pi^2}{6},\quad\text{and}\quad\sum_{n=1}^\infty\frac{1}{n^4}=\frac{\pi^4}{90}.
    \end{align*}
    {\bf Solution.} Let $\|\cdot\|:L_2(D)\rightarrow\mbb{R}_+$ be the norm on $L_2(D)$ induced by $\ip{\cdot}{\cdot}$. From part (b), Parseval's identity tells us that
    \begin{align*}
        \sum_{n=0}^\infty|\ip{f_1}{g_n}|^2=\|f_1\|^2=\int_Dt^2dt=\frac{\pi^3}{12}\quad&\Rightarrow\quad \frac{\pi^3}{12}=\sum_{n=1}^\infty\left|\frac{\sqrt{\pi}}{\sqrt{2}n}(-1)^{n+1}\right|^2\\
        &\Rightarrow\quad\frac{\pi^3}{12}=\sum_{n=1}^\infty \frac{\pi}{2n^2}\\
        &\Rightarrow\quad\frac{\pi^2}{6}=\sum_{n=1}^\infty\frac{1}{n^2}
    \end{align*}
    which is the first equality. For the second, 
    \begin{align*}
        \sum_{n=0}^\infty|\ip{f_2}{g_n}|^2=\|f_2\|^2=\int_Dt^4dt=\frac{\pi^5}{80}\quad&\Rightarrow\quad\frac{\pi^5}{80}=\left|\frac{\pi^{5/2}}{12}\right|^2+\sum_{n=1}^\infty\left|\frac{\sqrt{\pi}}{\sqrt{2}n^2}\right|^2\\
        &\Rightarrow\quad\frac{\pi^5}{80}-\frac{\pi^5}{144}=\frac{\pi}{2}\sum_{n=1}^\infty\frac{1}{n^4}\\
        &\Rightarrow\quad\frac{\pi^4}{40}-\frac{\pi^4}{72}=\sum_{n=1}^\infty\frac{1}{n^4}\\
        &\Rightarrow\quad\frac{9\pi^4-5\pi^4}{360}=\sum_{n=1}^\infty\frac{1}{n^4}\\
        &\Rightarrow\quad\frac{\pi^4}{90}=\sum_{n=1}^\infty\frac{1}{n^4}
    \end{align*}
    and we are done.\hfill{$\qed$}\\[5pt]
    {\bf Problem 3}. Let $(e_j)_{j\geq 0}$ be an orthonormal basis for the separable Hilbert space $(X,\ip{\cdot}{\cdot})$ over $\mbb{K}$. We define two
    sequences of vectors $(u_j)_{j\geq 0}$ and $(v_j)_{j\geq 0}$, where $u_j:=e_{2j}$ and $v_j:=e_{2j}+\frac{1}{j+1}e_{2j+1}$ for $j\geq 0$. Let
    \begin{align*}
        \mc{Y}:=\overline{\left\{\sum_{j=0}^n\alpha_ju_j:\;n\geq 0,\;\alpha_0,\dots,\alpha_n\in\mbb{K}\right\}},\quad\text{and}\quad \mc{Z}:=\overline{\left\{\sum_{j=0}^m\beta_jv_j:\;n\geq 0,\;\beta,\dots,\beta\in\mbb{K}\right\}}.
    \end{align*}
    Show that $\mc{Y}+\mc{Z}$ is not closed.\\[5pt]
    {\bf Solution}. We first show that $\mc{Y}+\mc{Z}$ is dense in $X$. For $n\geq 0$, $e_{2n}=u_n+0$, so $e_{2n}\in\mc{Y}+\mc{Z}$, and $e_{2n+1}=-(n+1)u_n+(n+1)v_n$, so $e_{2n+1}\in\mc{Y}+\mc{Z}$. That is, $\{e_n\}_{n\geq 0}\subset \mc{Y}+\mc{Z}$.
    Now, for any $x\in X$, $\exists!(\alpha_j)_{j\geq 0}$ so that $x=\sum_{j=0}^\infty\alpha_je_j=\lim_{N\rightarrow\infty}\sum_{j=0}^N\alpha_je_j$, and $(\sum_{j=0}^N\alpha_je_j)_{N\geq 0}\subset \mc{Y}+\mc{Z}$. From this we conclude that $\overline{\mc{Y}+\mc{Z}}=X$, so our query reduces to
    demonstrating that $X\supsetneq\mc{Y}+\mc{Z}$.\\[5pt]
    Now, consider the series $x^\ast=\sum_{j=0}^\infty\frac{1}{(j+1)}e_{2j+1}$. First,
    \begin{align*}
        \|x^\ast\|^2=\left\|\sum_{n=0}^\infty\frac{e_{2n+1}}{n+1}\right\|^2=\ip{\sum_{n=0}^\infty\frac{e_{2n+1}}{n+1}}{\sum_{n=0}^\infty\frac{e_{2n+1}}{n+1}}&=\ip{\lim_{N\rightarrow\infty}\sum_{n=0}^N\frac{e_{2n+1}}{n+1}}{\lim_{M\rightarrow\infty}\sum_{m=0}^M\frac{e_{2m+1}}{m+1}}\\
        &=\lim_{N\rightarrow\infty}\lim_{M\rightarrow\infty}\sum_{n=0}^N\sum_{m=0}^M\ip{\frac{e_{2n+1}}{n+1}}{\frac{e_{2m+1}}{m+1}}\\
        &=\lim_{N\rightarrow\infty}\lim_{M\rightarrow\infty}\sum_{n=0}^N\sum_{m=0}^M\frac{1}{n+1}\frac{1}{m+1}\ip{e_{2n+1}}{e_{2m+1}}\\
        &=\lim_{N\rightarrow\infty}\sum_{n=0}^N\frac{1}{(n+1)^2}\\
        &=\sum_{n=0}^\infty\frac{1}{(n+1)^2}\tag{1}\\
        &<\infty
    \end{align*}
    so $x^\ast\in X$. Now assume for the purpose of deriving a contradiction that $x^\ast\in\mc{Y}+\mc{Z}$. That is, $\exists y\in\mc{Y}$, $z\in\mc{Z}$ so that $x^\ast=y+z$. In particular, we now have
    \begin{align*}
        \|x^\ast\|^2=\ip{\sum_{n=0}^\infty\frac{e_{2n+1}}{n+1}}{y+z}=\ip{\sum_{n=0}^\infty\frac{e_{2n+1}}{n+1}}{y}+\ip{\sum_{n=0}^\infty\frac{e_{2n+1}}{n+1}}{z}\tag{2}
    \end{align*}
    but $y\in\mc{Y}$ implies that $\exists\left(\sum_{j=0}^J\alpha_je_{2j}\right)_{J\geq 0}\subset\mc{Y}$ so that $\lim_{J\rightarrow\infty}\sum_{j=0}^J\alpha_je_{2j}=y$, and similarly $\exists\left(\sum_{j=0}^J\beta_j(e_{2j}+\frac{e_{2j+1}}{j+1})\right)_{J\geq 0}\subset\mc{Z}$ with $\lim_{J\rightarrow\infty}\sum_{j=0}^J\beta_j(e_{2j}+\frac{e_{2j+1}}{j+1})=z$. We now compute
    (2) using these representations -- taking advantage of the continuity of the inner product (as above):
    \begin{align*}
        \ip{\sum_{n=0}^\infty\frac{e_{2n+1}}{n+1}}{y}=\ip{\sum_{n=0}^\infty\frac{e_{2n+1}}{n+1}}{\sum_{j=0}^\infty\alpha_je_{2j}}&=\lim_{N\rightarrow\infty}\lim_{J\rightarrow\infty}\sum_{n=0}^N\sum_{j=0}^J\frac{\overline{a_j}}{n+1}\ip{e_{2n+1}}{e_{2j}}=0
    \end{align*}
    and
    \begin{align*}
        \ip{\sum_{n=0}^\infty\frac{e_{2n+1}}{n+1}}{z}=\ip{\sum_{n=0}^\infty\frac{e_{2n+1}}{n+1}}{\sum_{j=0}^\infty\beta_j(e_{2j}+\frac{e_{2j+1}}{j+1})}&=\lim_{N\rightarrow\infty}\lim_{J\rightarrow\infty}\sum_{n=0}^N\sum_{j=0}^J\frac{1}{n+1}\overline{\beta_j}\left(\ip{e_{2n+1}}{e_{2j}}+\ip{e_{2n+1}}{\frac{e_{2j+1}}{j+1}}\right)\\
        &=\lim_{N\rightarrow\infty}\lim_{J\rightarrow\infty}\sum_{n=0}^N\sum_{j=0}^J\frac{1}{n+1}\overline{\beta_j}\frac{1}{j+1}\ip{e_{2n+1}}{e_{2j+1}}\\
        &=\lim_{N\rightarrow\infty}\sum_{n=0}^N\left(\frac{1}{n+1}\right)^2\overline{\beta_n}\\
    \end{align*}
    thus, for equality between (1) and (2), we need $\overline{\beta_n}=1$ for $n\geq 0$. In particular, $x^\ast=z$ and thus
    \begin{align*}
        \sum_{n=0}^\infty\frac{e_{2n+1}}{n+1}=\sum_{n=0}^\infty e_{2n}+\frac{e_{2n+1}}{n+1}
    \end{align*}
    and, furthermore,
    \begin{align*}
        \|x^\ast\|^2&=\ip{\sum_{n=0}^\infty e_{2n}+\frac{e_{2n+1}}{n+1}}{\sum_{j=0}^\infty e_{2j}+\frac{e_{2j+1}}{j+1}}\\
        &=\lim_{N\rightarrow\infty}\lim_{J\rightarrow\infty}\sum_{n=0}^N\sum_{j=0}^J\left(\ip{e_{2n}}{e_{2j}}+\frac{1}{j+1}\ip{e_{2n}}{e_{2j+1}}+\frac{1}{n+1}\ip{e_{2n+1}}{e_{2j}}+\frac{1}{(n+1)(j+1)}\ip{e_{2n+1}}{e_{2j+1}}\right)\\
        &=\lim_{N\rightarrow\infty}\sum_{n=0}^N\left(1+\frac{1}{(n+1)^2}\right)\\
        &=\infty
    \end{align*}
    which contradicts the previously established $\|x^\ast\|^2<\infty$. We conclude that $x^\ast\notin\mc{Y}+\mc{Z}$, so $\mc{Y}+\mc{Z}\neq X$, and thus $\mc{Y}+\mc{Z}\neq \overline{\mc{Y}+\mc{Z}}$. That is, $\mc{Y}+\mc{Z}$ is not closed.\hfill{$\qed$}\\[5pt]
    {\bf Problem 4}. Let $(X,\ip{\cdot}{\cdot})$ be a Hilbert space, and let $L\subset X$ be a subspace. The orthogonal complement of $L$ is defined $L^\perp:=\{x\in X:\ip{x}{y}=0\;\text{for every $y\in L$}\}$. Show that $L$ is closed if and only if $(L^\perp)^\perp=L$.\\[5pt]
    {\bf Proof}. Let us begin by supposing that $L$ is closed. Note that $(L^\perp)^\perp:=\{x\in X:\;\ip{x}{y}=0\;\text{for every $y\in L^\perp$.}\}$. Then, if $x\in L$, we have $\ip{x}{y}=0$ for all $y\in L^\perp$, so $x\in (L^\perp)^\perp$, and further $L\subseteq (L^\perp)^\perp$.
    For the reverse inclusion, now consider $x\in (L^\perp)^\perp$. Since $L$ is closed, $X=L\oplus L^\perp$, so $x=x_L+x_{L^\perp}$ for some $x_L\in L$ and $x_{L^\perp}\in L^\perp$. Assume for the purpose of deriving a contradiction that $x\notin L$. Then $x_{L^\perp}\neq 0$, 
    and thus
    \[0=\ip{x}{x_{L^\perp}}=\ip{x_L+x_{L^\perp}}{x_{L^\perp}}=\ip{x_L}{x_{L^\perp}}+\ip{x_{L^\perp}}{x_{L^\perp}}=\|x_{L^\perp}\|^2>0\]
    a contradiction. We conclude that $(L^\perp)^\perp\subseteq L$, so now $L=(L^\perp)^\perp$.\\[5pt]
    For the converse, suppose that $L=(L^\perp)^\perp$ and let $(\ell_j)_{j\geq 1}\subseteq L$ be such that $\ell_j\rightarrow\ell\in X$ as $j\rightarrow\infty$. But then, for all $y\in L^\perp$ and $j\geq 1$, $\ip{\ell_j}{y}=0$ since $\ell_j\in L=(L^\perp)^\perp$. Taking the limit:
    \begin{align*}
        0=\lim_{j\rightarrow\infty}\ip{\ell_j}{y}=\ip{\lim_{j\rightarrow\infty}\ell_j}{y}=\ip{\ell}{y}
    \end{align*}
    so $\ell\in (L^\perp)^\perp=L$. We conclude that any convergent sequence in $L$ has its limit in $L$, so $L$ is closed.\hfill{$\qed$} 
    \end{document}