\documentclass[10pt]{article}
\usepackage[margin=1.3cm]{geometry}

% Packages
\usepackage{amsmath, amsfonts, amssymb, amsthm}
\usepackage{bbm} 
\usepackage{dutchcal} % [dutchcal, calrsfs, pzzcal] calligraphic fonts
\usepackage{graphicx}
\usepackage[T1]{fontenc}
\usepackage[tracking]{microtype}

% Palatino for text goes well with Euler
\usepackage[sc,osf]{mathpazo}   % With old-style figures and real smallcaps.
\linespread{1.025}              % Palatino leads a little more leading

% Euler for math and numbers
\usepackage[euler-digits,small]{eulervm}

% Command initialization
\DeclareMathAlphabet{\pazocal}{OMS}{zplm}{m}{n} \graphicspath{{./images/}}

% Custom Commands
\newcommand{\bs}[1]{\boldsymbol{#1}} \newcommand{\E}{\mathbb{E}}
\newcommand{\var}[1]{\text{Var}\left(#1\right)}
\newcommand{\bp}[1]{\left({#1}\right)} \newcommand{\mbb}[1]{\mathbb{#1}}
\newcommand{\1}[1]{\mathbbm{1}_{#1}} \newcommand{\mc}[1]{\mathcal{#1}}
\newcommand{\nck}[2]{{#1\choose#2}} \newcommand{\pc}[1]{\pazocal{#1}}
\newcommand{\ra}[1]{\renewcommand{\arraystretch}{#1}}
\newcommand*{\floor}[1]{\left\lfloor#1\right\rfloor}
\newcommand*{\ceil}[1]{\left\lceil#1\right\rceil}
\newcommand{\ip}[2]{\left\langle#1,#2\right\rangle }

\DeclareMathOperator{\Var}{Var} \DeclareMathOperator{\Cov}{Cov}
\DeclareMathOperator{\diag}{diag}

\makeatletter
\def\Ddots{\mathinner{\mkern1mu\raise\p@
\vbox{\kern7\p@\hbox{.}}\mkern2mu
\raise4\p@\hbox{.}\mkern2mu\raise7\p@\hbox{.}\mkern1mu}}
\makeatother

\def\powertower#1#2{#1\ifnum#2>1 ^{\powertower{#1}{\numexpr#2-1\relax}}\fi}

\newtheorem{theorem}{Theorem}
\newtheorem{lemma}{Lemma}

\begin{document}

    \begin{center}
        {\bf\large{MATH 829: FUNCTIONAL ANALYSIS AND QUANTUM MECHANICS}}
        \smallskip
        \hrule
        \smallskip
        {\bf Assignment} 5\hfill {\bf Connor Braun} \hfill {\bf 2024-10-16}
    \end{center}
    \vspace{5pt}
    \begin{center}
        \begin{minipage}{\dimexpr\paperwidth-10cm}
            I did not work with anybody in completing this assignment.
        \end{minipage}
    \end{center}
    \noindent{\bf Problem 4}. Let $(X,\ip{\cdot}{\cdot})$ be a Hilbert space, and let $L\subset X$ be a subspace. The orthogonal complement of $L$ is defined $L^\perp:=\{x\in X:\ip{x}{y}=0\;\text{for every $y\in L$}\}$. Show that $L$ is closed if and only if $(L^\perp)^\perp=L$.\\[5pt]
    {\bf Proof}. Let us begin by supposing that $L$ is closed. Note that $(L^\perp)^\perp:=\{x\in X:\;\ip{x}{y}=0\;\text{for every $y\in L^\perp$.}\}$. Then, if $x\in L$, we have $\ip{x}{y}=0$ for all $y\in L^\perp$, so $x\in (L^\perp)^\perp$, and further $L\subseteq (L^\perp)^\perp$.
    For the reverse inclusion, now consider $x\in (L^\perp)^\perp$. Since $L$ is closed, $X=L\oplus L^\perp$, so $x=x_L+x_{L^\perp}$ for some $x_L\in L$ and $x_{L^\perp}\in L^\perp$. Assume for the purpose of deriving a contradiction that $x\notin L$. Then $x_{L^\perp}\neq 0$, 
    and thus
    \[0=\ip{x}{x_{L^\perp}}=\ip{x_L+x_{L^\perp}}{x_{L^\perp}}=\ip{x_L}{x_{L^\perp}}+\ip{x_{L^\perp}}{x_{L^\perp}}=\|x_{L^\perp}\|^2>0\]
    a contradiction. We conclude that $(L^\perp)^\perp\subseteq L$, so now $L=(L^\perp)^\perp$.\\[5pt]
    For the converse, suppose that $L=(L^\perp)^\perp$ and let $(\ell_j)_{j\geq 1}\subseteq L$ be such that $\ell_j\rightarrow\ell\in X$ as $j\rightarrow\infty$. But then, for all $y\in L^\perp$ and $j\geq 1$, $\ip{\ell_j}{y}=0$ since $\ell_j\in L=(L^\perp)^\perp$. Taking the limit:
    \begin{align*}
        0=\lim_{j\rightarrow\infty}\ip{\ell_j}{y}=\ip{\lim_{j\rightarrow\infty}\ell_j}{y}=\ip{\ell}{y}
    \end{align*}
    so $\ell\in (L^\perp)^\perp=L$. We conclude that any convergent sequence in $L$ has its limit in $L$, so $L$ is closed.\hfill{$\qed$} 
\end{document}