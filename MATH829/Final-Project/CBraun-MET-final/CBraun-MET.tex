\documentclass[11pt]{report}
\usepackage[margin=1.87cm]{geometry}

\usepackage{amsmath, amsfonts, amssymb, amsthm}
\usepackage[shortlabels]{enumitem}
\usepackage[makeroom]{cancel}
\usepackage{mathptmx}
\usepackage[T1]{fontenc}
\usepackage{graphicx}
\usepackage{bbm}
\usepackage{xcolor}
\usepackage{fancyhdr}
\usepackage{etoolbox}
\usepackage{titlesec}
\usepackage[bookmarks,hypertexnames=false,debug,linktocpage=true,hidelinks]{hyperref}

\titleformat{\section}[runin]
    {\normalfont\bfseries}{\thesection}{1em}{}
\renewcommand{\thesection}{\arabic{section}}

\renewcommand{\bibname}{\large\bfseries Bibliography} % For book/report class
% ======= Fancy Header after the first page ======
\pagestyle{fancy}
\fancyhead{} % Clears the standard fancy style
\fancyhead[L]{\normalfont\rightmark}%{\scshape\MakeLowercase{\rightmark}}
\fancyhead[R]{Connor Braun}%{\scshape\MakeLowercase{\rightmark}}
\fancyfoot[C]{\thepage}

\makeatletter
\renewcommand{\chaptermark}[1]{%
\markboth{%
    \ifnum\c@secnumdepth>\m@ne \@chapapp\ {\footnotesize\thechapter}. \ %
    \fi
#1%
}{}%
}
% ================================================

% ======= Hyperlinks for ToC and Bib ============= % Uses:
% <\usepackage[bookmarks,hypertexnames=false,debug,linktocpage=true,hidelinks]{hyperref}>
\hypersetup{colorlinks, linktoc=all, linkcolor={black}, citecolor={black},
urlcolor={black} }
% ================================================

\patchcmd{\thebibliography}{\section*{\refname}}{}{}{}

\newcommand{\bs}[1]{\boldsymbol{#1}} \newcommand{\mbb}[1]{\mathbb{#1}}
\newcommand{\mc}[1]{\mathcal{#1}}
\newcommand{\ra}[1]{\renewcommand{\arraystretch}{#1}}
\newcommand{\ip}[2]{\left\langle#1,#2\right\rangle }
\newcommand{\1}[1]{\mathbbm{1}_{\{#1\}}}

\DeclareMathOperator{\fix}{fix}
\DeclareMathOperator{\ran}{range}
\DeclareMathOperator{\proj}{Proj} 

\newtheorem{theorem}{Theorem}[section]
\newtheorem{lemma}{Lemma}[section]

\theoremstyle{definition}
\newtheorem{definition}{Definition}
\newtheorem{example}{Example}

\begin{document}\thispagestyle{empty}
    \begin{center}
        {\bf\large{Mean Ergodicity and Harmonic Analysis of Weakly Stationary
        Processes}}\\[5pt]
        by\\[5pt]
        Connor Braun\\[5pt]
        November, 2024
    \end{center}
    \vspace{5pt}
    \textbf{Acknowledgements} I would like to sincerely thank Professor Francesco Cellarosi for directing me to
    the book by Weber (\cite{Weber_2000}) and for our many illuminating conversations on ergodic theory and its applications.\\[5pt]
    \textbf{Abstract} [[Synopsis]]
    \section{Introduction}\label{sec0} At a glance, the old adage: "physical systems tend to
    equilibrium" (as codified by the laws of thermodynamics) appears quite banal
    and unassuming. However, if one recognizes equilibrium as a
    \textit{macroscopic} observation made of a system of very many interacting
    \textit{microscopic} entities, then the nontriviality of the statement
    begins to reveal itself; how can we understand the coexistence of a steady
    macrostate with a rapidly fluctuating microstate? This question remained of
    significant interest in early statistical mechanics from the late 19c. until
    circa 1930 when von Neumann \cite{Neumann_1932} and Birkhoff
    \cite{Birkhoff_1931} established the mean and pointwise ergodic theorems
    (respectively). These landmark results characterized macrostates as
    asymptotic averages over the temporal evolution of the microscopic system,
    and marked the origin of ergodic theory as a mathematical discipline. Over
    the intervening century, ergodic theory has been used to great effect across
    diverse subdisciplines of mathematics, ranging from stochastic processes
    \cite[theorem 17.0.1]{Meyn_Tweedie_1993} to number theory
    \cite{Furstenberg_1977}, \cite{Gorodnik_Nevo_2015}, \cite{Green_Tao_2008}
    and many more (see \cite{Eisner_Farkas_Haase_Nagel_2015} for a comprehensive
    overview). This wide applicability is perhaps surprising given the physical
    impetus for ergodic theory, and is the main motivation for this report. In
    particular, our goal is to highlight the applicability of mean-type ergodic
    theorems (such as von Neumann's) to diverse problems in mathematics by
    introducing weakly stationary processes on Hilbert spaces. Taking this
    space to be separable, we then showcase an early application to harmonic
    analysis, owing to \cite{Fan_1946}, yielding a sort of Fourier transform for
    a large class of dynamical systems.\\[5pt]
    \indent The document is organized as follows: in $\S$\ref{sec1},
    a high level overview of the conceptual basis, history and impetus for ergodic
    are provided from the perspective of classical statistical mechanics. In $\S$\ref{sec2},
    we define the Koopman operator and its fixed space in order to prove a generalized, operator-theoretic
    version of the mean ergodic theorem, of which von Neumann's is an immediate corollary.

    \section{Conceptual Background}\label{sec1}
    Following the expositions of \cite[ch.1]{Eisner_Farkas_Haase_Nagel_2015},
    and \cite[ch.2, $\S$5]{Reed_Simon_1972} we first examine the conceptual
    basis of ergodic theory, along with some relevant historical remarks. As a
    concrete, guiding example, consider a closed system of ideal gas consisting
    of $d\in\mbb{N}$ particles. From a classical mechanical perspective, this
    system is completely characterized by a state $x\in\mbb{R}^{6\times d}$,
    since each particle is bestowed with six degrees of freedom: position and
    velocity in a three-dimensional medium. The state space
    $\mbb{X}\subset\mbb{R}^{6\times d}$ are those configurations which are
    deemed admissable (e.g. particles cannot exceed the speed of light) and we
    take $x_0\in\mbb{X}$ to be some initial state, regarded as the configuration
    at time $t=0$. The laws of motion then define an operator
    $T:\mbb{X}\rightarrow\mbb{X}$ characterizing the flow of the system in
    discrete time: for $n\geq 0$, $x_0\mapsto x_n=T^nx_0$, where $T^0:=\mbb{I}$
    is the identity operator. The resulting process
    $\{x_n\}_{n\geq0}\subset\mbb{X}$ is called the system \textit{trajectory},
    and the pair $(\mbb{X},T)$ is referred to as a \textit{dynamical
    system}.\\[5pt]
    \indent At this point we face two practical limitations. First, it is not
    possible to determine $x_n$ for any $n\geq 0$ precisely, ($N$ could be on
    the order of $10^{23}$ for even a small quantity of gas) so we instead fix
    an observable $f\in L^2(\mbb{X})$ so that $f:(\mbb{X},T)\rightarrow\mbb{R}$
    (perhaps $f(x)$ returns a temperature reading associated to a state
    $x\in\mbb{X}$) inducing a new trajectory $\{f(x_n)\}_{n\geq 0}$. Second, for
    $n\geq 0$ we assume that $x_n\mapsto x_{n+1}$ spans an extremely short
    duration $\Delta>0$, rendering the individual measurements $f(x_n)$
    impossible to resolve. Instead, the probe observes an average over $N\geq 1$
    timesteps:
    \begin{align}
        \mu_N^T(f,x_0):=\frac{1}{N}\sum_{n=0}^{N-1}f(T^nx_0)\label{eq1}
    \end{align}
    and we might intuitively expect the \textit{time mean}
    $\lim_{N\rightarrow\infty}\mu^T_N(f,x_0)$, assuming it exists, to furnish
    some notion of equilibrium at the macroscale. Of course, (\ref{eq1}) is not
    yet satisfactory since it still depends on an $f(x_0)$. This brings us to
    1885, when Boltzmann proposed the the so-called \textit{Ergodenhypothese}
    (ergodic hypothesis) \cite{Hasenöhrl_Boltzmann_2012} (see
    \cite{Boltzmann_2019} for a translation). This was a sufficient condition on
    the dynamical system $(\mbb{X},T)$ to deduce the existence of a functional
    $\mu^T:L^2(\mbb{X})\rightarrow\mbb{R}$ such that
    \begin{align}
        \lim_{N\rightarrow\infty}\mu_N^T(f,x_0)=\mu^T(f)\label{eq2}
    \end{align}
    which is now independent of the initial state $x_0$. Informally, the number
    $\mu^T(f)$ turns out to be a sort of weighted average of $f$ over $\mbb{X}$,
    with the weights depending on the dynamics $T$ (i.e., accounting for how $T$
    might bias $\{x_n\}_{n\geq 0}$ towards specific regions of $\mbb{X}$) which
    justifies the catchphrase of ergodic theory: "the time mean equals the space
    mean". Boltzmann's hypothesis later underwent a technical reformulation,
    being redubbed the quasi-ergodic hypothesis, which explains the title of von
    Neumann's publication \cite{Neumann_1932}. 

    \section{The Hilbert Space Mean Ergodic Theorem}\label{sec2}
    In this section, we transition to the domain of functional analysis and
    prove the main result of \cite{Neumann_1932} as an immediate corollary of a
    slightly more general result. For this we follow the elegant proof presented
    in \cite[theorem 1.3.1]{Weber_2000} while blending the notation found there
    with that of \cite[ch.8]{Eisner_Farkas_Haase_Nagel_2015}. The principle
    structural difference between mean ergodic theorems and so-called
    \textit{pointwise} ergodic theorems (such as that of Birkhoff,
    \cite{Birkhoff_1931}) is that we study the asymptotic behavior of an
    observation functional (combining $T$ and $f$ from the previous section) in
    emancipation from the subjacent state space $\mbb{X}$ (as opposed to the
    trajectory $\{f(x_n)\}_{n\geq 0}$ directly). In what follows, let $T$ be a
    mapping defining a dynamical system. The main object permitting us to employ
    Hilbert space methods is the so-called {\it Koopman operator} associated to
    $T$.
    \begin{definition}[Koopman Operator]\label{def1} Let $T$ be a discrete
        dynamical flow, and $\mc{H}$ a Hilbert space. The linear operator
        $U_T:\mc{H}\rightarrow \mc{H}$ defined by $U_Tf:=f\circ T$ for all $f\in
        \mc{H}$ is called the \textit{Koopman operator} associated to $T$.
    \end{definition}
    This definition implicitly requires some regularity of $T$, which we shall
    clarify in the sequel. For a Koopman operator $U_T$, its \textit{fixed
    space} is defined
    \[\fix(U_T):=\{f\in\mc{H}:U_Tf=f\}=\ker(\mbb{I}-U_T)\] which is closed since
    $\ker(\mbb{I}-U_T)=(\ran(\mbb{I}-U_t^\ast))^{\perp}$. Note that $\fix(U_T)$ is nothing more than the eigenspace
    of $U_T$ corresponding to eigenvalue $1$. We shall denote
    $\mc{H}_T:=\fix(U_T)$ for notational expediency. These objects are
    sufficient to state the main result, which can be found in \cite[theorem
    8.6]{Eisner_Farkas_Haase_Nagel_2015}.
    \begin{theorem}[Mean Ergodic Theorem {\cite[theorem
        8.6]{Eisner_Farkas_Haase_Nagel_2015}}, {\cite[theorem
        1.3.1]{Weber_2000}}]\label{thm1} Let $\mc{H}$ be a Hilbert space, and
        $U:\mc{H}\rightarrow\mc{H}$ a bounded linear operator, denoted by writing
        $U\in(\mc{B}(\mc{H}),\|\cdot\|_{op})$. If $\|U\|_{op}\leq 1$, then
        \begin{align*}
            \overline{f}:=\lim_{N\rightarrow\infty}\frac{1}{N}\sum_{n=0}^{N-1}U^nf\quad\text{exists for all $f\in\mc{H}$.}
        \end{align*}
        Further, the map $P_{U}:f\rightarrow\overline{f}$ is the orthogonal
        projection of $f$ onto $\fix(U)$, and
        $\mc{H}=\fix(U)\oplus\overline{\ran}(\mbb{I}-U)$.
    \end{theorem}
    We say that $U$ is a \textit{contraction} if and only if $\|U\|_{op}\leq 1$. Note that this
    is weaker than the notion of contraction in the Banach fixed point theorem, which calls for $\|U\|_{op}<1$ instead.
    In what follows we can think of $U$ as the Koopman operator associated to $T$, writing $P_T:=P_{U}$ and $\mc{H}_T=\fix(U)$, but this need not be the case, in general.
    The rest of the section is dedicated to the proof of
    this theorem, following \cite[theorem 1.3.1]{Weber_2000} for which we induce
    the \textit{Ces\`aro averages}:
    \[C_N:=\frac{1}{N}\sum_{n=0}^{N-1}U^n,\quad N\geq 0.\]
    \begin{lemma}[{\cite[lemma 1.3.2]{Weber_2000}}]\label{lem4}
        Suppose $U$ is a contraction on a Hilbert space $\mc{H}$. Then
        $\fix(U^\ast)=\fix(U)$. 
    \end{lemma}
    \noindent{\textbf{Proof.}} We first show that $U^\ast$ is also a
    contraction. Let $\varphi:\mc{H}\rightarrow\mc{H}^\prime$ be the Riesz
    isometry, and recall that $U^\ast=\varphi^{-1}\circ U^\prime\circ\varphi$,
    where $(U^\prime\varphi(f))(g):=\ip{Ug}{f}$ for any $f,g\in\mc{H}$. Thus,
    for $f\in\mc{H}$,
    \begin{align}
        \|U^\ast f\|=\|(\varphi^{-1}U^\prime\varphi)(f)\|=\|U^\prime\varphi(f)\|_\ast=\sup_{\|g\|=1}|\ip{Ug}{f}|\leq\sup_{\|g\|=1}\|Ug\|\|f\|=\|U\|_{op}\|f\|\label{eq6}
    \end{align}
    where $\|\cdot\|_\ast$ is the norm on the dual $\mc{H}^\prime$, and the second equality holds since $\varphi^{-1}$ is an isometry. From (\ref{eq6}), we obtain
    \[\|U^\ast\|_{op}=\sup_{\|f\|=1}\|U^\ast f\|\leq\|U\|_{op}\leq 1\]
    and thus $U^\ast$ is a contraction. Now, notice that if $f\in\fix(U)$, then $\ip{Uf}{f}=\ip{f}{Uf}=\|f\|^2$, and
    $\ip{f}{U^\ast f}=\ip{U^\ast f}{f}=\|f\|^2$. Conversely, if $f\in\mc{H}$
    satisfies $\ip{f}{Uf}=\ip{Uf}{f}=\|f\|^2$, we have
    \begin{align*}
        \|Uf-f\|^2=\ip{Uf-f}{Uf-f}=\|Uf\|^2+\|f\|^2-\ip{f}{Uf}-\ip{Uf}{f}=\|Uf\|^2+\|f\|^2-2\|f\|^2\leq \|f\|^2-\|f\|^2=0
    \end{align*}
    where the inequality is due to $U$ being a contraction. We conclude that
    $Uf=f$, and therefore:
    \begin{align*}
        Uf=f\longleftrightarrow \ip{Uf}{f}=\ip{f}{Uf}=\|f\|^2\longleftrightarrow \ip{U^\ast f}{f}=\ip{f}{U^\ast f}{f}=\|f\|^2\longleftrightarrow U^\ast f=f.\tag*{$\qed$}
    \end{align*}
    \noindent\textbf{Proof} (Theorem \ref{thm1}, \cite[theorem 1.3.1]{Weber_2000})\textbf{.} With
    $\mc{H}_T:=\fix(U)$ and $\mc{H}_0=\overline{\ran}(\mbb{I}-U)$, we claim that
    $\mc{H}=\mc{H}_T\oplus\mc{H}_0$. First, if $f\in\mc{H}_T$ and $g\in\mc{H}$,
    \begin{align*}
        \ip{f}{g-Ug}=\ip{f}{g}-\ip{f}{Ug}=\ip{f}{g}-\ip{U^\ast f}{g}=0
    \end{align*}
    since $f\in\fix(U^\ast)$ by lemma \ref{lem4}, and so
    $\mc{H}_T\subseteq(\ran(\mbb{I}-U))^\perp$. If instead
    $f\in(\ran(\mbb{I}-U))^\perp$, then
    \[0=\ip{f}{g-Ug}=\ip{f}{g}-\ip{f}{Ug}=\ip{f}{g}-\ip{U^\ast
    f}{g}=\ip{f-U^\ast f}{g}\] which holds $\forall g\in\mc{H}$, so $U^\ast f=f$
    and $f\in\mc{H}_T$. Thus, we have $\mc{H}_T=(\ran(\mbb{I}-U))^\perp$ and
    further $\mc{H}=\mc{H}_T\oplus\mc{H}_0$, as claimed. Now consider
    $h\in\mc{H}_T\oplus \ran(\mbb{I}-U)$, which we have just shown is dense in
    $\mc{H}$. We can find $f\in\mc{H}_T$, $g\in\mc{H}$ so that $h=f+g-Ug$, from
    which it is clear that $P_Th$ exists, since
    \begin{align}
        \frac{1}{N}\sum_{n=0}^{N-1}U^n(f+g-Ug)=\frac{1}{N}\sum_{n=0}^{N-1}U^nf+\frac{1}{N}\sum_{n=0}^{N-1}(U^ng-U^{n+1}g)=f+\frac{1}{N}(g-U^Ng)\overset{N\rightarrow\infty}{\longrightarrow} f.\label{eq3}
    \end{align}
    Further, $\proj_{\mc{H}_T}h=f=P_Th$. Note that in (\ref{eq3}) we have also
    shown that $P_Tf=f$ for all $f\in\mc{H}_T$, and $C_N(g-Ug)\rightarrow 0$ for
    all $g\in\mc{H}$. To complete the proof, fix $f\in\mc{H}$, $\varepsilon>0$,
    and a sequence $\{f_n\}_{n\geq 1}\subseteq\mc{H}_T\oplus\ran(\mbb{I}-U)$ so
    that $f_n\rightarrow f$. For $n\geq 1$, we can find $g_n\in\mc{H}_T$,
    $h_n\in\ran(\mbb{I}-U)$ so that $f_n=g_n+h_n-Uh_n$, and 
    \begin{align*}
        f_n=g_n+h_n-Uh_n\longrightarrow g+h-Uh=f
    \end{align*}
    where $g\in\mc{H}_T$ (since $\mc{H}_T$ is closed) and
    $h-Uh\in\overline{\ran}(\mbb{I}-U)=\mc{H}_0$. We can find $n\geq 1$
    sufficiently large such that
    $\|(h-Uh)-(h_n-Uh_n)\|\leq\frac{\varepsilon}{2}$, after which we take $N\geq
    1$ large enough such that $\|C_N(h_n-Uh_n)\|\leq\frac{\varepsilon}{2}$.
    Then,
    \begin{align*}
        \|C_N\proj_{\mc{H}_0}f\|\leq\|C_N[(h-Uh)-(h_n-Uh_n)]\|+\|C_N(h_n-Uh_n)\|\leq \varepsilon
    \end{align*}
    since $C_N$ is a contraction. This means that
    $C_N\proj_{\mc{H}_0}f\rightarrow 0$, (i.e., $P_T\proj_{\mc{H}_0}f=0$) from
    which we finally obtain that for all $f\in\mc{H}$:
    \begin{align*}
        P_Tf=P_T(\proj_{\mc{H}_T}f+\proj_{\mc{H}_0}f)=\proj_{\mc{H}_T}f.\tag*{$\qed$}
    \end{align*}
    \indent Perhaps the most natural question following this is on the nature of
    $\mc{H}_T$; for which $f\in\mc{H}$ should we expect to have $Uf=f$? Returning to the
    physical case where $U_T$ is the Koopman operator and $\mc{H}=L^2(\mbb{X})$,
    the answer is determined entirely by $T$. However, for any constant function
    $f\in L^2(\mbb{X})$ and $x\in\mbb{X}$, we have $U_Tf(x)=(f\circ T)(x)=f(x)$
    regardless of $T$. In light of theorem \ref{thm1}, one can therefore regard
    constant functions as equilibrium observations in the sense that what one
    "observes" via $\{f(x_n)\}_{n\geq 1}$ will converge over time, regardless of
    where the system began. This discussion motivates the following definition.
    \begin{definition}[Ergodicity]\label{def2} We call $T$ \textit{ergodic} if
        and only if $\mc{H}_T=\{f\in\mc{H}:\;\text{$f$ is constant}\}$.
    \end{definition}
    It turns out that $T$ is ergodic if and only if any $A\subseteq\mbb{X}$
    satisfying $T^{-1}A=A$ is either vanishingly small or nearly the entire
    space -- both of which are formalized measure-theoretically \cite[$\S$II.5,
    p.59]{Reed_Simon_1972}. Let us now illustrate this concept with a relatively
    intuitive application of theorem \ref{thm1}.
    \begin{example}[Rotations on the Unit Sphere in $\mbb{C}$]
        Let $\mbb{X}:=\{z\in\mbb{C}:|z|=1\}=\{e^{2\pi xi}:x\in[0,1)\}$. For
        $\alpha\in[0,1)$, consider dynamics $T_\alpha:\mbb{X}\rightarrow\mbb{X}$
        such that $T_\alpha z=e^{2\pi\alpha i}z$. That is, $T_\alpha$ rotates
        $\mbb{X}$ by $2\pi\alpha$ radians. Take $\mc{H}:=L^2(\mbb{X})$ the
        Hilbert space of observables, and define the Koopman operator
        $U_\alpha:= U_{T_\alpha}:\mc{H}\rightarrow\mc{H}$ as in definition
        \ref{def1}. Then for $f,g\in\mc{H}$, we have
        \begin{align*}
            \ip{U_\alpha f}{U_\alpha g}=\int_\mbb{X}(U_\alpha f)(z)\overline{(U_\alpha g)(z)}dz&=\int_0^1f(e^{2\pi(x+\alpha)i})\overline{g(e^{2\pi(x+\alpha)i})}dx\\
            &=\int_\alpha^{1+\alpha}f(e^{2\pi ti})\overline{g(e^{2\pi ti})}dt\\
            &=\int_{0}^{1}f(e^{2\pi ti})\overline{g(e^{2\pi ti})}dt+\int_{1}^{1+\alpha}f(e^{2\pi ti})\overline{g(e^{2\pi ti})}dt-\int_{0}^\alpha f(e^{2\pi ti})\overline{g(e^{2\pi ti})}dt
        \end{align*}
        where this last term is just $\ip{f}{g}$, since
        $\int_{1}^{1+\alpha}f(e^{2\pi ti})\overline{g(e^{2\pi
        ti})}dt=\int_{0}^\alpha f(e^{2\pi ti})\overline{g(e^{2\pi ti})}dt$.
        This, along with the fact that $U_\alpha$ is invertible (with bounded
        inverse $U^{-1}_\alpha:=U_{-\alpha}$) means $U_\alpha$ is unitary, so
        $\|U\|_{op}=1$. Then by theorem \ref{thm1},
        \begin{align*}
            \frac{1}{N}\sum_{n=0}^{N-1}U^n_\alpha f\longrightarrow\proj_{\mc{H}_{T_\alpha}}f
        \end{align*}
        where $\mc{H}_{T_\alpha}:=\fix(U_\alpha)=\{g\in\mc{H}:g(e^{2\pi\alpha
        i}z)=g(z)\;\forall z\in\mbb{X}\}$, the set of observables which are
        invariant under rotations of $2\pi\alpha$ radians. It turns out that
        $T_\alpha$ is ergodic if and only if $\alpha\in[0,1)\setminus\mbb{Q}$.
        For a complete proof of this fact, we refer the reader to
        \cite[proposition 7.15]{Eisner_Farkas_Haase_Nagel_2015}. For some
        intuition as to why $T_\alpha$ is not ergodic when $\alpha$ is rational,
        let $\alpha=\frac{1}{2}$ and consider $h\in\mc{H}$ given by
        $h(z)=\1{\arg(z)\in[0,\frac{1}{2}\pi)}+\1{\arg(z)\in[\pi,\frac{3}{2}\pi)}$.
        Then $h\in\mc{H}_{T_\alpha}$, since $\arg(T_\alpha z)=\arg(e^{\pi
        i}z)\in[\pi,\frac{3}{2}\pi)$ $\forall z\in\mbb{X}$ with
        $\arg(z)\in[0,\frac{1}{2}\pi)$, and vise versa, but $h$ is not constant
        on $\mbb{X}$.
    \end{example}
    We conclude this section by noting that the statement of von Neumann's mean ergodic theorem \cite{Neumann_1932} is identical to theorem \ref{thm1},
    but specifically for Koopman operators $U_T$ associated to $T$ from so-called \textit{measure preserving} dynamical systems $(\mbb{X},T)$. That this is an immediate corollary of theorem \ref{thm1} follows that fact that if $T$ is measure preserving,
    then $U_T$ is unitary \cite[Koopman's lemma; p.57]{Reed_Simon_1972}. In particular, $\|U_T\|_{op}=1$, so theorem \ref{thm1} applies.\\[5pt]
    \indent Having sidestepped the foundational notions of measure-theoretic ergodic theory, we proceed to some applications of perhaps more esoteric mathematical intrigue.
    \section{Weakly Stationary Sequences}\label{sec3} In  \S\ref{sec2} we
    found that any contraction $U$ on a Hilbert space $\mc{H}$ is mean ergodic
    in the sense that $P_U:f\mapsto \overline{f}$ exists for all $f\in\mc{H}$.
    Unitary operators are a class of such contractions, since any unitary
    $U:\mc{H}\rightarrow\mc{H}$ satisfies $\|U\|_{op}=1$. Recall that $U$ is
    \textit{unitary} if it is invertible, linear, and satisfies
    $\ip{Uf}{Ug}=\ip{f}{g}$ for all $f,g\in\mc{H}$, and that this latter
    property means $U$ is an \textit{isometry}. It turns out these characterize
    a large class of $\mc{H}$-valued processes, which are known as \textit{weakly
    stationary} (or \textit{wide-sense stationary} in the context of stochastic
    processes -- see \cite[ch.15]{Koralov_Sinai_2007}).
    \begin{definition}[Weakly Stationary Process]\label{def3} Let $\mc{H}$ be a
        Hilbert space. A sequence $\{f_n\}_{n\geq 1}\subset\mc{H}$ is called
        \textit{weakly stationary} (WS), or \textit{second-order stationary} if for
        any $n,m,k\in\mbb{N}$, $\ip{f_{n+k}}{f_{m+k}}=\ip{f_n}{f_m}$.
    \end{definition}
    It is readily apparent that for any unitary $U:\mc{H}\rightarrow\mc{H}$ and
    $f_0\in\mc{H}$, the sequence $\{f_n\}_{n\geq 1}$ defined by $f_n:=U^nf_0$ is
    WS, since
    $\ip{f_{n+k}}{f_{m+k}}=\ip{U^kf_n}{U^kf_m}=\ip{(U^\ast)^kU^kf_n}{f_m}=\ip{f_n}{f_m}$.
    It turns out that the converse is true as well, for which we require the
    following lemma, the proof for which can be found in \cite[lemma 2.1.2]{Weber_2000}.
    \begin{lemma}[Isometry Extension {\cite[lemma 2.1.2]{Weber_2000}}]\label{lem5}
        Let $\mc{H}$ be a Hilbert space, $M\subseteq\mc{H}$, and
        $U:M\rightarrow\mc{H}$ an isometry. Then $U$ can be extended to an
        isometric mapping on the closed subspace $\mc{H}_M\subseteq\mc{H}$
        spanned by $M$.  
    \end{lemma}
    \noindent\textbf{Proof.} Let $\mc{L}(M)$ be the set of linear combinations
    from $M$, so that for any $f\in\mc{L}(M)$, $\exists m\geq 1$ and
    $f_1,f_2,\dots,f_m\in M$ such that $f=\sum_{j=1}^m\alpha_j f_j$ for some
    scalars $\alpha_j$, $1\leq j\leq m$. Define a new operator
    $\hat{U}:\mc{L}(M)\rightarrow\mc{H}$ by
    $\hat{U}f:=\sum_{j=1}^m\alpha_jUf_j$. This map is well-defined since if
    $f=\sum_{j=1}^m\beta_jf_j$ for another set of scalars $\beta_j$, $1\leq
    j\leq m$, we write $\gamma_j:=\alpha_j-\beta_j$ and find
    \begin{align*}
        \left\|\sum_{j=1}^m\gamma_jUf_j\right\|^2=\ip{\sum_{j=1}^m\gamma_jUf_j}{\sum_{j=1}^m\gamma_jUf_j}=\sum_{j=1}^m\sum_{k=1}^m\gamma_j\overline{\gamma_j}\ip{Uf_j}{Uf_k}=\sum_{j=1}^m\sum_{k=1}^m\gamma_j\overline{\gamma_j}\ip{f_j}{f_k}=\left\|\sum_{j=1}^m\gamma_jf_j\right\|^2=0.
    \end{align*}
    By precisely the same manipulations, we may take another linear combination
    $g$ and find that $\ip{\hat{U}f}{\hat{U}g}=\ip{f}{g}$, so $\hat{U}$ is an
    isometry. It is also clear that $\hat{U}$ is linear. Any element
    $f\in\mc{H}_M$ can be written $f=\lim_{n\rightarrow\infty}f_n$, where $f_n$
    are each linear combinations of $M$, and $\{f_n\}_{n\geq 1}$ is Cauchy. That
    is,
    \begin{align*}
        0=\lim_{n,m\rightarrow\infty}\|f_n-f_m\|^2=\lim_{n,m\rightarrow\infty}\ip{f_n-f_m}{f_n-f_m}=\lim_{n,m\rightarrow\infty}\ip{\hat{U}f_n-\hat{U}f_m}{\hat{U}f_n-\hat{U}f_m}=\lim_{n,m\rightarrow\infty}\|\hat{U}f_n-\hat{U}f_m\|^2
    \end{align*}
    so $\{\hat{U}f_n\}_{n\geq 1}$ is Cauchy too, and we may define
    $\widetilde{U}f:=\lim_{n\rightarrow\infty}\hat{U}f_n$ for any such
    $f\in\mc{H}_M$. This new operator $\widetilde{U}:\mc{H}_M\rightarrow\mc{H}$
    is also well-defined, and remains an isometry by continuity of the inner
    product.\hfill{$\qed$}
    \begin{theorem}[{\cite[theorem 2.1.3]{Weber_2000}}]\label{thm2} Let
        $F=\{f_n\}_{n\geq 1}\subseteq\mc{H}$ be weakly stationary, and
        $\mc{H}_F\subseteq\mc{H}$ the closed subspace spanned by $F$. Then there
        exists a unitary $U:\mc{H}_F\rightarrow\mc{H}_F$ such that
        $Uf_n=f_{n+1}$ for $n\geq 1$.
    \end{theorem}
    \noindent\textbf{Proof.} As suggested, define $Uf_n:=f_{n+1}$ for $n\geq 1$.
    This is an isometry, since for $n,m\geq 1$,
    $\ip{Uf_n}{Uf_m}=\ip{f_{n+1}}{f_{m+1}}=\ip{f_n}{f_m}$, by the weak
    stationarity of $\{f_n\}_{n\geq 1}$. In light of lemma \ref{lem5}, $U$
    extends to a linear isometry on $\mc{H}_F$, which we shall also denote with
    $U$. Thus we need only show that $U$ is surjective. Fixing $g\in\mc{H}_F$,
    $\exists \{g_n\}_{n\geq 1}\subseteq\mc{L}(F)$ so that
    $\lim_{n\rightarrow\infty}g_n=g$. In particular,
    \begin{align*}
        g_n=\sum_{j=1}^{m_n}\alpha_{n,j}f_{k^n_j},\quad\text{so}\quad g_n=U\left(\sum_{j=1}^{m_n}\alpha_{n,j}f_{k^n_j-1}\right)=:Ug^\prime_n,\quad\text{with}\quad g^\prime_n\in\mc{L}(F),\quad n\geq 1,
    \end{align*}
    where $\{k^n_j\}_{j\geq 1}\subseteq\mbb{N}$ is a subsequence depending on
    $n$. Using the same trick as for lemma \ref{lem5}, $\{g_n^\prime\}_{n\geq
    1}$ is Cauchy, so taking $g^\prime:=\lim_{n\rightarrow\infty}g^\prime_n$ we
    have
    $Ug^\prime=\lim_{n\rightarrow\infty}Ug^\prime_n=\lim_{n\rightarrow\infty}g_n=g$.
    \hfill{$\qed$}\\[5pt]
    \indent Some technical remarks on theorem \ref{thm2} are germane to our
    purposes. For a WS process $F=\{f_n\}_{n\geq 1}\subseteq\mc{H}$, we refer to
    the associated unitary operator $U$ as its \textit{dynamics}, in analogy to
    the application from \S \ref{sec0}. Further, assuming the dynamics
    $U:\mc{H}\rightarrow\mc{H}$ to be unitary is without any loss of generality,
    since otherwise we may regard $F$ as $\mc{H}_F$-valued, where $\mc{H}_F$ is
    a Hilbert space in its own right and we are guaranteed
    $U:\mc{H}_F\rightarrow\mc{H}_F$ is unitary. Lastly, we may recharacterize WS
    processes as those satisfying $\ip{f_n}{f_m}=\Gamma(n-m)$ for $n\geq m\geq
    0$ and some function $\Gamma$. Indeed, when this is the case we get
    \begin{align*}
        \ip{f_{n+k}}{f_{m+k}}=\Gamma(n+k-m-k)=\Gamma(n-m)=\ip{f_n}{f_m}.
    \end{align*}
    To handle the case where $n<m$, simply define
    $\Gamma(n-m)=\overline{\Gamma(m-n)}$. Conversely, if we begin with a WS
    sequence having dynamics $U$, then
    $\ip{f_n}{f_m}=\ip{U^nf_0}{U^mf_0}=\ip{U^{n-m}f_0}{f_0}=:\Gamma(n-m)$ since
    $U^\ast=U^{-1}$. For example, any orthonormal system
    $\{e_n\}_{n\in\mbb{Z}}\subseteq\mc{H}$ is WS, since
    $\ip{e_n}{e_m}=\delta_{n-m}$. A similarly trivial example is
    $\{e^{-in\lambda}f_n\}_{n\geq 1}$ where we set
    $\lambda\in\mbb{T}:=\{x\mod2\pi:x\in\mbb{R}\}$, provided $\{f_n\}_{n\geq 1}$
    is WS to begin with. Indeed,
    $\ip{e^{-in\lambda}f_n}{e^{-im\lambda}f_m}=e^{-i(n-m)\lambda}\ip{f_n}{f_m}$.
    Simple though it may be, this observation, in conjunction with the mean
    ergodic theorem, enables a sort of Fourier-type representation for any WS
    process.
    \section{Harmonic Decomposition of Weakly Stationary Processes}\label{sec4}
    Let $\mbb{T}:=\{x\mod2\pi:x\in\mbb{R}\}$ and $F=\{f_n\}_{n\geq
    1}\subseteq\mc{H}$ be WS with unitary dynamics $U$, where $f\in\mc{H}$ is
    such that for any $n\geq 1$, $f_n=U^nf$. Taking $\lambda\in\mbb{T}$, we
    shall sometimes write $U_\lambda=e^{-i\lambda}U$ to denote the dynamics of
    the \textit{twisting process} $\{e^{-in\lambda}f_n\}_{n\geq 1}$, which, as
    we previously remarked, is also WS. Our objective is to represent each
    element of $F$ in terms of the mean ergodic projections $P_{U_\lambda}f$,
    which we denote with an alternative notation:
    \begin{align}
        \Lambda(f,U,\lambda):=\lim_{N\rightarrow\infty}\frac{1}{N}\sum_{n=1}^N e^{-in\lambda}U^nf,\quad\lambda\in\mbb{T}
    \end{align}
    which exists for all $\lambda\in\mbb{T}$, $f\in\mc{H}$ by theorem \ref{thm1}.
    For this, we shall follow the steps found in\cite[\S6]{Fan_1946} closely. Our first observation is
    that the inner product of WS $f_n$ with its mean ergodic projection is constant in $n\geq 1$.
    \begin{lemma}[{\cite[corollary 4]{Fan_1946}}]\label{lem6}
        Let $U:\mc{H}\rightarrow\mc{H}$, be unitary and $f\in\mc{H}$ so that
        $\{U^nf\}_{n\geq 1}$ is WS. Then
        $\ip{\Lambda(f,U,0)}{f_n}=\ip{\Lambda(f,U,0)}{f_m}$ for all $n,m\geq 1$. Further, $\ip{\Lambda(f,U,0)}{f_n}=\ip{f_n}{\Lambda(f,U,0)}=\|\Lambda(f,U,0)\|^2$ for any $n\geq 1$.
    \end{lemma}
    \noindent\textbf{Proof.} Observe that $\Lambda(f,U,0)=P_Uf\in\fix(U)$, so
    $UP_Uf=P_Uf$. Of course, this means that
    \begin{align*}
        \left\|P_Uf-\frac{1}{N}\sum_{n=2}^{N+1}U^nf\right\|=\left\|UP_Uf-U\left(\frac{1}{N}\sum_{n=1}^{N}U^nf\right)\right\|=\left\|P_Uf-\frac{1}{N}\sum_{n=1}^NU^nf\right\|\overset{N\rightarrow\infty}{\longrightarrow} 0
    \end{align*}
    since $U$ is unitary. Repeating this, we obtain
    $P_Uf=\lim_{N\rightarrow\infty}\frac{1}{N}\sum_{n=m}^{N+m-1}U^nf$ for $m\geq1$
    by induction. For $n,m\geq 1$,
    \begin{align*}
        \ip{\Lambda(f,U,0)}{f_n}=\lim_{N\rightarrow\infty}\frac{1}{N}\sum_{k=n}^{N+n-1}\ip{f_k}{f_n}=\lim_{N\rightarrow\infty}\frac{1}{N}\sum_{k=m}^{N+m-1}\ip{f_k}{f_m}=\ip{\Lambda(f,U,0)}{f_m}
    \end{align*}
    which follows from the weak stationarity of $\{f_n\}_{n\geq
    1}$. Using this fact, we can show the second claim:
    \begin{align*}
        \ip{\Lambda(f,U,0)}{f_n}=\frac{1}{N}\sum_{k=1}^N\ip{\Lambda(f,U,0)}{f_k}=\ip{\Lambda(f,U,0)}{C_Nf}
    \end{align*}
    such that
    $\|\Lambda(f,U,0)\|^2=\lim_{N\rightarrow\infty}\ip{\Lambda(f,U,0)}{C_Nf}=\ip{\Lambda(f,U,0)}{f_n}$,
    regardless of $n\geq 1$.\hfill{$\qed$}\\[5pt]
    \indent For a twisting process, the mean ergodic projection satisfies
    $\Lambda(f,U,\lambda)=\Lambda(f,U_\lambda,0)$ such that
    $\ip{\Lambda(f,U,\lambda)}{e^{-in\lambda}f_n}=\|\Lambda(f,U,\lambda)\|^2$
    for any $n\geq 1$ by lemma \ref{lem6}. This result can be used to determine
    the inner product between the mean ergodic projection between two twisting
    processes. With $\mu,\lambda\in\mbb{T}$, we have
    \begin{align}
        \ip{\Lambda(f,U,\lambda)}{\Lambda(f,U,\mu)}=\lim_{N\rightarrow\infty}\frac{1}{N}\sum_{n=1}^N\ip{e^{-in\lambda}f_n}{\Lambda(f,U,\mu)}&=\lim_{N\rightarrow\infty}\frac{1}{N}\sum_{n=1}^Ne^{-in(\lambda-\mu)}\ip{e^{-in\mu}f_n}{\Lambda(f,U,\mu)}\notag\\
        &=\|\Lambda(f,U,\mu)\|^2\left(\lim_{N\rightarrow\infty}\frac{1}{N}\sum_{n=1}^Ne^{-in(\lambda-\mu)}\right)\label{eq4}
    \end{align}
    such that $\ip{\Lambda(f,U,\lambda)}{\Lambda(f,U,\mu)}=0$ whenever
    $\lambda\neq\mu$. Thus, we have obtained an uncountable orthogonal family
    $\{\Lambda(f,U,\lambda)\}_{\lambda\in\mbb{T}}\subseteq\mc{H}$. However,
    assuming $\mc{H}$ to be separable (as we henceforth do) forces all but at
    most countably many of these to be zero, allowing us to set
    $\{\lambda\in\mbb{T}:\Lambda(f,U,\lambda)\neq
    0\in\mc{H}\}=\{\lambda_j\}_{j=1}^\infty$. Extracting these $\lambda_j$ is analogous
    to the discrete time Fourier transform of a signal, motivating the following definition.
    \begin{definition}[Fourier Series for WS Processes {\cite[\S6]{Fan_1946}}]\label{def4}
        With $\{\lambda_j\}_{j=1}^\infty$ as above, we call $\{\sum_{j=1}^\infty e^{in\lambda_j}\Lambda(f,U,\lambda_j)\}_{n\geq 1}$
        the \textit{Fourier series} corresponding to $\{f_n\}_{n\geq 1}$.
    \end{definition}
    Foremost is the technical question of whether the series $\sum_{j=1}^\infty
    e^{-in\lambda_j}\Lambda(f,U,\lambda_j)$ converges for $n\geq 1$, but also to
    determine how these might be used to represent $f_n$. To address these, define
    \begin{align}
        P^N_n:=\sum_{j=1}^Ne^{in\lambda_j}\Lambda(f,U,\lambda_j),\quad\text{and}\quad Q^N_n:=f_n-P^N_n,\quad n,N\geq 1
    \end{align}
    yielding the decomposition $f_n=P^N_n+Q^N_n$ for $n\geq 1$. The following theorem addresses the behavior of this
    representation as $N\rightarrow\infty$.
    \begin{lemma}[{\cite[\S6]{Fan_1946}}]\label{lem7}
        Fix any $N\geq 1$, and let $\{f_n\}_{n\geq 1}$, $\{P^N_n\}_{n\geq 1}$ and $\{Q^N_n\}_{n\geq 1}$ be as above. 
        Then for any $n,m\geq 1$, $\ip{P^N_n}{Q^N_m}=0$, and we have both
        \begin{align*}
            \ip{f_n}{f_m}=\ip{P^N_n}{P^N_m}+\ip{Q^N_n}{Q^N_m},\quad\text{and}\quad\ip{P^N_n}{P^N_m}=\sum_{j=1}^N e^{i(n-m)\lambda_j}\Lambda(f,U,\lambda_j).
        \end{align*}
        In particular, both $\{P^N_n\}_{n\geq 1}$ and $\{Q^N_n\}_{n\geq 1}$ are WS.
    \end{lemma}
    \noindent\textbf{Proof.} Using (\ref{eq4}), for $j,n\geq1$ we obtain
    \begin{align*}
        \ip{\Lambda(f,U,\lambda_j)}{P^N_n}=\sum_{k=1}^N\ip{\Lambda(f,U,\lambda_j)}{e^{in\lambda_k}\Lambda(f,U,\lambda_k)}=\begin{cases}
            e^{-in\lambda_j}\|\Lambda(f,U,\lambda_j)\|^2,\quad&\text{if $j\leq N$}\\
            0,\quad&\text{otherwise.}
        \end{cases}
    \end{align*}
    This, along with $\ip{\Lambda(f,U,\lambda_j)}{f_n}=e^{-in\lambda_j}\|\Lambda(f,U,\lambda_j)\|^2$ (which uses \ref{lem6}) immediately implies
    \begin{align*}
        \ip{\Lambda(f,U,\lambda_j)}{Q^N_n}=\ip{\Lambda(f,U,\lambda_j)}{f_n}-\ip{\Lambda(f,U,\lambda_j)}{P^N_n}=\begin{cases}
            0,\quad&\text{if $j\leq N$}\\
            e^{-in\lambda_j}\|\Lambda(f,U,\lambda_j)\|^2,\quad&\text{otherwise.}
        \end{cases}
    \end{align*}
    The latter implies the first claim that $\{P^N_n\}_{n\geq 1}$ and $\{Q^N_n\}_{n\geq 1}$ are orthogonal, since for any $n,m\geq 1$,
    \begin{align*}
        \ip{P^N_n}{Q^n_m}=\sum_{j=1}^Ne^{in\lambda_j}\ip{\Lambda(f,U,\lambda_j)}{Q^N_m}=0.
    \end{align*}
    The second claim follows this orthogonality:
    \begin{align*}
        \ip{f_n}{f_m}=\ip{P^N_n+Q^N_n}{P^N_m+Q^N_m}&=\ip{P^N_n}{P^n_m}+\ip{P^N_n}{Q^N_m}+\ip{Q^N_n}{P^N_m}+\ip{Q^N_n}{Q^N_m}\\
        &=\ip{P^N_n}{P^N_m}+\ip{Q^N_n}{Q^N_m}.
    \end{align*}
    The final claim follows from (\ref{eq4}), since
    \begin{align*}
        \ip{P^N_n}{P^N_m}=\sum_{j=1}^N\sum_{k=1}^Ne^{in\lambda_j}e^{-im\lambda_k}\ip{\Lambda(f,U,\lambda_j)}{\Lambda(f,U,\lambda_k)}=\sum_{j=1}^Ne^{i(n-m)\lambda_j}\|\Lambda(f,U,\lambda_j)\|^2
    \end{align*}
    which depends on only the difference $n-m$, meaning $\{P^N_n\}_{n\geq 1}$ is WS. It is plain from definition \ref{def3} that the difference of two WS processes is WS, so $\{Q^N_n\}_{n\geq 1}$ is WS also.\hfill{$\qed$}\\[5pt]
    \indent Which is all that is needed to demonstrate the main result of this section. In particular, that the full, nontruncated Fourier series, formally defined
    \begin{align*}
        P_n:=\sum_{j=1}^\infty e^{in\lambda_j}\Lambda(f,U,\lambda_j),\quad\text{with remainder}\quad Q_n:=f_n-P_n,\quad n\geq 1
    \end{align*}
    converges in $\mc{H}$ uniformly in $n$.
    \begin{theorem}[Fourier Decomposition of WS Processes {\cite[theorem 10]{Fan_1946}}]
        Let $\{f_n\}_{n\geq 1}$ be WS in a separable Hilbert space $\mc{H}$ with dynamics $U$. Having extracted the frequencies $\{\lambda_j\}_{j=1}^\infty=\{\lambda\in\mbb{T}:\Lambda(f,U,\lambda)\neq 0\}$
        via the mean ergodic theorem, the Fourier series $\{P_n\}_{n\geq 1}$ converges in $\mc{H}$ uniformly in $n$, both it and the remainders $\{Q_n\}_{n\geq 1}$ are
        WS, and $\ip{P_n}{Q_m}=0$ for any $n,m\geq 1$. Further, the series $\sum_{j=1}^\infty\|\Lambda(f,U,\lambda_j)\|^2<\infty$, and in fact
        \begin{align*}
            \sum_{j=1}^\infty\|\Lambda(f,U,\lambda_j)\|^2\leq \|f_n\|^2
        \end{align*}
        for any $n\geq 1$. Lastly, $\{P_n\}_{n\geq 1}$ has the same Fourier series as $\{f_n\}_{n\geq 1}$, and the Fourier series of $\{Q_n\}_{n\geq 1}$ is null.
    \end{theorem}
    \noindent\textbf{Proof.} By lemma \ref{lem7}, for $n\geq 1$ we have $\ip{P^N_n}{P^N_n}=\sum_{j=1}^N\|\Lambda(f,U,\lambda_j)\|^2$, such that
    \begin{align}
        \|f_n\|^2=\|Q^N_n\|^2+\|P^N_n\|^2\quad\Rightarrow\quad 0\leq \|Q^N_n\|^2=\|f_n\|^2-\sum_{j=1}^N\|\Lambda(f,U,\lambda_j)\|^2\quad\Rightarrow\quad \sum_{j=1}^\infty\|\Lambda(f,U,\lambda_j)\|^2\leq \|f_n\|^2\label{eq5}
    \end{align}
    and so $\sum_{j=1}^\infty\|\Lambda(f,U,\lambda_j)\|^2<\infty$. In particular, the partial sums are Cauchy, so for $\varepsilon>0$, we may take $N>M\geq 1$ sufficiently large so guarantee $\sum_{j=1}^N\|\Lambda(f,U,\lambda_j)\|^2-\sum_{j=1}^{M-1}\|\Lambda(f,U,\lambda_j)\|^2=\sum_{j=M}^N\|\Lambda(f,U,\lambda_j)\|^2<\varepsilon$. Then,
    \begin{align*}
        \left\|\sum_{j=M}^Ne^{in\lambda_j}\Lambda(f,U,\lambda_j)\right\|^2=\sum_{j=M}^N\sum_{k=M}^Ne^{in\lambda_j}e^{-in\lambda_k}\ip{\Lambda(f,U,\lambda_j)}{\Lambda(f,U,\lambda_k)}&=\sum_{j=M}^N\|\Lambda(f,U,\lambda_j)\|^2<\varepsilon
    \end{align*}
    which follows from (\ref{eq4}), i.e. the orthogonality of $\{\Lambda(f,U,\lambda_j)\}_{j=1}^\infty$. But now the partial sums $\{\sum_{j=1}^Ne^{in\lambda_j}\Lambda(f,U,\lambda_j)\}_{N\geq 1}$ are Cauchy in $\mc{H}$, and the $N>M\geq 1$ required
    for this $\varepsilon$-bound were independent of $n\geq 1$, so the $P_n$ converge uniformly in $n$. From lemma (\ref{lem7}),
    \begin{align*}
        \ip{P^N_n}{Q^N_m}=0\quad\Rightarrow\quad\lim_{N\rightarrow\infty}\ip{P^N_n}{Q^N_m}=\ip{P_n}{Q_m}=0
    \end{align*} 
    for $n,m\geq 1$. In the same way, we obtain $\ip{f_n}{f_m}=\ip{P_n}{P_m}+\ip{Q_n}{Q_m}$, but also
    \begin{align*}
        \ip{P_n}{P_m}=\lim_{N\rightarrow\infty}\ip{P^N_n}{P^N_m}=\sum_{j=1}^\infty e^{i(n-m)\lambda_j}\|\Lambda(f,U,\lambda_j)\|^2
    \end{align*}
    which signifies $\{P_n\}_{n\geq 1}$ (and hence $\{Q_n\}_{n\geq 1}$) as WS. [[Show $\{P_n\}_{n\geq 1}$ is its own Fourier series, and $\{Q_n\}_{n\geq 1}$'s is null following \cite[theorem 10]{Fan_1946}]]\hfill{$\qed$}\\[5pt]
    \indent In order for $Q_n=0$ for $n\geq 1$, it is necessary and sufficient that $\sum_{j=1}^\infty e^{i\lambda_j}\Lambda(f,U,\lambda_j)=f_1$ since, when this holds:
    \begin{align*}
        f_1=P_1\quad\Rightarrow\quad f_2=UP_1\quad\Rightarrow\quad\cdots\quad\Rightarrow\quad f_n=U^{n-1}P_1,\quad n\geq 1.
    \end{align*}
    In this case, we have been able to express $\{f_n\}_{n\geq 1}\subseteq\mc{H}$ in terms of a set of nontrivial mean ergodic projections associated to a collection of twisting processes $\{e^{-in\lambda_j}f_n\}_{n\geq 1}$.\\[5pt] 
    [[Concluding remarks, tying back to the narrative from the introduction]]
    \newpage
    \bibliography{CBraun-MET}
    \bibliographystyle{ieeetr}
\end{document}
    % ==============================================
    % \vspace{10pt}
    % consider the set $F\subseteq\mc{H}$ for which the theorem holds:
    % \[F:=\{f\in\mc{H}:\;\lim_{N\rightarrow\infty}C_N\proj_{\mc{H}_0}f=0\}.\]
    % By the previous calculation, we have $\mc{H}_T\oplus(\ran(\mbb{I}-U))^\perp\subseteq F$, so if $F$ is closed, then $F=\mc{H}$. First, for $N\geq 1$
    % \begin{align*}
    %     \left\|C_N\right\|_{op}=\|\frac{1}{N}\sum_{n=0}^{N-1}U^n\|_{op}\leq\frac{1}{N}\sum_{n=0}^{N-1}\|U\|_{op}^n\leq 1
    % \end{align*}
    % so $C_N$ is a contraction for $N\geq 1$. Now consider a sequence $\{f_n\}_{n\geq 1}\subset F$ such that $f_n\rightarrow f$. Clearly, we then have $g_n:=\proj_{\mc{H}_0}f_n\rightarrow\proj_{\mc{H}_0}f=:g$. Fixing $\varepsilon>0$, take $n\geq 1$ large enough so that $\|g-g_n\|\leq\frac{\varepsilon}{2}$,
    % and then $N$ large enough so that $\|C_Ng_n\|\leq\frac{\varepsilon}{2}$. Then,
    % \begin{align*}
    %     \|C_Ng\|\leq\|C_Ng-C_Ng_n\|+\|C_Ng_n\|\leq\|C_N(g-g_n)\|+\frac{\varepsilon}{2}\leq \varepsilon.
    % \end{align*}
    % Thus, $C_N\proj_{\mc{H}_0}f\rightarrow 0$, so $f\in F$, and $F$ is closed.
    % \begin{lemma}[{\cite[lemma 8.2]{Eisner_Farkas_Haase_Nagel_2015}}]
    %     Let $\mc{H}$ be a Hilbert space, $U_T\in\mc{B}(\mc{H})$, and for $N\geq 0$, define the \textit{Ces\`aro mean}
    %     \begin{align*}
    %         C_N:=\frac{1}{N}\sum_{n=0}^{N-1}U_T^{n}.
    %     \end{align*}
    %     Then we have the following facts:

    %     \begin{enumerate}[(i)]
    %         \item If $f\in\fix(U_T)$, then $C_Nf=f$ for all $f\in\mc{H}$.
    %         \item If $C_Nf\overset{N\rightarrow\infty}{\rightarrow}g$, then $U_Tg=g$ and $C_NU_Tf\overset{N\rightarrow\infty}{\rightarrow}g$.
    %         \item If $\frac{1}{N}U_T^Nf\rightarrow 0$ $\forall f\in\mc{H}$, then $C_Nf\rightarrow 0$ for all $f\in\ran(\mbb{I}-U_T)$.
    %         \item If $C_Nf\overset{N\rightarrow\infty}{\rightarrow}g$, then $f-g\in\overline{\ran}(\mbb{I}-U_T)$.
    %     \end{enumerate}
    % \end{lemma}
    % \noindent\textbf{Proof} [[Follow \cite[lemma 8.2]{Eisner_Farkas_Haase_Nagel_2015}. The proof is quite long and tedious, can this go in an appendix?]]
    % \begin{lemma}[{\cite[lemma 8.3]{Eisner_Farkas_Haase_Nagel_2015}}]\label{lem2}
    %     If $U_T\in\mc{B}(\mc{H})$, then $F:=\{f\mc{H}:\;P_Tf:=\lim_{N\rightarrow\infty}C_Nf$ exists$\}$ is $U_T$ invariant, and $\fix(U_T)\subseteq F$. Further,
    %     $P_T:F\rightarrow F$ is a projection onto $\fix(U_T)$, and $U_TP_T=P_TU_T=P_T$.
    % \end{lemma}
    % \noindent\textbf{Proof} [[Follow \cite[lemma 8.3]{Eisner_Farkas_Haase_Nagel_2015}]]
    % \begin{definition}[Mean Ergodicity]\label{def2}
    %     The operator $P_T:F\rightarrow\fix(U_T)$ in lemma (\ref{lem2}) is called the \textit{mean ergodic projection} associated to $U_T$. Further, if $F=\mc{H}$, we call $U_T$ \textit{mean ergodic}.
    % \end{definition}
    % \begin{lemma}[Strong Convergence Lemma {\cite[p.123, ex.1]{Eisner_Farkas_Haase_Nagel_2015}}]\label{lem3}
    %     Let $(V_n)_{n\geq 1}\subseteq\mc{B}(\mc{H})$. If $\exists M\geq 0$ such that $\sup_{n\geq 1}\|V_n\|_{op}\leq M$, then
    %     \[G:=\{f\in\mc{H}:\lim_{n\rightarrow\infty}V_nf\;\text{exists}\}\]
    %     is a closed subspace of $\mc{H}$ and the operator $V:G\rightarrow\mc{H}$ defined by $Vf=\lim_{n\rightarrow\infty}V_nf$, is a bounded linear operator.
    % \end{lemma}
    % \begin{lemma}
    %     If $\sup_{N\geq 0}\|U_T^N\|_{op}<\infty$ and $\frac{1}{N}U_T^Nf\rightarrow 0$ for all $f\in\mc{H}$, then $F$
    %     is closed, $P_T:F\rightarrow\fix(U_T)$ is bounded, and $\ker(P_T)=\overline{\ran}(\mbb{I}-U_T)$.
    % \end{lemma}
    % \noindent\textbf{Proof} [[We only need $F$ closed, $P_T:F\rightarrow F$ a projection onto $\fix(U_T)$ with kernel $\overline{\ran}(\mbb{I}-U_T)$. The first two are from lemma (\ref{lem3}), the third from lemma (\ref{lem2}) and the last is to be proven here.]]\\[5pt]
    % \textbf{Proof of theorem \ref{thm1}}
    % [[Perhaps I should spare myself the work and go straight to \cite[theorem 1.3.1]{Weber_2000}]]
    % \