\documentclass[10pt]{article}
\usepackage[margin=1.3cm]{geometry}

% Packages
\usepackage{amsmath, amsfonts, amssymb, amsthm}
\usepackage{bbm} 
\usepackage{dutchcal} % [dutchcal, calrsfs, pzzcal] calligraphic fonts
\usepackage{graphicx}
\usepackage[T1]{fontenc}
\usepackage[tracking]{microtype}

% Palatino for text goes well with Euler
\usepackage[sc,osf]{mathpazo}   % With old-style figures and real smallcaps.
\linespread{1.025}              % Palatino leads a little more leading

% Euler for math and numbers
\usepackage[euler-digits,small]{eulervm}

% Command initialization
\DeclareMathAlphabet{\pazocal}{OMS}{zplm}{m}{n} \graphicspath{{./images/}}

% Custom Commands
\newcommand{\bs}[1]{\boldsymbol{#1}} \newcommand{\E}{\mathbb{E}}
\newcommand{\var}[1]{\text{Var}\left(#1\right)}
\newcommand{\bp}[1]{\left({#1}\right)} \newcommand{\mbb}[1]{\mathbb{#1}}
\newcommand{\1}[1]{\mathbbm{1}_{#1}} \newcommand{\mc}[1]{\mathcal{#1}}
\newcommand{\nck}[2]{{#1\choose#2}} \newcommand{\pc}[1]{\pazocal{#1}}
\newcommand{\ra}[1]{\renewcommand{\arraystretch}{#1}}
\newcommand*{\floor}[1]{\left\lfloor#1\right\rfloor}
\newcommand*{\ceil}[1]{\left\lceil#1\right\rceil}

\DeclareMathOperator{\Var}{Var} \DeclareMathOperator{\Cov}{Cov}
\DeclareMathOperator{\diag}{diag}

\makeatletter
\def\Ddots{\mathinner{\mkern1mu\raise\p@
\vbox{\kern7\p@\hbox{.}}\mkern2mu
\raise4\p@\hbox{.}\mkern2mu\raise7\p@\hbox{.}\mkern1mu}}
\makeatother

\def\powertower#1#2{#1\ifnum#2>1 ^{\powertower{#1}{\numexpr#2-1\relax}}\fi}

\newtheorem{theorem}{Theorem}
\newtheorem{lemma}{Lemma}

\begin{document}

    \begin{center}
        {\bf\large{MATH 829: FUNCTIONAL ANALYSIS AND QUANTUM MECHANICS}}
        \smallskip
        \hrule
        \smallskip
        {\bf Assignment} 4\hfill {\bf Connor Braun} \hfill {\bf 2024-10-03}
    \end{center}
    \vspace{5pt}
    \begin{center}
        \begin{minipage}{\dimexpr\paperwidth-10cm}
            I did not work with anybody in completing this assignment.
        \end{minipage}
    \end{center}
    \noindent{\bf Problem 1} Prove that if a real vector space $(X,\|\cdot\|)$
    has a Schauder basis, then it is separable by following the steps
    below.\\[5pt]
    {\bf a)} Suppose there is a Schauder basis $(e_n)_{n\geq 1}\subset X$, and
    without loss of generality that $\|e_n\|=1$ for $n\geq 1$. Explain why for
    every $x\in X$ and $\varepsilon>0$ $\exists N\geq 1$ such that
    \[\left\|x-\sum_{n=1}^N\alpha_n e_n\right\|<\varepsilon\] for some
    $(\alpha_n)_{n\geq 1}\subset\mbb{R}$.\\[5pt]
    {\bf Solution}. Since $(e_n)_{n\geq 1}$ is a Schauder basis, fixing $x\in X$
    there exists some $(\alpha_n)_{n\geq 1}$ such that
    $x=\sum_{n=1}^\infty\alpha_n e_n$. Fixing also some $\varepsilon>0$, we have
    $x=\sum_{n=1}^\infty\alpha_n
    e_n=\lim_{N\rightarrow\infty}\sum_{n=1}^N\alpha_ne_n$, and so the definition
    of a limit applied to the sequence of partial sums yields an $N^\prime\geq
    1$ so that $N\geq N^\prime$ implies
    \begin{align*}
        \left\|x-\sum_{n=1}^N\alpha_n e_n\right\|<\varepsilon.\tag*{$\qed$}
    \end{align*}
    {\bf b)} Let $N\geq 1$ and $\alpha_1,\dots, \alpha_N$ be as in (a). Show
    that for $1\leq n\leq N$, $\exists r_n\in\mbb{Q}$ such that
    $|\alpha_n-r_n|<\tfrac{\varepsilon}{N}$.\\[5pt]
    {\bf Solution}. Since $\mbb{Q}$ is dense in $\mbb{R}$, for $1\leq n\leq N$
    we can find $(r_n^j)_{j\geq 1}\subseteq\mbb{Q}$ so that $r^j_n\rightarrow
    \alpha_n$ as $j\rightarrow\infty$. In particular, $\exists J_n\geq 1$ so
    that $j\geq J_n$ $\Rightarrow$ $|r_n^j-\alpha_n|<\tfrac{\varepsilon}{N}$.
    With $J:=\max\{J_i\}_{i=1}^N$, and $j\geq J$, set $r_n:=r_n^j$ for $1\leq
    n\leq N$. Then $(r_n)_{n=1}^N$ satisfies
    $|r_n-\alpha_n|<\tfrac{\varepsilon}{N}$ by
    construction.\hfill{$\qed$}\\[5pt]
    {\bf c)} Let $r_1,\dots, r_N$ be as in (b) and consider $y=\sum_{n=1}^N
    r_ne_n$. Show that $\|x-y\|<2\varepsilon$. Explain why this concludes the
    proof.\\[5pt]
    {\bf Solution} For better clarity, let us suppose that $N$ in (a) was
    instead chosen large enough so that
    $\|x-\sum_{n=1}^N\alpha_ne_n\|<\tfrac{\varepsilon}{2}$, and subsequently
    each $r_n$ so that $|r_n-\alpha_n|<\tfrac{\varepsilon}{2N}$ for $1\leq n\leq
    N$. Then,
    \begin{align*}
        \|x-y\|=\left\|x-\sum_{n=1}^N\alpha_ne_n+\sum_{n=1}^N\alpha_ne_n-\sum_{n=1}^Nr_ne_n\right\|&\leq \left\|x-\sum_{n=1}^N\alpha_ne_n\right\|+\left\|\sum_{n=1}^N(\alpha_n-r_n)e_n\right\|\\
        &\leq \tfrac{\varepsilon}{2}+\sum_{n=1}^N|\alpha_n-r_n|\|e_n\|\\
        &\leq\tfrac{\varepsilon}{2}+N\tfrac{\varepsilon}{2N}\\
        &=\varepsilon.
    \end{align*}
    We have shown that $\forall x\in X$, $\exists y\in Y\subset X$ so that
    $\|x-y\|<\varepsilon$, where $Y:=\{y\in X:\exists N\geq
    1,\;(r_n)_{n=1}^N\subset\mbb{Q}\;\text{with}\;y=\sum_{n=1}^Nr_ne_n\}$. That
    is, $Y$ is dense in $X$. For separability we need only for $Y$ to be
    countable.\\[5pt]
    To show this, define an injection $g$ so that $\forall y\in Y$,
    $y\overset{g}{\mapsto} (N,r)\in\mbb{N}\times R$, where $r=(r_n)_{n=1}^N$ and
    $R$ is the set of finite subsets of $\mbb{Q}$. Next, we show the set
    $R\subset 2^\mbb{Q}$ is countable. Define $R_n\subset R$ the set of subsets
    of cardinality $n$ for $n\geq 1$. Then, an obvious bijection can be
    established between $R_n$ and $\mbb{Q}^n$ for $n\geq 1$ (by imposing a
    partial order on the elements of each set in $R_n$) and subsequently,
    \[|R|=\left|\bigcup_{n\in\mbb{N}}R_n\right|=\sum_{n\geq\mbb{N}}|R_n|=\sum_{n\in\mbb{N}}|\mbb{Q}^n|\]
    which is a countable infinity, since $\mbb{Q}^n$ is countable for $n\geq 1$
    (the last equality above holds since $R_n\cap R_m=\emptyset$ for $n\neq m$).
    Of course, the product $\mbb{N}\times R$ is then also countable, so with the
    restricted inverse $g_Y^{-1}:g(Y)\rightarrow Y$, defined as
    $g_Y^{-1}(N,r):=g^{-1}(N,r)$ for all $(N,r)\in g(Y)$, the pair $(g,g_Y)$
    defines a bijection between $Y$ and $g(Y)$. Finally, $g(Y)\subset
    \mbb{N}\times R$ implies $g(Y)$ is at most countable, so $Y$ is at most
    countable too.\hfill{$\qed$}
\end{document}