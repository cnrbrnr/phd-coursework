\documentclass[10pt]{article}
\usepackage[margin=1.3cm]{geometry}

% Packages
\usepackage{amsmath, amsfonts, amssymb, amsthm}
\usepackage{bbm} 
\usepackage{dutchcal} % [dutchcal, calrsfs, pzzcal] calligraphic fonts
\usepackage{graphicx}
\usepackage[T1]{fontenc}
\usepackage[tracking]{microtype}

% Palatino for text goes well with Euler
\usepackage[sc,osf]{mathpazo}   % With old-style figures and real smallcaps.
\linespread{1.025}              % Palatino leads a little more leading

% Euler for math and numbers
\usepackage[euler-digits,small]{eulervm}

% Command initialization
\DeclareMathAlphabet{\pazocal}{OMS}{zplm}{m}{n} \graphicspath{{./images/}}

% Custom Commands
\newcommand{\bs}[1]{\boldsymbol{#1}} \newcommand{\E}{\mathbb{E}}
\newcommand{\var}[1]{\text{Var}\left(#1\right)}
\newcommand{\bp}[1]{\left({#1}\right)} \newcommand{\mbb}[1]{\mathbb{#1}}
\newcommand{\1}[1]{\mathbbm{1}_{#1}} \newcommand{\mc}[1]{\mathcal{#1}}
\newcommand{\nck}[2]{{#1\choose#2}} \newcommand{\pc}[1]{\pazocal{#1}}
\newcommand{\ra}[1]{\renewcommand{\arraystretch}{#1}}
\newcommand*{\floor}[1]{\left\lfloor#1\right\rfloor}
\newcommand*{\ceil}[1]{\left\lceil#1\right\rceil}

\DeclareMathOperator{\Var}{Var} \DeclareMathOperator{\Cov}{Cov}
\DeclareMathOperator{\diag}{diag}

\makeatletter
\def\Ddots{\mathinner{\mkern1mu\raise\p@
\vbox{\kern7\p@\hbox{.}}\mkern2mu
\raise4\p@\hbox{.}\mkern2mu\raise7\p@\hbox{.}\mkern1mu}}
\makeatother

\def\powertower#1#2{#1\ifnum#2>1 ^{\powertower{#1}{\numexpr#2-1\relax}}\fi}

\newtheorem{theorem}{Theorem}
\newtheorem{lemma}{Lemma}

\begin{document}

    \begin{center}
        {\bf\large{MATH 829: FUNCTIONAL ANALYSIS AND QUANTUM MECHANICS}}
        \smallskip
        \hrule
        \smallskip
        {\bf Assignment} 4\hfill {\bf Connor Braun} \hfill {\bf 2024-10-03}
    \end{center}
    \vspace{5pt}
    \noindent{\bf Problem 2} Prove the theorem stated on the
    assignment by following the steps below.\\[5pt]
    {\bf a)} Explain why $\lim_{k\rightarrow \infty} x_{n_k+1}=T(z)$.\\[5pt]
    {\bf Solution}. With $(x_{n_k})_{k\geq 1}$ the subsequence from condition
    (iii) in the statement of the theorem, $\lim_{k\rightarrow\infty}x_{n_k}= z$
    by hypothesis. Then, by (ii) (i.e., the continuity of $T$ at $z$) we have 
    \[\lim_{k\rightarrow\infty}x_{n_k+1}=\lim_{k\rightarrow\infty}T(x_{n_k})=T(\lim_{k\rightarrow\infty}x_{n_k})=T(z).\tag*{$\qed$}\]
    {\bf b)} Assume for the purpose of deriving a contradiction that $z\neq
    T(z)$. Fix $0<\eta<\tfrac{1}{3}d(z,T(z))$. Let $B_1=B_\eta(z)$, and
    $B_2=B_\eta(T(z))$. Explain why $\exists K\geq 1$ such that $k\geq K$
    implies that $x_{n_k}\in B_1$ and $x_{n_k+1}\in B_2$. Deduce that, for
    $k\geq K$, 
    \[d(x_{n_k},x_{n_k}+1)>\eta.\] {\bf Solution}. Since $x_{n_k}\rightarrow z$
    as $k\rightarrow\infty$, $\exists K_1\geq 1$ so that $k\geq K_1$
    $\Rightarrow$ $d(z,x_{n_k})<\eta$. But from part (a), $\exists K_2\geq 1$ so
    that $k\geq K_2$ $\Rightarrow$ $d(x_{n_k+1},T(z))<\eta$. Setting
    $K=\max\{K_1,K_2\}$, both $d(x_{n_k},z)<\eta$ and $d(x_{n_k+1},T(z))<\eta$
    hold. Then, with $k\geq K$:
    \begin{align*}
        d(x_{n_k},x_{n_k+1})&\geq |d(x_{n_k},z)-d(z,x_{n_k+1})|\tag{1}\\
        &=d(z,x_{n_k+1})-d(x_{n_k},z)\tag{2}\\
        &>d(z,x_{n_k+1})-\eta\\
        &\geq|d(z,T(z))-d(x_{n_k+1},T(z))|-\eta\tag{3}\\
        &=d(z,T(z))-d(x_{n_k+1},T(z))-\eta\tag{4}\\
        &>3\eta-\eta-\eta\tag{5}\\
        &=\eta
    \end{align*}
    as desired. Steps (1) and (3) are obtained from the reverse triangle
    inequality, (2) by observing that $d(x_{n_k+1},z)\geq d(z,x_{n_k})$ (since
    $x_{n_k}\in B_1$, but $x_{n_k+1}\notin B_1$) and (4) from the similar
    observation that $d(z,T(z))\geq d(x_{n_k+1},T(z))$ (since $x_{n_k+1}\in B_2$
    but $z\notin B_2$). The appearance of $3\eta$ in (5) follows the definition
    of $\eta$, namely that $3\eta<d(z,T(z))$.\hfill{$\qed$}\\[5pt]
    {\bf c)} Using the hypothesis that $\exists\lambda\in(0,\tfrac{1}{2})$ so
    that $\forall x\in X$:
    \[d(T(x),T(y))\leq \lambda\left[d(x,T(x))+d(y,T(y))\right],\] deduce that
    \[d(x_{n_k+1},x_{n_k+2})\leq\frac{\lambda}{1-\lambda}d(x_{n_k},x_{n_k+1}).\]
    {\bf Solution}. This bound is obtained directly from the hypothesis:
    \begin{align*}
        d(x_{n_k+1},x_{n_k+2})=d(T(x_{n_k}),T(x_{n_k+1}))&\leq\lambda\left[d(x_{n_k},T(x_{n_k}))+d(x_{n_k+1},T(x_{n_k+1}))\right]\\
        &=\lambda\left[d(x_{n_k},x_{n_k+1})+d(x_{n_k+1},x_{n_k+2})\right]\\
        \Rightarrow\qquad\qquad d(x_{n_k+1},x_{n_k+2})(1-\lambda)&\leq d(x_{n_k},x_{n_k+1})\leq \lambda d(x_{n_k},x_{n_k+1})\\
        \Rightarrow\qquad\qquad d(x_{n_k+1},x_{n_k+2})&\leq \frac{\lambda}{1-\lambda}d(x_{n_k},x_{n_k+1})\tag*{$\qed$}
    \end{align*}
    {\bf d)} Deduce that for $\ell>k\geq K$, we have
    \begin{align*}
        d(x_{n_\ell},x_{n_\ell+1})\leq\left(\frac{\lambda}{1-\lambda}\right)^{n_\ell-n_k}d(x_{n_k},x_{n_k+1}).
    \end{align*}
    {\bf Solution}. This is a direct consequence of the bound found in (c).
    Specifically, with $\ell>k\geq K$ as prescribed (and $K$ as in (b)):
    \begin{align*}
        d(x_{n_\ell},x_{n_\ell+1})\leq \left(\frac{\lambda}{1-\lambda}\right)d(x_{n_\ell-1},x_{n_\ell})&\leq \left(\frac{\lambda}{1-\lambda}\right)^2d(x_{n_\ell-2},x_{n_\ell-1})\leq\dots\leq \left(\frac{\lambda}{1-\lambda}\right)^{n_\ell-n_k}d(x_{n_k},x_{n_k+1})\tag{$\qed$}
    \end{align*}
    {\bf e)} Explain why taking the limit as $\ell\rightarrow\infty$ in (d)
    yields a contradiction to the lower bound in (b), thus showing that
    $z=T(z)$.\\[5pt]
    {\bf Solution}. Since $\lambda\in(0,\tfrac{1}{2})$,
    $\tfrac{\lambda}{1-\lambda}<1$, so
    $\lim_{\ell\rightarrow\infty}(\tfrac{\lambda}{1-\lambda})^{n_\ell-n_k}=0$.
    Applying this,
    \begin{align*}
        0\leq\lim_{\ell\rightarrow\infty}d(x_{n_\ell},x_{n_\ell+1})\leq\lim_{\ell\rightarrow\infty}\left(\frac{\lambda}{1-\lambda}\right)^{n_\ell-n_k}d(x_{n_k},x_{n_k+1})=0
    \end{align*}
    so $\lim_{\ell\rightarrow\infty}d(x_{n_\ell},x_{n_\ell+1})=0$, and we can
    find $L\geq 1$ so that $\ell\geq L$ $\Rightarrow$
    $d(x_{n_\ell},x_{n_\ell+1})<\eta$. But this contradicts $\ell\geq K$
    $\Rightarrow$ $d(x_{n_\ell},x_{n_\ell+1})>\eta$ from (b). To resolve this,
    we conclude $z=T(z)$.\\[5pt]
    {\bf f)} Prove that the fixed point $z$ is unique.\\[5pt]
    {\bf Solution} Suppose $z^\prime$ is another fixed point of $T$. Then,
    \begin{align*}
        0\leq d(z,z^\prime)=d(T(z),T(z^\prime))\leq\lambda\left[d(z,T(z))+d(z^\prime,T(z^\prime))\right]=\lambda\left[d(T(z),T(z))+d(T(z^\prime),T(z^\prime))\right]=0
    \end{align*}
    where since $d$ is a metric, we conclude $z=z^\prime$, and thus the fixed
    point $z$ is unique. Note that, in the above chain, the penultimate
    inequality follows the assumption that $z^\prime$ is a fixed point of
    $T$.\hfill{$\qed$}\\[5pt]
\end{document}