\documentclass[10pt]{article}
\usepackage[margin=1.3cm]{geometry}

% Packages
\usepackage{amsmath, amsfonts, amssymb, amsthm}
\usepackage{bbm} 
\usepackage{dutchcal} % [dutchcal, calrsfs, pzzcal] calligraphic fonts
\usepackage{graphicx}
\usepackage[T1]{fontenc}
\usepackage[tracking]{microtype}
\usepackage{stmaryrd}

% Palatino for text goes well with Euler
\usepackage[sc,osf]{mathpazo}   % With old-style figures and real smallcaps.
\linespread{1.025}              % Palatino leads a little more leading

% Euler for math and numbers
\usepackage[euler-digits,small]{eulervm}

% Command initialization
\DeclareMathAlphabet{\pazocal}{OMS}{zplm}{m}{n}
\graphicspath{{./images/}}

% Custom Commands
\newcommand{\bs}[1]{\boldsymbol{#1}}
\newcommand{\E}{\mathbb{E}}
\newcommand{\var}[1]{\text{Var}\left(#1\right)}
\newcommand{\bp}[1]{\left({#1}\right)}
\newcommand{\mbb}[1]{\mathbb{#1}}
\newcommand{\1}[1]{\mathbbm{1}_{#1}}
\newcommand{\mc}[1]{\mathcal{#1}}
\newcommand{\nck}[2]{{#1\choose#2}}
\newcommand{\pc}[1]{\pazocal{#1}}
\newcommand{\ra}[1]{\renewcommand{\arraystretch}{#1}}
\newcommand*{\floor}[1]{\left\lfloor#1\right\rfloor}
\newcommand*{\ceil}[1]{\left\lceil#1\right\rceil}

\DeclareMathOperator{\Var}{Var} \DeclareMathOperator{\Cov}{Cov}
\DeclareMathOperator{\diag}{diag}
\DeclareMathOperator{\sgn}{sgn}

\newtheorem{theorem}{Theorem}
\newtheorem{lemma}{Lemma}

\newenvironment{indented}[1][2cm]{\setlength{\leftskip}{#1}}{\setlength{\leftskip}{0pt}}

\begin{document}

    \begin{center}
        {\bf\large{MATH 829: FUNCTIONAL ANALYSIS AND QUANTUM MECHANICS}}
        \smallskip
        \hrule
        \smallskip
        {\bf Assignment} 1\hfill {\bf Connor Braun} \hfill {\bf 2024-09-10}
    \end{center}
    \vspace{5pt}
    \noindent{\bf Problem 2} Recall the normed vector spaces $(\ell_p,\|\cdot\|_p)$,
    where $1\leq p\leq \infty$.\\[5pt]
    {\bf a)} Show that if $1\leq p,\leq q<\infty$, then $\|x\|_q\leq
    \|x\|_p$.\\[5pt]
    {\bf Solution}. Fix $1\leq p<q<\infty$ (the case of $q=p$ holds trivially)
    and take $x\in\ell_p$, supposing first that $\|x\|_p\leq 1$. That is,
    $\sum_{j=1}^\infty|x_j|^p\leq 1$, so in particular, $|x_j|\leq 1$ $\forall
    j\geq 1$. But then $|x_j|^p\geq |x_j|^q$ $\forall j\geq 1$, so
    $\sum_{j=1}^\infty|x_j|^q\leq \sum_{j=1}^\infty|x_j|^p\leq 1$, and further
    $(\sum_{j=1}^\infty|x_j|^q)^{1/q}\leq (\sum_{j=1}^\infty|x_j|^p)^{1/p}$,
    i.e. $\|x\|_q\leq\|x\|_p$ (which holds since both series are $\leq 1$, and
    $q>p$).\\[5pt]
    More generally, for $x\in\ell_p$ such that $x\neq 0$ (which is also trivial:
    $\|0\|_p=0\geq 0=\|0\|_q$) so that $0<\|x\|_p<\infty$, we consider
    $\tilde{x}:=\tfrac{1}{\|x\|_p}x$, which satisfies
    $\|\tilde{x}\|_p=\tfrac{1}{\|x\|_p}\|x\|_p=1$. But then
    \begin{align*}
        \|\tilde{x}\|_q\leq 1\quad\Rightarrow\quad\frac{1}{\|x\|_p}\|x\|_q\leq 1\quad\Rightarrow\quad \|x\|_q\leq\|x\|_p
    \end{align*}
    with the first expression following the preliminary
    case.\hfill{$\qed$}\\[5pt]
    {\bf b)} Part (a) shows that $\ell_p\subseteq\ell_q$ for $p<q$. Determine
    whether either of the statements
    \begin{align*}
        \ell_p=\bigcap_{q>p}\ell_q\tag{$\mathfrak{A}$}\\
        \ell_q=\bigcup_{1\leq p<q}\ell_p\tag{$\mathfrak{B}$}
    \end{align*}
    are true, and provide a proof.\\[5pt]
    {\bf Solution}. Both statements are false, and we will begin with the
    relatively simpler counterexample for $\mathfrak{A}$. Consider the sequence
    $x=(\tfrac{1}{n^{1/p}})_{n\geq 1}$. Then $x\in\ell_q$ for all $q>p$ since
    \begin{align*}
        \|x\|_q^q=\sum_{n\geq 1}\frac{1}{n^{q/p}}<\infty
    \end{align*}
    because $q/p>1$. However, $\|x\|^p_p=\sum_{n\geq 1}\tfrac{1}{n}$ diverges,
    so $x\notin\ell_p$, disproving $\mathfrak{A}$.\\[5pt]
    As a counterexample to $\mathfrak{B}$, fix some $q>p$ and consider the
    sequence $x=(x_n)_{n\geq 1}$ with
    \begin{align*}
        x_n=\frac{1}{[(n+2)(\log(n+2))^2]^{1/q}},\quad n\geq 1,\quad\text{and}\quad\|x\|_q^q=\sum_{n\geq 1}\frac{1}{(n+2)(\log(n+2))^2}
    \end{align*}
    To show $x\in\ell_q$, we will use the integral test (since the $x_j$ can be
    interpretted by a continuous, decreasing function on $[1,\infty)$):
    \begin{align*}
        \int_1^\infty\frac{dx}{(x+2)(\log(x+2))^2}=\int_{\log(3)}^\infty\frac{(x+2)du}{(x+2)u^2}&=\int_{\log(3)}^\infty\frac{du}{u^2}\tag{change of variables $u=\log(x+2)$}\\
        &=-\frac{1}{u}\bigg|^\infty_{\log(3)}\\
        &=\lim_{u\rightarrow\infty}-\frac{1}{u}+\frac{1}{\log(3)}<\infty
    \end{align*}
    which implies that $\|x\|_q^q<\infty$, so $x\in\ell_q$. We next claim that
    for any $1\leq p<q$, $\|x\|_p=\infty$. First, observe that
    $\log(n+2)>\log(e)=1$ for all $n\geq 1$. Then, whenever $0<m<1$ we get
    \begin{align*}
        \frac{1}{[(n+2)(\log(n+2))^2]^m}=\frac{1}{(n+2)^m(\log(n+2))^{2m}}\geq\frac{1}{(n+2)^m(\log(n+2))^2},\quad n\geq 1.
    \end{align*}
    In particular, this says $\sum_{n\geq
    1}\tfrac{1}{[(n+2)(\log(n+2))^2]^m}\geq\sum_{n\geq
    1}\frac{1}{(n+2)^m(\log(n+2))^2}$. Fixing such an $m\in(0,1)$, let us once
    again use the integral test to determine the convergence of the latter
    series (this is once again justified by the existence of a continuous,
    decreasing interpolant):
    \begin{align*}
        \int_1^\infty\frac{dx}{(x+2)^m(\log(x+2))^2}&=\int_{\log(3)}^\infty\frac{(x+2)^{1-m}du}{u^2}\tag{change of variables $u=\log(x+2)$}\\
        &=\int_{\log(3)}^\infty\frac{e^{u(1-m)}du}{u^2}\tag{$(x+2)=e^u$}\\
        &=\frac{-e^{u(1-m)}}{u}\bigg|_{\log(3)}^\infty+\int_{\log(3)}^\infty\frac{(1-m)e^{u(1-m)}du}{u}\tag{integration by parts}\\
        &\geq \lim_{u\rightarrow\infty}-\frac{e^{u(1-m)}}{u}+\frac{3^{(1-m)}}{\log(3)}\tag{$\int_{\log(3)}^\infty\tfrac{(1-m)e^{u(1-m)}du}{u}\geq 0$}\\
        &=\infty
    \end{align*} 
    where the limit is easily evaluated with L'H\^opital's rule given the
    indeterminate $\tfrac{\infty}{\infty}$. Since this integral diverges, so too
    does the series; that is
    \begin{align*}
        \sum_{n\geq 1}\frac{1}{[(n+2)(\log(n+2))^2]^m}=\infty,\quad 0<m<1.
    \end{align*}
    Of course, with $q>p\geq 1$, this means that
    \begin{align*}
        \|x\|_p^p=\sum_{n\geq 1}\frac{1}{[(n+2)(\log(n+2))^2]^{p/q}}=\infty,\quad\text{since $0<p/q<1$.}
    \end{align*}
    In total, we've found a sequence $x$ with $x\in\ell_q$, but
    $x\notin\bigcup_{1\leq p<q}\ell_p$. Thus, $\mathfrak{B}$ is
    false.\hfill{$\qed$}
\end{document}