\documentclass[10pt]{article}
\usepackage[margin=1.3cm]{geometry}

% Packages
\usepackage{amsmath, amsfonts, amssymb, amsthm}
\usepackage{bbm} 
\usepackage{dutchcal} % [dutchcal, calrsfs, pzzcal] calligraphic fonts
\usepackage{graphicx}
\usepackage[T1]{fontenc}
\usepackage[tracking]{microtype}
\usepackage{stmaryrd}

% Palatino for text goes well with Euler
\usepackage[sc,osf]{mathpazo}   % With old-style figures and real smallcaps.
\linespread{1.025}              % Palatino leads a little more leading

% Euler for math and numbers
\usepackage[euler-digits,small]{eulervm}

% Command initialization
\DeclareMathAlphabet{\pazocal}{OMS}{zplm}{m}{n}
\graphicspath{{./images/}}

% Custom Commands
\newcommand{\bs}[1]{\boldsymbol{#1}}
\newcommand{\E}{\mathbb{E}}
\newcommand{\var}[1]{\text{Var}\left(#1\right)}
\newcommand{\bp}[1]{\left({#1}\right)}
\newcommand{\mbb}[1]{\mathbb{#1}}
\newcommand{\1}[1]{\mathbbm{1}_{#1}}
\newcommand{\mc}[1]{\mathcal{#1}}
\newcommand{\nck}[2]{{#1\choose#2}}
\newcommand{\pc}[1]{\pazocal{#1}}
\newcommand{\ra}[1]{\renewcommand{\arraystretch}{#1}}
\newcommand*{\floor}[1]{\left\lfloor#1\right\rfloor}
\newcommand*{\ceil}[1]{\left\lceil#1\right\rceil}

\DeclareMathOperator{\Var}{Var} \DeclareMathOperator{\Cov}{Cov}
\DeclareMathOperator{\diag}{diag}
\DeclareMathOperator{\sgn}{sgn}

\newtheorem{theorem}{Theorem}
\newtheorem{lemma}{Lemma}

\newenvironment{indented}[1][2cm]{\setlength{\leftskip}{#1}}{\setlength{\leftskip}{0pt}}

\begin{document}

    \begin{center}
        {\bf\large{MATH 829: FUNCTIONAL ANALYSIS AND QUANTUM MECHANICS}}
        \smallskip
        \hrule
        \smallskip
        {\bf Assignment} 1\hfill {\bf Connor Braun} \hfill {\bf 2024-09-10}
    \end{center}
    \vspace{5pt}
    \noindent{\bf Problem 3}. Recall the space $(\ell_2,\|\cdot\|_2)$, where
    $\ell_2=\{x=(x_n)_{n\geq 1}:\|x\|_2<\infty\}$ and $\|x\|_2=\left(\sum_{n\geq
    1}|x_n|^2\right)^{1/2}$. Consider the set $V$ given by
    \begin{align*}
        V:=\left\{x=(x_n)_{n\geq 1}:\sum_{n=1}^\infty n^2|x_n|^2\leq 1\right\}.
    \end{align*}
    {\bf a)} Prove $V$ is a convex subset of $\ell_2$.\\[5pt]
    {\bf Proof}. For any $x\in V$, we have $\|x\|_2=\left(\sum_{n\geq
    1}|x_n|^2\right)^{1/2}<\left(\sum_{n\geq 1}n^2|x_n|^2\right)^{1/2}=1$, so
    indeed, $x\in\ell_2$ and thus $V\subseteq\ell_2$. To demonstrate convexity,
    take $\lambda\in[0,1]$ and some $x,y\in V$. Then the convex combination
    $\lambda x+(1-\lambda)y\in V$ too, as the following shows:
    \begin{align*}
        \sum_{n\geq 1}n^2|\lambda x_n+(1-\lambda)y_n|^2&\leq \sum_{n\geq 1}n^2(|\lambda x_n|+|(1-\lambda)y_n|)^2\tag{3}\\
        &=\lambda^2\left[\sum_{n\geq 1}n^2|x_n|^2\right]+(1-\lambda)^2\left[\sum_{n\geq 1}n^2|y_n|^2\right]+2\lambda(1-\lambda)\left[\sum_{n\geq 1}n^2|x_ny_n|\right]\\
        &\leq\lambda^2+(1-\lambda)^2+2\lambda(1-\lambda)\left[\sum_{n\geq 1}|nx_nny_n|\right]\tag{4}\\ 
        &\leq \lambda^2+(1-\lambda)^2+2\lambda(1-\lambda)\left(\sum_{n\geq 1}n^2|x_n|^2\right)^{1/2}\left(\sum_{n\geq 1}n^2|y_n|^2\right)^{1/2}\tag{5}\\
        &\leq \lambda^2+(1-\lambda)^2+2\lambda(1-\lambda)\\
        &=(\lambda+(1-\lambda))^2\\
        &=1.
    \end{align*}
    Elaborating, (3) follows the triangle inequality, (4) the assumption that
    $x,y\in V$, and (5) by H\"older's inequality with conjugate pair
    $(2,2)$.\hfill{$\qed$}\\[5pt]
    {\bf b)} Show that $V$ is centrally symmetric.\\[5pt]
    {\bf Solution}. Take $x\in V$. Then $\sum_{n\geq 1}n^2|-x_n|^2=\sum_{n\geq
    1}n^2|x_n|^2\leq 1$. Thus, $x\in V$ $\Rightarrow$ $-x\in
    V$.\hfill{$\qed$}\\[5pt]
    {\bf c)} Show that $V$ has an empty interior by solving the next three
    subproblems.\\[5pt]
    {\bf c.i)} Suppose, for the purposes of deriving a contradiction, that
    $S^\prime=B_r(x_0)\subset V$ for some $x_0\in V$, $r>0$. Show this implies
    $S^{\prime\prime}=B_r(-x_0)$, and that these facts further imply
    $B_r(0)\subset V$.\\[5pt]
    {\bf Proof}. Let us suppose by way of deriving a contradition that
    $B_r(x_0)\subset V$ for some $x_0\in V$, $r>0$, as prescribed. Then $-x_0\in
    V$ by the central symmetry of $V$, and for any $x\in B_r(-x_0)$ we get
    \begin{align*}
        \|x-(-x_0)\|_2<r\quad\Rightarrow\quad|-1|\|(-x)-x_0\|_2<r\quad\Rightarrow\quad (-x)\in B_r(x_0)\subset V
    \end{align*}
    whereby the central symmetry of $V$, $-x\in V$ $\Rightarrow$  $x\in V$, so
    $B_r(-x_0)\subset V$ as well. To show $B_r(0)\subset V$, observe first that
    $0\in V$ (since $\sum_{n\geq 1}n^2|0|=0\leq 1$) and consider a point $x\in
    B_r(0)$ so that $\|x\|_2<r$. Then $x+x_0\in B_r(x_0)$ and $x-x_0\in
    B_r(-x_0)$, since $\|x+x_0-x_0\|<r$ and $\|x-x_0-(-x_0)\|<r$ (respectively).
    But $V$ is convex, so the convex combination
    \begin{align*}
        \frac{1}{2}(x+x_0)+\frac{1}{2}(x-x_0)=x
    \end{align*}
    is also in $V$. That is, $B_r(0)\subset V$, as desired.\hfill{$\qed$}\\[5pt]
    {\bf c.ii)} Use (c.i) to claim that on every ray emanating from $0\in V$
    there must lie a segment belonging entirely to $V$.\\[5pt]
    {\bf Proof}. Let $x\in\ell_2$, $x\neq 0$ (for otherwise the ray is trivially
    a subset of $B_r(0)$) and $\Lambda:=\{\lambda x:\lambda\geq 0\}$. Provided
    $0\leq \lambda<\tfrac{r}{\|x\|_2}$, we get
    \begin{align*}
        \|\lambda x\|_2<\frac{r}{\|x\|_2}\|x\|_2=r,\quad\Rightarrow\quad \lambda x\in B_r(0)\subset V
    \end{align*}
    so that $\Lambda^\prime:=\{\lambda
    x:\lambda\in[0,\tfrac{r}{\|x\|_2})\}\subset\Lambda$ defining a segment of
    the ray emanating from $0$ in the direction of $x$ is satisfies
    $\Lambda^\prime\subset B_r(0)\subset V$.\hfill{$\qed$}\\[5pt]
    {\bf c.iii)} Use the ray connecting $0$ and $x=(n^{-1})_{n\geq 1}$ to derive
    a contradition.\\[5pt]
    {\bf Solution}. First, observe that $x\in\ell_2$, since
    $\|x\|_2^2=\sum_{n\geq 1}\tfrac{1}{n^2}<\infty$. Then by (c.ii), $\exists
    \lambda\geq 0$ so that $\lambda x\in B_r(0)\subset V$. In particular, we may
    take $\lambda>0$ since $\tfrac{r}{\|x\|_2}>0$. From this we obtain
    \begin{align*}
        \sum_{n\geq 1}n^2\|\lambda\frac{1}{n}|^2\leq 1\quad\Rightarrow\quad\lambda\sum_{n\geq 1}\frac{n^2}{n^2}\leq 1\quad\Rightarrow\quad\lambda\cdot\infty\leq 1\quad\lightning
    \end{align*}
    which is a clear contradiction. To resolve this, it must be the case that
    $B_r(x_0)\nsubseteq V$ for any choice of $x_0\in\ell_2$, $r>0$. That is, $V$
    has empty interior.\hfill{$\qed$}
\end{document}