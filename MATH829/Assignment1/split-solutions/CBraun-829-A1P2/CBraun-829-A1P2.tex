\documentclass[10pt]{article}
\usepackage[margin=1.3cm]{geometry}

% Packages
\usepackage{amsmath, amsfonts, amssymb, amsthm}
\usepackage{bbm} 
\usepackage{dutchcal} % [dutchcal, calrsfs, pzzcal] calligraphic fonts
\usepackage{graphicx}
\usepackage[T1]{fontenc}
\usepackage[tracking]{microtype}

% Palatino for text goes well with Euler
\usepackage[sc,osf]{mathpazo}   % With old-style figures and real smallcaps.
\linespread{1.025}              % Palatino leads a little more leading

% Euler for math and numbers
\usepackage[euler-digits,small]{eulervm}

% Command initialization
\DeclareMathAlphabet{\pazocal}{OMS}{zplm}{m}{n}
\graphicspath{{./images/}}

% Custom Commands
\newcommand{\bs}[1]{\boldsymbol{#1}}
\newcommand{\E}{\mathbb{E}}
\newcommand{\var}[1]{\text{Var}\left(#1\right)}
\newcommand{\bp}[1]{\left({#1}\right)}
\newcommand{\mbb}[1]{\mathbb{#1}}
\newcommand{\1}[1]{\mathbbm{1}_{#1}}
\newcommand{\mc}[1]{\mathcal{#1}}
\newcommand{\nck}[2]{{#1\choose#2}}
\newcommand{\pc}[1]{\pazocal{#1}}
\newcommand{\ra}[1]{\renewcommand{\arraystretch}{#1}}
\newcommand*{\floor}[1]{\left\lfloor#1\right\rfloor}
\newcommand*{\ceil}[1]{\left\lceil#1\right\rceil}

\DeclareMathOperator{\Var}{Var}
\DeclareMathOperator{\Cov}{Cov}
\DeclareMathOperator{\diag}{diag}

\newtheorem{theorem}{Theorem}
\newtheorem{lemma}{Lemma}

\begin{document}

    \begin{center}
        {\bf\large{MATH 829: FUNCTIONAL ANALYSIS AND QUANTUM MECHANICS}}
        \smallskip
        \hrule
        \smallskip
        {\bf Assignment} 1\hfill {\bf Connor Braun} \hfill {\bf 2024-09-10}
    \end{center}

    \vspace{5pt}
    \noindent {\bf Problem 2} Prove that $\tfrac{a^{(n)}_{m+1}}{a^{(n)}_m}=\frac{2m-1}{2m+1}\frac{m-n}{m+n}$.\\[5pt]
    {\bf Proof}. Fix the truncation $n\geq 1$, and let $\mc{N}:=\{1,2,\dots,n\}$. For $1\leq m\leq n$, define the sets $C_m^-=\{k\in\mc{N}:k<m\}$, $C_m^+=\{k\in\mc{N}:m<k\}$
    and $C_m=C_m^-\cup C_m^+$. Further, define the matrix
    \begin{align*}
        \mc{X}^{(n)}=\begin{pmatrix}
            1 & 1 & 1 & \cdots & 1\\
            1^2 & 3^2 & 5^2 & \cdots & (2n-1)^2\\
            1^4 & 3^4 & 5^4 & \cdots & (2n-1)^4\\
            \vdots & \vdots & \vdots& \ddots & \vdots\\
            1^{2n-2} & 3^{2n-2} & 5^{2n-2} & \cdots & (2n-1)^{2n-2} 
        \end{pmatrix}
    \end{align*}
    and for $i\in\mc{N}$ denote with $\mc{X}^{(n)}_i$ the matrix $\mc{X}^{(n)}$ where column $i$ has been replaced with $(1,0,0,\dots,0)^T\in\mbb{R}^n$. Then all of $\mc{X}^{(n)}$, $\mc{X}^{(n)}_i$ for $i\in\mc{N}$ are
    Vandermonde, and so admit the determinants
    \begin{align*}
        \det(\mc{X}^{(n)})=\prod_{1\leq k<j\leq n}((2j-1)^2-(2k-1)^2),\quad\text{and}\quad\det(\mc{X}_i^{(n)})=\prod_{1\leq k<j\leq n}(x^i_j-x^i_k)
    \end{align*}
    where $x_j^i=(2j-1)^2$ if $j\neq i$ and $x_i^i=0$. By Cramer's rule, the ratio of interest can be written
    \begin{align}
        \frac{a^{(n)}_{m+1}}{a^{(n)}_m}=\frac{\det(\mc{X}^{(n)}_{m+1})}{\det(\mc{X}^{(n)}_m)}&=\frac{\prod_{1\leq i<j\leq n}(x_j^{m+1}-x_i^{m+1})}{\prod_{1\leq i<j\leq n}(x_j^{m}-x_i^{m})}\notag\\
        &=\frac{\prod_{j\in C_{m+1}^-}(-(2j-1)^2)\prod_{j\in C_{m+1}^+}((2j-1)^2)\prod_{\substack{i,j\in C_{m+1} \\ i<j}}((2j-1)^2-(2i-1)^2)}{\prod_{j\in C_{m}^-}(-(2j-1)^2)\prod_{j\in C_{m}^+}((2j-1)^2)\prod_{\substack{i,j\in C_{m} \\ i<j}}((2j-1)^2-(2i-1)^2)}.\label{eq4}
    \end{align}
    Now, the fact that $C_m^-\subset C_{m+1}^-$ and $C_{m+1}^+\subset C_m^+$ while $C_{m+1}^-\setminus C_{m}^-=\{m\}$ and $C_m^+\setminus C_{m+1}^+=\{m+1\}$ means that equation (\ref{eq4}) simplifies to
    \begin{align*}
        \frac{a^{(n)}_{m+1}}{a^{(n)}_m}=\frac{-(2m-1)^2}{(2m+1)^2}\frac{\prod_{j<m}((2m-1)^2-(2j-1)^2)\prod_{j>m}((2j-1)^2-(2m-1)^2)}{\prod_{j<m+1}((2m+1)^2-(2j-1)^2)\prod_{j>m+1}((2j-1)^2-(2m+1)^2)}.
    \end{align*}
    By expanding the quadratic terms in each of the indexed products and simplifying (which is extremely straightforward and tedious, thus omitted) we next obtain
    \begin{align*}
        \frac{a^{(n)}_{m+1}}{a^{(n)}_m}=\frac{-(2m-1)^2}{(2m+1)^2}\frac{\prod_{j<m}4(m-j)[m+j-1]\prod_{j>m}4(j-m)[j+m-1]}{\prod_{j<m+1}4(m+j)[m-j+1]\prod_{j>m+1}4(m+j)[j-m-1]}.
    \end{align*}
    Noticing that the $m$-th factor of the bottom left product is equal to the first factor of the top right, this can be further simplified
    \begin{align*}
        \frac{a^{(n)}_{m+1}}{a^{(n)}_m}&=\frac{-(2m-1)^2}{(2m+1)^2}\frac{\prod_{j<m}(m-j)[m+j-1]\prod_{j>m+1}(j-m)[j+m-1]}{\prod_{j<m}(m+j)[m-j+1]\prod_{j>m+1}(m+j)[j-m-1]}\\
        &=\frac{-(2m-1)^2}{(2m+1)^2}\prod_{j<m}\frac{(m-j)[m+j-1]}{(m+j)[m-j+1]}\prod_{j>m+1}\frac{(j-m)[j+m-1]}{(m+j)[j-m-1]}.
    \end{align*}
    Now, both of $P_m:=\prod_{j<m}\tfrac{(m-j)[m+j-1]}{(m+j)[m-j+1]}$ and $Q_m=\prod_{j>m+1}\tfrac{(j-m)[j+m-1]}{(m+j)[j-m-1]}$ have some nice telescopic properties:
    \begin{align*}
        P_m&=\frac{(m-1)}{(m+1)}\times\frac{(m-2)(m+1)}{(m+2)(m-1)}\times\frac{(m-3)(m+2)}{(m+3)(m-2)}\times\cdots\times\frac{(2)(2m-3)}{(2m-2)(3)}\times\frac{(1)(2m-2)}{(2m-1)(2)}\\
        &=\frac{1}{(2m-1)}\\
        Q_m&=\frac{(2)(2m+1)}{(2m+2)(1)}\times\frac{(3)(2m+2)}{(2m+3)(2)}\times\frac{(4)(2m+3)}{(2m+4)(3)}\times\cdots\times\frac{(n-m-1)(n+m-2)}{(n+m-1)(n-m-2)}\times\frac{(n-m)(n+m-1)}{(m+n)(n-m-1)}\\
        &=\frac{(2m+1)(n-m)}{(m+n)}
    \end{align*}
    and with these we have the result:
    \begin{align*}
        \frac{a^{(n)}_{m+1}}{a^{(n)}_m}=\frac{-(2m-1)^2}{(2m+1)^2}\frac{1}{(2m-1)}\frac{-(m-n)}{(m+n)}=\frac{2m-1}{2m+1}\frac{m-n}{m+n}.\tag*{$\qed$}
    \end{align*}
\end{document}