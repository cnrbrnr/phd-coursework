\documentclass[10pt]{article}
\usepackage[margin=1.3cm]{geometry}

% Packages
\usepackage{amsmath, amsfonts, amssymb, amsthm}
\usepackage{bbm} 
\usepackage{dutchcal} % [dutchcal, calrsfs, pzzcal] calligraphic fonts
\usepackage{graphicx}
\usepackage[T1]{fontenc}
\usepackage[tracking]{microtype}

% Palatino for text goes well with Euler
\usepackage[sc,osf]{mathpazo}   % With old-style figures and real smallcaps.
\linespread{1.025}              % Palatino leads a little more leading

% Euler for math and numbers
\usepackage[euler-digits,small]{eulervm}

% Command initialization
\DeclareMathAlphabet{\pazocal}{OMS}{zplm}{m}{n}
\graphicspath{{./images/}}

% Custom Commands
\newcommand{\bs}[1]{\boldsymbol{#1}}
\newcommand{\E}{\mathbb{E}}
\newcommand{\var}[1]{\text{Var}\left(#1\right)}
\newcommand{\bp}[1]{\left({#1}\right)}
\newcommand{\mbb}[1]{\mathbb{#1}}
\newcommand{\1}[1]{\mathbbm{1}_{#1}}
\newcommand{\mc}[1]{\mathcal{#1}}
\newcommand{\nck}[2]{{#1\choose#2}}
\newcommand{\pc}[1]{\pazocal{#1}}
\newcommand{\ra}[1]{\renewcommand{\arraystretch}{#1}}
\newcommand*{\floor}[1]{\left\lfloor#1\right\rfloor}
\newcommand*{\ceil}[1]{\left\lceil#1\right\rceil}

\DeclareMathOperator{\Var}{Var}
\DeclareMathOperator{\Cov}{Cov}
\DeclareMathOperator{\diag}{diag}

\newtheorem{theorem}{Theorem}
\newtheorem{lemma}{Lemma}

\begin{document}

    \begin{center}
        {\bf\large{MATH 829: FUNCTIONAL ANALYSIS AND QUANTUM MECHANICS}}
        \smallskip
        \hrule
        \smallskip
        {\bf Assignment} 1\hfill {\bf Connor Braun} \hfill {\bf 2024-09-10}
    \end{center}
    \vspace{5pt}
    \noindent {\bf Problem 3} Recall that $a_m=\tfrac{(-1)^m}{(2m-1)}\tfrac{4}{\pi}$, $m\geq 1$ is the solution found by Fourier to the problem
    \begin{align*}
        1=\sum_{m=1}^\infty a_m\cos((2m-1)x),\quad x\in(-\tfrac{\pi}{2},\tfrac{\pi}{2}).
    \end{align*}
    Let $f_m(x)=\tfrac{4}{\pi}\tfrac{(-1)^m}{(2m-1)}\cos((2m-1)x)$ for $x\in(-\tfrac{\pi}{2},\tfrac{\pi}{2})$. Explain why the term-by-term differentiation
    \begin{align*}
        \frac{d}{dx}\sum_{m=1}^\infty f_m(x)=\sum_{m=1}^\infty\frac{d}{dx}f_m(x)
    \end{align*}
    does not follow the Weierstrass M-test.\\[5pt]
    {\bf Solution}. Using the method hinted at in the question, one hopes to apply theorem 6.11 found in the MATH 281 reader to establish the validity of this interchange.
    Since the $f_m\in C^1((-\tfrac{\pi}{2},\tfrac{\pi}{2}))$ for $m\geq 1$, this requires one to certify that
    \begin{align*}
        \sum_{m=1}^n\frac{d}{dx}f_m(x)\overset{?}{\rightarrow} g(x)\quad\text{uniformly on $(-\tfrac{\pi}{2},\tfrac{\pi}{2})$, as}\quad n\rightarrow\infty.
    \end{align*}
    However, this is not possible using the Weierstrass M-test. To see this, note that
    \[\frac{d}{dx}f_m(x)=\frac{4}{\pi}\frac{(-1)^{m+1}}{(2m-1)}\sin((2m-1)x)(2m-1)=\frac{4}{\pi}(-1)^{m+1}\sin((2m-1)x)\]
    and fix the point $x^\ast=\tfrac{\pi}{4}\in(-\tfrac{\pi}{2},\tfrac{\pi}{2})$. Then for $m\geq 1$, $\sin((2m-1)x^\ast)=\pm\tfrac{\sqrt{2}}{2}$ (and is in fact $4$-periodic in $m$). At this point
    \begin{align*}
        \left|\frac{d}{dx}f_m(x)\big|_{x=x^\ast}\right|=\frac{2\sqrt{2}}{\pi}
    \end{align*}
    and so the sequence $\{\tfrac{d}{dx}f_m\}_{m\geq 1}$ cannot even be pointwise bounded by the terms of a converging series, let alone uniformly as
    the Weierstrass M-test requires. Driving this point home, we have
    \begin{align*}
        \infty=\sum_{m=1}^\infty\frac{1}{\pi}\leq \sum_{m=1}^\infty\frac{d}{dx}f_m(x)\big|_{x=x^\ast}
    \end{align*}
    so there is not sequence $\{M_m\}_{m\geq 1}$ satisfying $|f_m(x)|\leq M_m$ for $x\in(-\tfrac{\pi}{2},\tfrac{\pi}{2})$ such that $\sum_{m=1}^\infty M_m<\infty$.\hfill{$\qed$}\\[5pt]
\end{document}