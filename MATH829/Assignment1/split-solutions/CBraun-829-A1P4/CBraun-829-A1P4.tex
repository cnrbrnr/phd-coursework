\documentclass[10pt]{article}
\usepackage[margin=1.3cm]{geometry}

% Packages
\usepackage{amsmath, amsfonts, amssymb, amsthm}
\usepackage{bbm} 
\usepackage{dutchcal} % [dutchcal, calrsfs, pzzcal] calligraphic fonts
\usepackage{graphicx}
\usepackage[T1]{fontenc}
\usepackage[tracking]{microtype}

% Palatino for text goes well with Euler
\usepackage[sc,osf]{mathpazo}   % With old-style figures and real smallcaps.
\linespread{1.025}              % Palatino leads a little more leading

% Euler for math and numbers
\usepackage[euler-digits,small]{eulervm}

% Command initialization
\DeclareMathAlphabet{\pazocal}{OMS}{zplm}{m}{n}
\graphicspath{{./images/}}

% Custom Commands
\newcommand{\bs}[1]{\boldsymbol{#1}}
\newcommand{\E}{\mathbb{E}}
\newcommand{\var}[1]{\text{Var}\left(#1\right)}
\newcommand{\bp}[1]{\left({#1}\right)}
\newcommand{\mbb}[1]{\mathbb{#1}}
\newcommand{\1}[1]{\mathbbm{1}_{#1}}
\newcommand{\mc}[1]{\mathcal{#1}}
\newcommand{\nck}[2]{{#1\choose#2}}
\newcommand{\pc}[1]{\pazocal{#1}}
\newcommand{\ra}[1]{\renewcommand{\arraystretch}{#1}}
\newcommand*{\floor}[1]{\left\lfloor#1\right\rfloor}
\newcommand*{\ceil}[1]{\left\lceil#1\right\rceil}

\DeclareMathOperator{\Var}{Var}
\DeclareMathOperator{\Cov}{Cov}
\DeclareMathOperator{\diag}{diag}

\newtheorem{theorem}{Theorem}
\newtheorem{lemma}{Lemma}

\begin{document}

    \begin{center}
        {\bf\large{MATH 829: FUNCTIONAL ANALYSIS AND QUANTUM MECHANICS}}
        \smallskip
        \hrule
        \smallskip
        {\bf Assignment} 1\hfill {\bf Connor Braun} \hfill {\bf 2024-09-10}
    \end{center}
    \vspace{5pt}
    \noindent{\bf Problem 4} Solve the following two subproblems, and suppose $\mbb{K}$ is a field.\\[5pt]
    {\bf a)} Let $X=\{a:\mbb{N}\rightarrow\mbb{K}|a=(a_m)_{m\geq 1}\}$ and $E=\{x\in X:x=(x_1,0,x_3,x_4,\dots)\}$. Compute $\dim(X)$, $\dim(E)$ and $\dim(X/E)$.\\[5pt]
    {\bf Solution}. For $n\geq 1$, define $e^n=(e^n_m)_{M\geq 1}\in X$ so that $e^n_n=1$ and $e^n_m=0$ if $n\neq m$. Then, fix $N\in\mbb{N}$, and scalars $\{\alpha_m\}_{m=1}^N$. Suppose that
    \[\sum_{m=1}^N\alpha_me^m=0.\] 
    With this, we obtain
    \begin{align*}
        0=\sum_{m=1}^N\alpha_me^m=(\alpha_1e^1_1,\alpha_2e^2_2,\dots,\alpha_Ne^N_N,0,0,\dots)=(\alpha_1,\alpha_2,\dots,\alpha_N,0,0,\dots)
    \end{align*}
    so indeed $a_m=0$ for $1\leq m\leq N$, and $\{e^m\}_{m=1}^N$ are therefore linearly independent. Since this holds for arbitrary $N\in\mbb{N}$, we conclude that $\dim(X)=\infty$.
    Now for $N\in\mbb{N}$, define $B^N=\{u_j\}_{j=1}^N=\{e^m\}_{\substack{m=1 \\ m\neq 2}}^{N+1}\subset E$. In precisely the same manner,
    \begin{align*}
        0=\sum_{j=1}^N\alpha_ju_j=\sum_{m\neq 2}^{N+1}\alpha_me^m=(\alpha_1e^1_1,0,\alpha_2e^3_3,\alpha_3e^4_4,\dots,\alpha_Ne^{N+1}_{N+1},0,0,\dots)=(\alpha_1,0,\alpha_2,\alpha_3,\dots,\alpha_N,0,0,\dots)
    \end{align*}
    and again, it must be that $\alpha_m=0$ for $1\leq m\leq N$, whereby the arbitrariness of $N$, $\dim(E)=\infty$. Lastly, define $u=(0,1,0,0,\dots)\in X$. Take $\lambda\in\mbb{K}$, and suppose that $\lambda u\in E$. Then
    $\lambda u=(0,\lambda,0,0,\dots)\in E\Rightarrow \lambda=0$, so $\{u\}$ is linearly independent relative to $E$. Now, fix some $v=(v_1,v_2,\dots)\in X$, and note that $v$ admits the decomposition
    $v=v_2u+w$, where $w:=(v_1,0,v_3,v_4,\dots)\in E$. To see that the pair $(v_2,w)$ is unique for this decomposition, suppose we have another $(v_2^\prime,w^\prime)$. Then
    \begin{align*}
        v_2u+w=v_2^\prime u+w^\prime\quad\Rightarrow\quad u(v_2-v_2^\prime)+(w-w^\prime)=0.
    \end{align*}
    which implies $u(v_2-v_2^\prime)$ is the additive inverse of $(w-w^\prime)\in E$, so $u(v_2-v_2^\prime)\in E\Rightarrow (v_2-v_2^\prime)=0$, and subsequently $w=w^\prime$. Thus, $\{u\}$ is a basis of $X/E$, so $\dim(X/E)=1$.\hfill{$\qed$}\\[5pt]
    {\bf b)} Let $X$ be the space of all bi-infinite, $\mbb{K}$-valued sequences $x=(x_n)_{n=-\infty}^{\infty}$ such that each of the limits
    \begin{align*}
        \lim_{k\rightarrow\infty}x_{2k},\quad\lim_{k\rightarrow\infty}x_{2k+1},\quad\text{and}\quad\lim_{k\rightarrow-\infty}x_k
    \end{align*}
    exist. Let $E$ be the space of all bi-infinite $\mbb{K}$-valued sequences such that $\lim_{k\rightarrow-\infty}x_k=\lim_{k\rightarrow\infty}x_k=0$. Prove that $E$ is a subspace of $X$
    and find both a basis and the dimension of $X/E$.\\[5pt]
    {\bf Solution}. First, notice that $E\subseteq X$, since $\lim_{k\rightarrow\infty}x_k=0$ implies both $\lim_{k\rightarrow\infty}x_{2k}=0$ and $\lim_{k\rightarrow\infty}x_{2k+1}=0$. Further, the set $E$ is closed under the usual
    elementwise addition and scalar multiplication operations. Taking $\alpha\in\mbb{K}$ and $v,w\in E$,
    \begin{align*}
        \lim_{k\rightarrow\pm\infty}(\alpha v+w)=\alpha\lim_{k\rightarrow\infty}v_k+\lim_{k\rightarrow\infty}w_k=\alpha\cdot0+0=0
    \end{align*}
    so $\alpha v+w\in E$. Thus, $E$ is a subspace of $X$. For the second objective, define the bi-infinite sequences $b^1=(b^1_m)_{m=-\infty}^\infty$, $b^2=(b^2_m)_{m=-\infty}^\infty$, and $b^3=(b^3_m)_{m=-\infty}^\infty$ with
    \begin{align*}
        b^1_m=\begin{cases}
            1,\quad&\text{if $m<0$}\\
            0,\quad&\text{o.w.}
        \end{cases},\quad
        b^2_m=\begin{cases}
            1,\quad&\text{if $2|m$, $m\geq 0$}\\
            0,\quad&\text{o.w.}
        \end{cases},\quad\text{and}\quad
        b^3_m=\begin{cases}
            1,\quad&\text{if $2\nmid m$, $m\geq 0$}\\
            0,\quad&\text{o.w.}
        \end{cases}.
    \end{align*}
    All of these are clearly in $X$, as they reduce to a sequence of either $1$'s or $0$'s in each of the qualifying limits. Now, to show linear independence relative to $E$, take scalars $\lambda_1,\lambda_2,\lambda_3\in\mbb{K}$, and suppose that $\sum_{k=1}^3\lambda_kb^k\in E$. Then we have
    \begin{align*}
        \lim_{k\rightarrow-\infty}(\lambda_1b^1_k+\lambda_2b^2_k+\lambda_3b^3_k)=0\quad&\Rightarrow\quad\lambda_1(\lim_{k\rightarrow-\infty}b^1_k)+\lambda_2(\lim_{k\rightarrow -\infty}b^2_k)+\lambda_3(\lim_{k\rightarrow -\infty}b^3_k)=0\\
        &\Rightarrow\quad\lambda_1+0+0=0\\
        &\Rightarrow\quad\lambda_1=0.
    \end{align*}
    and similarly,
    \begin{align*}
        \lim_{k\rightarrow\infty}(\lambda_1b^1_k+\lambda_2b^2_k+\lambda_3b^3_k)=0\quad&\Rightarrow\quad\lambda_2(\lim_{k\rightarrow \infty}b^2_k)+\lambda_3(\lim_{k\rightarrow \infty}b^3_k)=0
    \end{align*}
    but this time, neither $\lim_{k\rightarrow\infty}b^2_k$ nor $\lim_{k\rightarrow\infty}b^3_k$ exist, necessitating $\lambda_2=\lambda_3=0$ for the combination to even reside in $E$ in the first place.
    Since it must be that $\lambda_1=\lambda_2=\lambda_3=0$, $\{b^1,b^2,b^3\}$ is found linearly independent relative to $E$. Now, take $x=\{x_m\}_{m=-\infty}^\infty\in X$ and define the scalars
    \begin{align*}
        \lim_{k\rightarrow-\infty}x_k=C^-,\quad\lim_{k\rightarrow\infty}x_{2k}=C^e,\quad\text{and}\quad\lim_{k\rightarrow\infty}x_{2k+1}=C^o.
    \end{align*}
    Then, observe that $x$ admits the decomposition
    \begin{align*}
        x=(C^-b^1+C^eb^2+C^ob^3)+(x-C^-b^1-C^eb^2-C^ob^3).
    \end{align*}
    Taking some limits of the second term, we have
    \begin{align*}
        \lim_{k\rightarrow-\infty}(x_k-C^-b^1_k-C^eb^2_k-C^ob^3_k)&=C^--C^-+0+0=0\\
        \lim_{k\rightarrow\infty}(x_{2k}-C^-b^1_{2k}-C^eb^2_{2k}-C^ob^3_{2k})&=C^e-0-C^e-0=0\\
        \lim_{k\rightarrow\infty}(x_{2k+1}-C^-b^1_{2k+1}-C^eb^2_{2k+1}-C^ob^3_{2k+1})&=C^o-0-0-C^o=0
    \end{align*}
    where since the latter two limits agree, we have $\lim_{k\rightarrow\infty}(x-C^-b^1-C^eb^2-C^ob^3)=0$, so $(x-C^-b^1-C^eb^2-C^ob^3)\in E$. 
    The uniqueness of the above decomposition for $x$ follows precisely the same reasoning as was demonstrated in part (a) of this problem.
    Since such a unique decomposition exists $\forall x\in X$, we conclude the set $\{b^1,b^2,b^3\}$ is a basis of $X/E$, and moreover $\dim(X/E)=3$.\hfill{$\qed$}
\end{document}