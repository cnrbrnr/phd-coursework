\documentclass[10pt]{article}
\usepackage[margin=1.3cm]{geometry}

% Packages
\usepackage{amsmath, amsfonts, amssymb, amsthm}
\usepackage{bbm} 
\usepackage{dutchcal} % [dutchcal, calrsfs, pzzcal] calligraphic fonts
\usepackage{graphicx}
\usepackage[T1]{fontenc}
\usepackage[tracking]{microtype}

% Palatino for text goes well with Euler
\usepackage[sc,osf]{mathpazo}   % With old-style figures and real smallcaps.
\linespread{1.025}              % Palatino leads a little more leading

% Euler for math and numbers
\usepackage[euler-digits,small]{eulervm}

% Command initialization
\DeclareMathAlphabet{\pazocal}{OMS}{zplm}{m}{n}
\graphicspath{{./images/}}

% Custom Commands
\newcommand{\bs}[1]{\boldsymbol{#1}}
\newcommand{\E}{\mathbb{E}}
\newcommand{\var}[1]{\text{Var}\left(#1\right)}
\newcommand{\bp}[1]{\left({#1}\right)}
\newcommand{\mbb}[1]{\mathbb{#1}}
\newcommand{\1}[1]{\mathbbm{1}_{#1}}
\newcommand{\mc}[1]{\mathcal{#1}}
\newcommand{\nck}[2]{{#1\choose#2}}
\newcommand{\pc}[1]{\pazocal{#1}}
\newcommand{\ra}[1]{\renewcommand{\arraystretch}{#1}}
\newcommand*{\floor}[1]{\left\lfloor#1\right\rfloor}
\newcommand*{\ceil}[1]{\left\lceil#1\right\rceil}

\DeclareMathOperator{\Var}{Var}
\DeclareMathOperator{\Cov}{Cov}
\DeclareMathOperator{\diag}{diag}

\newtheorem{theorem}{Theorem}
\newtheorem{lemma}{Lemma}

\begin{document}

    \begin{center}
        {\bf\large{MATH 829: FUNCTIONAL ANALYSIS AND QUANTUM MECHANICS}}
        \smallskip
        \hrule
        \smallskip
        {\bf Assignment} 2\hfill {\bf Connor Braun} \hfill {\bf 2024-09-19}
    \end{center}
    \vspace{5pt}
    \begin{center}
        \begin{minipage}{\dimexpr\paperwidth-10cm}
            I did not work with anybody in completing this assignment.
        \end{minipage}
    \end{center}
    \noindent\noindent{\bf Problem 1} Prove the following:
    \begin{theorem}
        For $1\leq p<\infty$, the normed vector space $(\ell_p,\|\cdot\|_p)$ is
        Banach.
    \end{theorem} 
    \noindent{\bf a)} Assume that the sequence $(x^n)_{n\geq 1}\subset\ell_p$ is
    Cauchy. State what this means with the $\varepsilon-N$ definition.\\[5pt]
    {\bf Solution}. For any $\varepsilon>0$, $\exists N\geq 1$ s.t.
    \[n,m\geq N\quad\Rightarrow\quad \|x^n-x^m\|_p<\varepsilon\] so, in
    particular, with $x^n=(x^n_j)_{j\geq 1}$ for $n\geq 1$ (a convention I will
    use throughout this assignment) we have
    \begin{align}
        \left(\sum_{j\geq 1}|x^n_j-x^m_j|^p\right)^{1/p}<\varepsilon.\label{eq1}
    \end{align}
    \hfill{$\qed$}\\[5pt]
    {\bf b)} Fix $\varepsilon>0$. Explain why $\exists N\geq 1$ such that for
    each $j\geq 1$ we have $|x^m_j-x^n_j|<\varepsilon$, provided $n,m\geq
    N$.\\[5pt]
    {\bf Solution}. From the Cauchy condition in (a) (and inequality
    (\ref{eq1}), in particular), we have
    \begin{align}
        |x^n_j-x^m_j|^p\leq\sum_{k\geq1}|x^n_k-x^m_k|^p<(\varepsilon^\prime)^p\label{eq2}
    \end{align}
    for any $j\geq 1$, $\varepsilon^\prime>0$, provided $n,m\geq
    N(\varepsilon^\prime)$. This is because $|\cdot|$ is nonnegative on
    $\mbb{K}$. Simply taking $\varepsilon^\prime=\varepsilon$, $n,m\geq
    N(\varepsilon^\prime)=N$, the above certifies that
    $|x^n_j-x^m_j|<\varepsilon$ for $n,m\geq N$, $j\geq 1$.\hfill{$\qed$}\\[5pt]
    {\bf c)} Deduce that, for each $j\geq 1$, the $\mbb{K}$-valued sequence
    $(x^n_j)_{n\geq 1}$ must be convergent. For each $j\geq 1$, define
    $x_j:=\lim_{n\rightarrow\infty}x^n_j$, a limit in
    $(\mbb{K},|\cdot|)$.\\[5pt]
    {\bf Solution}. Part (b) certifies that for any $j\geq 1$, $(x^n_j)_{n\geq
    1}$ is Cauchy. That is, for any $\varepsilon>0$ one can find $N\geq 1$ so
    that $n,m\geq N$ implies $|x^n_j-x^m_j|<\varepsilon$. Assuming $\mbb{K}$ is
    complete, we deduce that $x^n_j\rightarrow x_j$ as
    $n\rightarrow\infty$.\hfill{$\qed$}\\[5pt]
    {\bf d)} Let $J\geq 1$ and explain why, with $N$ as in (b), we have
    $\sum_{j=1}^J|x^m_j-x^n_j|^p<\varepsilon^p$ provided $n,m\geq N$.\\[5pt]
    {\bf Solution}. As previously, $|\cdot|$ is nonnegative on $\mbb{K}$, so
    with $J\geq 1$:
    \begin{align}
        \sum_{j=1}^J|x^m_j-x^n_j|^p\leq\sum_{j=1}^\infty|x^n_j-x^m_j|^p<\varepsilon^p\label{eq4}
    \end{align}
    with the second inequality following (\ref{eq2}). \hfill{$\qed$}\\[5pt]
    {\bf e)} With $x_j$ as in (c), let $n\rightarrow\infty$ to obtain
    $\sum_{j=1}^J|x^m_j-x_j|^p<\varepsilon^p$, provided $m\geq N$. Deduce that,
    for $m\geq N$, we have
    \begin{align}
        \left(\sum_{j=1}^J|x_j|^p\right)^{1/p}<\varepsilon+\left(\sum_{j=1}^J|x^m_j|^p\right)^{1/p}.\label{eq3}
    \end{align}
    \noindent{\bf Solution}. Take $J\geq 1$, $N$ as in (b), and $n,m\geq N$.
    Then we compute the limit of (\ref{eq4}):
    \begin{align*}
        \lim_{n\rightarrow\infty}\sum_{j=1}^J|x^m_j-x^n_j|^p\leq\varepsilon^p\quad\Rightarrow\quad\sum_{j=1}^J\lim_{n\rightarrow\infty}|x^m_j-x^n_j|^p\leq\varepsilon^p\quad\Rightarrow\quad\sum_{j=1}^J|x^m_j-x_j|^p\leq\varepsilon^p
    \end{align*}
    where the last implication follows the continuity of $|\cdot|$. But then we
    have
    \begin{align*}
        \left(\sum_{j=1}^J|x_j|^p\right)^{1/p}=\left(\sum_{j=1}^J|x^m_j-x_j-x^m_j|^p\right)^{1/p}&\leq\left(\sum_{j=1}^J|x^m_j|^p\right)^{1/p}+\left(\sum_{j=1}^J|x^m_j-x_j|^p\right)^{1/p}<\varepsilon+\left(\sum_{j=1}^J|x^m_j|^p\right)^{1/p}
    \end{align*}
    where the first inequality follows the triangle inequality for the $p$-norm
    on $\mbb{K}^J$, and the second relies upon $m\geq N$. \hfill{$\qed$}\\[5pt]
    {\bf f)} Let $J\rightarrow\infty$ in the second inequality of (e) to deduce
    that the sequence $x=(x_j)_{j\geq 1}$ belongs to $\ell_p$.\\[5pt]
    {\bf Solution}. For any $m\geq 1$, $x^m\in\ell_p$, so $\|x^m\|_p<\infty$.
    Following the suggestion with $m\geq N$:
    \begin{align*}
        \lim_{J\rightarrow\infty}\left(\sum_{j=1}^J|x_j|^p\right)^{1/p}\leq\varepsilon+\lim_{J\rightarrow\infty}\left(\sum_{j=1}^J|x^m_j|^p\right)^{1/p}\quad\Rightarrow\quad \left(\sum_{j=1}^\infty|x_j|^p\right)^{1/p}\leq\varepsilon+\left(\sum_{j=1}^\infty|x_j^m|^p\right)^{1/p}<\infty
    \end{align*}
    which says that $\|x\|_p<\infty$. Since $x_j\in\mbb{K}$ for $j\geq 1$, this
    means $x\in\ell_p$.\hfill{$\qed$}\\[5pt]
    {\bf g)} Let $J\rightarrow\infty$ in the first inequality of (e) to deduce
    that $\lim_{m\rightarrow\infty}x^m=x$. Explain why this concludes the proof
    of the theorem.\\[5pt]
    {\bf Solution}. Just as prescribed, when $m\geq N$:
    \begin{align}
        \lim_{J\rightarrow\infty}\sum_{j=1}^J|x^m_j-x_j|^p\leq\lim_{J\rightarrow\infty}\varepsilon^p\quad\Rightarrow\quad\sum_{j=1}^\infty|x^m_j-x_j|^p\leq\varepsilon^p\Rightarrow\|x^m-x\|_p\leq\varepsilon.\label{eq5}
    \end{align}
    Now, since $\varepsilon$ was arbitrary, this says that $x^m\rightarrow
    x\in(\ell_p,\|\cdot\|_p)$. Since we accept that $(\ell_p,\|\cdot\|_p)$ is a
    normed vector space {\it a priori}, to show it is Banach requires only that
    it is complete -- that all Cauchy sequences are convergent. But (\ref{eq5})
    was derived only under the assumption that $(x^m)_{m\geq 1}$ is Cauchy, so
    we are indeed done.\hfill{$\qed$}
\end{document}