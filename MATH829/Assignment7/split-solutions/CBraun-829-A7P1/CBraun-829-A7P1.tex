\documentclass[10pt]{article}
\usepackage[margin=1.3cm]{geometry}

% Packages
\usepackage{amsmath, amsfonts, amssymb, amsthm}
\usepackage{bbm} 
\usepackage{dutchcal} % [dutchcal, calrsfs, pzzcal] calligraphic fonts
\usepackage{graphicx}
\usepackage[T1]{fontenc}
\usepackage[tracking]{microtype}

% Palatino for text goes well with Euler
\usepackage[sc,osf]{mathpazo}   % With old-style figures and real smallcaps.
\linespread{1.025}              % Palatino leads a little more leading

% Euler for math and numbers
\usepackage[euler-digits,small]{eulervm}

% Command initialization
\DeclareMathAlphabet{\pazocal}{OMS}{zplm}{m}{n} \graphicspath{{./images/}}

% Custom Commands
\newcommand{\bs}[1]{\boldsymbol{#1}} \newcommand{\E}{\mathbb{E}}
\newcommand{\var}[1]{\text{Var}\left(#1\right)}
\newcommand{\bp}[1]{\left({#1}\right)} \newcommand{\mbb}[1]{\mathbb{#1}}
\newcommand{\1}[1]{\mathbbm{1}_{#1}} \newcommand{\mc}[1]{\mathcal{#1}}
\newcommand{\nck}[2]{{#1\choose#2}} \newcommand{\pc}[1]{\pazocal{#1}}
\newcommand{\ra}[1]{\renewcommand{\arraystretch}{#1}}
\newcommand*{\floor}[1]{\left\lfloor#1\right\rfloor}
\newcommand*{\ceil}[1]{\left\lceil#1\right\rceil}
\newcommand{\ip}[2]{\left\langle#1,#2\right\rangle }

\DeclareMathOperator{\Var}{Var} \DeclareMathOperator{\Cov}{Cov}
\DeclareMathOperator{\diag}{diag}
\DeclareMathOperator{\ran}{range}

\makeatletter
\def\Ddots{\mathinner{\mkern1mu\raise\p@
\vbox{\kern7\p@\hbox{.}}\mkern2mu
\raise4\p@\hbox{.}\mkern2mu\raise7\p@\hbox{.}\mkern1mu}}
\makeatother

\def\powertower#1#2{#1\ifnum#2>1 ^{\powertower{#1}{\numexpr#2-1\relax}}\fi}

\newtheorem{theorem}{Theorem}
\newtheorem{lemma}{Lemma}

\begin{document}

    \begin{center}
        {\bf\large{MATH 829: FUNCTIONAL ANALYSIS AND QUANTUM MECHANICS}}
        \smallskip
        \hrule
        \smallskip
        {\bf Assignment} 7\hfill {\bf Connor Braun} \hfill {\bf 2024-10-30}
    \end{center}
    \vspace{5pt}
    \begin{center}
        \begin{minipage}{\dimexpr\paperwidth-10cm}
            I would like to thank Ben van Eden for the discussion regarding problem 3. In particular, for the construction of $(x^k)_{k\geq 1}$ 
            used to show that $\mc{D}(A^\ast)\subseteq\mc{D}(A)$. All other problems were completed independently.
        \end{minipage}
    \end{center}
    \noindent{\bf Problem 1}. Let $(X,\ip{\cdot}{\cdot})$ be a Hilbert space, and
    $A:\mc{D}(A)\subset X\rightarrow X$ be a densely defined linear operator.
    Prove that $(\ran(A))^\perp=\ker(A^\ast)$.\\[5pt]
    {\bf Proof}. Recalling that $\ker(A^\ast):=\{x\in\mc{D}(A^\ast):A^\ast
    x=0\}$, take $y\in\ran(A)$. Then $y=Az$ for some $z\in\mc{D}(A)$, and
    \[\ip{x}{y}=\ip{x}{Az}=\overline{\ip{Az}{x}}=\overline{\ip{z}{A^\ast x}}=0\]
    where the penultimate equality holds since $x\in\ker(A^\ast)\subseteq
    \mc{D}(A^\ast)$ and $z\in\mc{D}(A)$. Thus, $x\in(\ran(A))^\perp$, and
    $\ker(A^\ast)\subseteq(\ran(A))^\perp$. For the reverse inclusion, take
    $y\in(\ran(A))^\perp$. It is easy to see that $y\in\mc{D}(A^\ast)$, since
    for any $x\in\mc{D}(A)$
    \[|\ip{Ax}{y}|=0\leq 1\cdot\|x\|\] where the first equality
    holds since $Ax\in\ran(A)$ and $y\perp\ran(A)$. But then, for all
    $z\in\mc{D}(A)$:
    \[0=\ip{Az}{x}=\ip{z}{A^\ast x}\] so $A^\ast x\perp\mc{D}(A)$. Since
    $\mc{D}(A)$ is dense in $X$, we conclude that $A^\ast x=0$, so
    $x\in\ker(A^\ast)$. Thus, $(\ran(A))^\perp\subseteq\ker(A^\ast)$, so
    $(\ran(A))^\perp=\ker(A^\ast)$.\hfill{$\qed$}\\[5pt]
\end{document}