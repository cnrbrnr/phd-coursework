\documentclass[10pt]{article}
\usepackage[margin=1.3cm]{geometry}

% Packages
\usepackage{amsmath, amsfonts, amssymb, amsthm}
\usepackage{bbm} 
\usepackage{dutchcal} % [dutchcal, calrsfs, pzzcal] calligraphic fonts
\usepackage{graphicx}
\usepackage[T1]{fontenc}
\usepackage[tracking]{microtype}

% Palatino for text goes well with Euler
\usepackage[sc,osf]{mathpazo}   % With old-style figures and real smallcaps.
\linespread{1.025}              % Palatino leads a little more leading

% Euler for math and numbers
\usepackage[euler-digits,small]{eulervm}

% Command initialization
\DeclareMathAlphabet{\pazocal}{OMS}{zplm}{m}{n} \graphicspath{{./images/}}

% Custom Commands
\newcommand{\bs}[1]{\boldsymbol{#1}} \newcommand{\E}{\mathbb{E}}
\newcommand{\var}[1]{\text{Var}\left(#1\right)}
\newcommand{\bp}[1]{\left({#1}\right)} \newcommand{\mbb}[1]{\mathbb{#1}}
\newcommand{\1}[1]{\mathbbm{1}_{#1}} \newcommand{\mc}[1]{\mathcal{#1}}
\newcommand{\nck}[2]{{#1\choose#2}} \newcommand{\pc}[1]{\pazocal{#1}}
\newcommand{\ra}[1]{\renewcommand{\arraystretch}{#1}}
\newcommand*{\floor}[1]{\left\lfloor#1\right\rfloor}
\newcommand*{\ceil}[1]{\left\lceil#1\right\rceil}
\newcommand{\ip}[2]{\left\langle#1,#2\right\rangle }

\DeclareMathOperator{\Var}{Var} \DeclareMathOperator{\Cov}{Cov}
\DeclareMathOperator{\diag}{diag}
\DeclareMathOperator{\ran}{range}

\makeatletter
\def\Ddots{\mathinner{\mkern1mu\raise\p@
\vbox{\kern7\p@\hbox{.}}\mkern2mu
\raise4\p@\hbox{.}\mkern2mu\raise7\p@\hbox{.}\mkern1mu}}
\makeatother

\def\powertower#1#2{#1\ifnum#2>1 ^{\powertower{#1}{\numexpr#2-1\relax}}\fi}

\newtheorem{theorem}{Theorem}
\newtheorem{lemma}{Lemma}

\begin{document}

    \begin{center}
        {\bf\large{MATH 829: FUNCTIONAL ANALYSIS AND QUANTUM MECHANICS}}
        \smallskip
        \hrule
        \smallskip
        {\bf Assignment} 7\hfill {\bf Connor Braun} \hfill {\bf 2024-10-30}
    \end{center}
    \noindent{\bf Problem 3}. Let $A:\mc{D}(A)\subset \ell_2\rightarrow \ell_2$ be the operator so that $Ax=(nx_n)_{n\geq 1}$ for any $x=(x_n)_{n\geq 1}\in\mc{D}(A)$. We define
    \begin{align*}
        \mc{D}(A):=\left\{x\in\ell_2:\sum_{n=1}^\infty|nx_n|^2<\infty\right\}.
    \end{align*} 
    Prove that $A$ is self-adjoint.\\[5pt]
    {\bf Proof}. We first show that $A$ is densely defined. Let $z=(z_n)_{n\geq 1}\in\ell_2$, and define $x^k=(x^k_n)_{n\geq 1}$ so that $x_n^k=z_n$ for $1\leq n\leq k$ and $x^k_n=0$ when $n>k$. Clearly $\{x^k\}_{N\geq 1}\subseteq\mc{D}(A)$, since
    \begin{align*}
        \sum_{n=1}^\infty|nx^k_n|^2=\sum_{n=1}^k|nx^k_n|^2<\infty
    \end{align*}
    since this is a finite sum of finite terms. However:
    \begin{align*}
        \|z-x^k\|^2=\sum_{n=1}^\infty|z_n-x^k_n|^2=\sum_{n>k}|z_n|^2=\sum_{n=1}^\infty|z_n|^2-\sum_{n=1}^k|z_n|^2,\quad k\geq 1
    \end{align*}
    so $\lim_{k\rightarrow\infty}\|z-x^k\|^2=\|z\|^2-\|z\|^2=0$, so $x^k\rightarrow z$ as $k\rightarrow\infty$, and we conclude that $\mc{D}(A)$ is dense in $\ell_2$.
    Now take $x,y\in\mc{D}(A)$ arbitrary. We find
    \begin{align*}
        |\ip{Ax}{y}|=\left|\sum_{n=1}^\infty nx_n\overline{y_n}\right|\leq\sum_{n=1}^\infty |x_n n\overline{y_n}|\leq\left(\sum_{n=1}^\infty|x_n|^2\right)^{1/2}\left(\sum_{n=1}^\infty|ny_n|^2\right)^{1/2}=:C_y\|x\|
    \end{align*}
    where $0<C_y<\infty$ since $y\in\mc{D}(A)$. But this says that $y\in\mc{D}(A^\ast)$, so $\mc{D}(A)\subseteq\mc{D}(A^\ast)$. Now, take $y\in\mc{D}(A^\ast)$ and consider the sequence $x^k=(x^k_n)_{n\geq 1}$ with
    \begin{align*}
        x^k_n=\begin{cases}
            \frac{ny_n}{\left(\sum_{n=1}^k|ny_n|^2\right)^{1/2}},\quad&\text{for $1\leq n\leq k$}\\
            0,\quad&\text{when $n>k$.}
        \end{cases}
    \end{align*}
    In particular, it is easy to see that
    \[\|x^k\|^2=\frac{1}{\left(\sum_{n=1}^k|ny_n|^2\right)}\sum_{n=1}^\infty |x^k_n|^2=\frac{1}{\left(\sum_{n=1}^k|ny_n|^2\right)}\sum_{n=1}^k |ny_n|^2=1,\quad k\geq 1.\]
    Furthermore, $x^k\in\mc{D}(A)$ for any $k\geq 1$ since 
    \begin{align*}
        \sum_{n=1}^\infty |nx^k_n|^2=\frac{1}{\left(\sum_{n=1}^k|ny_n|^2\right)}\sum_{n=1}^kn^4|y_n|^2<\infty
    \end{align*}
    once again because the sum is finite. Now, we find
    \begin{align*}
        \frac{1}{\left(\sum_{n=1}^k|ny_n|^2\right)^{1/2}}\sum_{n=1}^k|ny_n|^2&=\frac{1}{\left(\sum_{n=1}^k|ny_n|^2\right)^{1/2}}\left|\sum_{n=1}^kn(ny_n)(\overline{y_n})\right|\\
        &=\left|\sum_{n=1}^knx^k_n(\overline{y_n})\right|\tag{5}\\
        &=\left|\sum_{n=1}^\infty nx^k_n\overline{y_n}\right|\tag{6}\\
        &=\left|\ip{Ax^k}{y}\right|\\
        &\leq C_y\|x^k\|\tag{7}\\
        &=C_y.\tag{8}
    \end{align*}
    To elaborate: (5) holds by moving the fraction into the sum, (6) since $x^k_n=0$ for $n>k$ and (7) since $y\in\mc{D}(A^\ast)$ and $x^k\in\mc{D}(A)$. The constant $0<C_y<\infty$ is furnished by the definition of $y\in\mc{D}(A^\ast)$, and (8)
    follows since $\|x^k\|=1$ for $k\geq 1$. Taking limits,
    \begin{align*}
        \lim_{k\rightarrow\infty}\frac{1}{\left(\sum_{n=1}^k|ny_n|^2\right)^{1/2}}\sum_{n=1}^k|ny_n|^2\leq C_y\quad\Rightarrow\quad\frac{1}{\left(\sum_{n=1}^\infty|ny_n|^2\right)^{1/2}}\sum_{n=1}^\infty|ny_n|^2\leq C_y\quad\Rightarrow\quad \left(\sum_{n=1}^\infty|ny_n|^2\right)^{1/2}<C_y<\infty
    \end{align*}
    so $y\in\mc{D}(A)$, and $\mc{D}(A)\subseteq\mc{D}(A^\ast)$, and we conclude $\mc{D}(A)=\mc{D}(A^\ast)$. Finally, we observe that for any $x,y\in\mc{D}(A)=\mc{D}(A^\ast)$,
    \begin{align*}
        \ip{Ax}{y}=\sum_{n=1}^\infty nx_n\overline{y_n}=\sum_{n=1}^\infty x_n\overline{ny_n}=\ip{x}{Ay}
    \end{align*}
    which says that $A^\ast\big|_{\mc{D}(A)}=A$, and $A\big|_{\mc{D}(A^\ast)}=A^\ast$, so $A^\ast\subset A$ and $A\subset A^\ast$, so $A=A^\ast$ -- i.e., $A$ is self-adjoint.\hfill{$\qed$}
    \end{document}
\end{document}