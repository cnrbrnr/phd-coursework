\documentclass[10pt]{article}
\usepackage[margin=1.3cm]{geometry}

% Packages
\usepackage{amsmath, amsfonts, amssymb, amsthm}
\usepackage{bbm} 
\usepackage{dutchcal} % [dutchcal, calrsfs, pzzcal] calligraphic fonts
\usepackage{graphicx}
\usepackage[T1]{fontenc}
\usepackage[tracking]{microtype}

% Palatino for text goes well with Euler
\usepackage[sc,osf]{mathpazo}   % With old-style figures and real smallcaps.
\linespread{1.025}              % Palatino leads a little more leading

% Euler for math and numbers
\usepackage[euler-digits,small]{eulervm}

% Command initialization
\DeclareMathAlphabet{\pazocal}{OMS}{zplm}{m}{n} \graphicspath{{./images/}}

% Custom Commands
\newcommand{\bs}[1]{\boldsymbol{#1}} \newcommand{\E}{\mathbb{E}}
\newcommand{\var}[1]{\text{Var}\left(#1\right)}
\newcommand{\bp}[1]{\left({#1}\right)} \newcommand{\mbb}[1]{\mathbb{#1}}
\newcommand{\1}[1]{\mathbbm{1}_{#1}} \newcommand{\mc}[1]{\mathcal{#1}}
\newcommand{\nck}[2]{{#1\choose#2}} \newcommand{\pc}[1]{\pazocal{#1}}
\newcommand{\ra}[1]{\renewcommand{\arraystretch}{#1}}
\newcommand*{\floor}[1]{\left\lfloor#1\right\rfloor}
\newcommand*{\ceil}[1]{\left\lceil#1\right\rceil}
\newcommand{\ip}[2]{\left\langle#1,#2\right\rangle }

\DeclareMathOperator{\Var}{Var} \DeclareMathOperator{\Cov}{Cov}
\DeclareMathOperator{\diag}{diag}
\DeclareMathOperator{\ran}{range}

\makeatletter
\def\Ddots{\mathinner{\mkern1mu\raise\p@
\vbox{\kern7\p@\hbox{.}}\mkern2mu
\raise4\p@\hbox{.}\mkern2mu\raise7\p@\hbox{.}\mkern1mu}}
\makeatother

\def\powertower#1#2{#1\ifnum#2>1 ^{\powertower{#1}{\numexpr#2-1\relax}}\fi}

\newtheorem{theorem}{Theorem}
\newtheorem{lemma}{Lemma}

\begin{document}

    \begin{center}
        {\bf\large{MATH 829: FUNCTIONAL ANALYSIS AND QUANTUM MECHANICS}}
        \smallskip
        \hrule
        \smallskip
        {\bf Assignment} 7\hfill {\bf Connor Braun} \hfill {\bf 2024-10-30}
    \end{center}
    \noindent{\bf Problem 2}. Let $(X,\ip{\cdot}{\cdot})$ be a Hilbert space over
    $\mbb{K}$ and denote $X^2:=X\times X$.\\[5pt]
    {\bf a)} Let $\ip{\cdot}{\cdot}_{X^2}:X^2\times X^2\rightarrow\mbb{K}$ be
    defined as
    $\ip{(x,y)}{(x^\prime,y^\prime)}_{X^2}:=\ip{x}{x^\prime}+\ip{y}{y^\prime}$.
    Show that $\ip{\cdot}{\cdot}_{X^2}$ is an inner product on $X^2$. Specify
    the norm induced by this inner product, and show $X^2$ equipped with it is
    itself a Hilbert space.\\[5pt]
    {\bf Proof}. First we'll show that this new function satisfies the necessary
    properties of an inner product. Take $(u,v)\in X^2$. Then
    \[\ip{(u,v)}{(u,v)}_{X^2}=\ip{u}{u}+\ip{v}{v}\geq0.\] If instead $(u,v)$ are
    such that $\ip{(u,v)}{(u,v)}_{X^2}=0$, Then both $\ip{u}{u}$ and $\ip{v}{v}$
    must be zero, so both $u,v=0$. For the final two axioms, take
    $\alpha,\beta\in\mbb{K}$ and $(u,v),(x,y),(w,z)\in X^2$, all arbitrary. Then
    linearity in the first argument follows:
    \begin{align*}
        \ip{\alpha(u,v)+\beta(x,y)}{(w,z)}_{X^2}=\ip{(\alpha u+\beta x, \alpha v+\beta y)}{(w,z)}_{X^2}&=\ip{\alpha u+\beta x}{w}+\ip{\alpha v+\beta y}{z}\\
        &=\alpha\ip{u}{w}+\beta\ip{x}{w}+\alpha\ip{v}{z}+\beta\ip{y}{z}\\
        &=\alpha(\ip{u}{w}+\ip{v}{z})+\beta(\ip{x}{w}+\ip{y}{z})\\
        &=\alpha\ip{(u,v)}{(w,z)}_{X^2}+\beta\ip{(x,y)}{(w,z)}_{X^2}.
    \end{align*}
    Conjugate symmetry follows similarly:
    \begin{align*}
        \ip{(u,v)}{(x,y)}_{X^2}=\ip{u}{x}+\ip{v}{y}=\overline{\ip{x}{u}}+\overline{\ip{y}{v}}=\overline{\ip{x}{u}+\ip{y}{v}}=\overline{\ip{(x,y)}{(u,v)}}_{X^2}.
    \end{align*}
    With this, we can conclude so far that $(X^2,\ip{\cdot}{\cdot}_{X^2})$ is
    pre-Hilbert. The norm induced by this inner product is
    \begin{align*}
        \|(x,y)\|_{X^2}=\left[\ip{(x,y)}{(x,y)}\right]^{1/2}=\left[\ip{x}{x}+\ip{y}{y}\right]^{1/2}=\left[\|x\|^2+\|y\|^2\right]^{1/2}.
    \end{align*}
    Henceforth, we shall drop the subscript on the norm and inner product when
    the subjacent space is contextually apparent. Finally, we claim that $X^2$ is Banach when equipped with this norm. 
    Suppose that $\{(x_n,y_n)\}_{n\geq 1}\subseteq X^2$ is Cauchy. That is, for $\varepsilon>0$ fixed we can find $N\geq 1$ so that $n,m\geq N$ implies
    \begin{align*}
        \|(x_n,y_n)-(x_m,y_m)\|^2<\varepsilon\quad\Rightarrow\quad \|(x_n-x_m, y_n-y_m)\|^2<\varepsilon\quad\Rightarrow\quad\|x_n-x_m\|^2+\|y_n-y_m\|^2<\varepsilon.
    \end{align*}
    This lattermost implication means that both $\|x_n-x_m\|^2<\varepsilon$ and $\|y_n-y_m\|^2<\varepsilon$. That is, both $(x_n)_{n\geq 1}$ and $(y_n)_{n\geq 1}$ are Cauchy in $X$.
    Since $X$ is complete, $\exists x,y\in X$ so that $x_n\rightarrow x$ and $y_n\rightarrow y$ both as $n\rightarrow\infty$. Thus,
    \begin{align*}
        \lim_{n\rightarrow\infty}\|(x_n,y_n)-(x,y)\|^2=\lim_{n\rightarrow\infty}\|(x_n-x,y_n-y)\|^2=\lim_{n\rightarrow\infty}\|x_n-x\|^2+\lim_{n\rightarrow\infty}\|y_n-y\|^2=0
    \end{align*}
    certifying that $(x_n,y_n)\rightarrow(x,y)$ as $n\rightarrow\infty$ in $X^2$. Thus, $(X,\|\cdot\|_{X^2})$ is Banach, and we may further conclude $(X^2,\ip{\cdot}{\cdot}_{X^2})$ is a Hilbert space.\hfill{$\qed$}\\[5pt]
    {\bf b)} Let $A:\mc{D}(X)\subset X\rightarrow X$ be a linear operator. Then, let $\mc{U}:X^2\rightarrow X^2$ be the operator defined $\mc{U}(x,y)=(y,-x)$. Prove that if $A$ is densely defined, then
    \[\Gamma(A^\ast)=(\mc{U}(\Gamma(A)))^\perp\]
    where the orthocomplement is with respect to the inner product germane to $X^2$.\\[5pt]
    {\bf Proof}. First, suppose that $(x,y)\in\Gamma(A^\ast)$, such that $y=A^\ast x$ and $\exists C_x>0$ so that $|\ip{Az}{x}|\leq C_x\|z\|$ for all $z\in\mc{D}(A)$ (that is, $x\in \mc{D}(A^\ast)$).
    Now, take $(w,z)\in\mc{U}(\Gamma(A))$, so that $(-z,w)\in\Gamma(A)$. That is, $-z\in D(A)$ and $w=-Az$. Then we compute
    \begin{align*}
        \ip{(x,y)}{(w,z)}=\ip{x}{w}+\ip{y}{z}&=\ip{x}{-Az}+\ip{A^\ast x}{z}\\
        &=-\overline{\ip{Az}{x}}+\ip{A^\ast x}{z}\\
        &=-\overline{\ip{z}{A^\ast x}}+\ip{A^\ast x}{z}\tag{1}\\
        &=\ip{A^\ast x}{z}-\ip{A^\ast x}{z}\\
        &=0
    \end{align*}
    Where (1) holds since $x\in\mc{D}(A^\ast)$ and $z\in\mc{D}(A)$ by assumption. Since $(x,y)\perp \mc{U}(\Gamma(A))$, $\Gamma(A^\ast)\subseteq(\mc{U}(\Gamma(A)))^\perp$.
    Before proceeding, we require a technical lemma:
    \begin{lemma}\label{lem1}
        With $\mc{U}$, $X^2$ as above and $M\subseteq X^2$ we have
        \begin{align*}
            \mc{U}(M^\perp)=(\mc{U}(M))^\perp.
        \end{align*}
    \end{lemma}
    {\bf Proof}. If $(x,y)\in\mc{U}(M^\perp)$, then $(-y,x)\perp M$. Then, with $(w,z)\in\mc{U}(M)$, $(-z,w)\in M$, so 
    \[0=\ip{(-y,x)}{(-z,w)}=\ip{-y}{-z}+\ip{x}{w}=\ip{y}{z}+\ip{x}{w}=\ip{(x,y)}{(w,z)}\]
    so $(x,y)\perp\mc{U}(M)$, and $\mc{U}(M^\perp)\subseteq(\mc{U}(M))^\perp$. Conversely, with $(x,y)\in(\mc{U}(M))^\perp$,
    we have for any $(w,z)\in M$ that
    \[0=\ip{(x,y)}{\mc{U}(w,z)}=\ip{x}{z}+\ip{y}{-w}\tag{2}\]
    since $\mc{U}(w,z)\in\mc{U}(M)$. At the same time, noting $(x,y)=\mc{U}(-y,x)$, we have
    \begin{align*}
        \ip{(-y,x)}{(w,z)}=\ip{-y}{w}+\ip{x}{z}=\ip{y}{-w}+\ip{x}{z}=0
    \end{align*}
    where the final equality was found in (2). This says $(-y,x)\perp M$, so $(x,y)=\mc{U}(-y,x)\in\mc{U}(M^\perp)$, and therefore $(\mc{U}(M))^\perp\subseteq\mc{U}(M^\perp)$.
    Thus, $\mc{U}(M^\perp)=(\mc{U}(M))^\perp$, as claimed.\hfill{$\qed$}\\[5pt]
    Armed with this, suppose instead that $(x,y)\in(\mc{U}(\Gamma(A)))^\perp$. By lemma \ref{lem1}, $(x,y)\in\mc{U}((\Gamma(A))^\perp)$, so $(-y,x)\in\Gamma(A)^\perp$. We proceed to argue that $x\in\mc{D}(A^\ast)$.
    For this, take $w\in\mc{D}(A)$ and define $z=Aw$
    so that $(w,z)\in\Gamma(A)$. Then we have
    \begin{align*}
        0=\ip{(-y,x)}{(w,z)}=\ip{-y}{w}+\ip{x}{z}\quad\Rightarrow\quad\ip{x}{z}=\ip{y}{w}\quad\Rightarrow\quad \overline{\ip{Aw}{x}}=\overline{\ip{w}{y}}
    \end{align*}
    and thus,
    \begin{align*}
        |\ip{Aw}{x}|=|\ip{w}{y}|\leq\|y\|\|w\|
    \end{align*}
    by the Cauchy-Schwarz inequality. Since $w\in\mc{D}(A)$ was arbitrary and $\|y\|$ is independent of $w$, we have $x\in\mc{D}(A^\ast)$.
    Now suppose $(u,v)\in \mc{U}(\Gamma(A))$, so that $(-v,u)\in\Gamma(A)$ and further both $-v\in\mc{D}(A)$ (and $v\in\mc{D}(A)$, since $D(A)$ is a vector subspace of $X$) and $u=-Av$.
    With these facts, we compute
    \begin{align*}
        0=\ip{(x,y)}{(u,v)}=\ip{x}{u}+\ip{y}{v}\quad\Rightarrow\quad \ip{x}{-Av}+\ip{y}{v}=0\quad&\Rightarrow\quad \ip{y}{v}-\ip{A^\ast x}{v}=0\tag{3}\\
        &\Rightarrow\quad \ip{y-A^\ast x}{v}=0
    \end{align*}
    which says that $y-A^\ast x\perp\mc{D}(A)$ since $v\in\mc{D}(A)$ was arbitrary. But $\mc{D}(A)$ is dense in $X$, so we conclude that $y-A^\ast x=0$, i.e. $y=A^\ast x$. Since $x\in\mc{D}(A^\ast)$ and $y=A^\ast x$, $(x,y)\in\Gamma(A^\ast)$,
    so now $(\mc{U}(\Gamma(A)))^\perp\subseteq\Gamma(A^\ast)$. With both inclusions, we conclude that indeed $\Gamma(A^\ast)=(\mc{U}(\Gamma(A)))^\perp$.\hfill{$\qed$}\\[5pt]
    {\bf c)} Explain why (b) implies that if $(\mc{U}(\Gamma(A)))^\perp$ is the graph of some operator, then this operator is $A^\ast$.\\[5pt]
    {\bf Proof}. Let $B:\mc{D}(B)\subseteq X\rightarrow X$ be a linear operator with graph $\Gamma(B)$, and suppose that $\Gamma(B)=(\mc{U}(\Gamma(A)))^\perp$. Then (b) shows that $\Gamma(B)=\Gamma(A^\ast)$. If $x\in\mc{D}(B)$ and $y:=Bx$, then $(x,y)\in\Gamma(B)=\Gamma(A)$,
    so $x\in\mc{D}(A)$ and $y=Ax$. The reverse is obviously true, so $\mc{D}(B)=\mc{D}(A)$, and $Ax=Bx$ for all $x\in\mc{D}(A)$. Thus $A\subset B$ and $B\subset A$, so $B=A$. \hfill{$\qed$}\\[5pt]
    {\bf d)} Suppose that $A$ is a densely defined and closed linear operator. Show that both $X^2=\mc{U}(\Gamma(A))\oplus\Gamma(A^\ast)$ and $X^2=\mc{U}(\Gamma(A^\ast))\oplus\Gamma(A)$.\\[5pt]
    {\bf Proof}. Let us simply assert that $\mc{D}(A)$ is a vector subspace of $X$. This is manifestly obvious, for otherwise would violate the linearity of $A$. Similarly, it is clear that $\Gamma(A)$ and $\mc{U}(\Gamma(A))$ are vector subspaces of $X^2$, and
    we shall spare ourselves the tedious exercise of showing this -- suffice to say that these follow from the linearity of $A$ and $\mc{U}$. Let us show that $\mc{U}(\Gamma(A))$ is closed. Suppose $\{(x_n,y_n)\}_{n\geq 1}\subseteq\mc{U}(\Gamma(A))$
    and $(x_n,y_n)\rightarrow(x,y)$ as $n\rightarrow\infty$ with respect to $\|\cdot\|_{X^2}$. Then for $n\geq 1$, $(-y_n,x_n)\in\Gamma(A)$ and
    \begin{align*}
        \lim_{n\rightarrow\infty}\|(-y_n,x_n)-(-y,x)\|^2=\lim_{n\rightarrow\infty}\|x_n-x\|^2+\lim_{n \rightarrow\infty}\|y_n-y\|^2=0
    \end{align*}
    so $(-y_n,x_n)\rightarrow(-y,x)$ as $n\rightarrow\infty$. Since $\Gamma(A)$ is closed, $(-y,x)\in\Gamma(A)$, and $\mc{U}(-y,x)=(x,y)\in\mc{U}(\Gamma(A))$. Thus, $\mc{U}(\Gamma(A))$ is closed, so
    \[X^2=\mc{U}(\Gamma(A))\oplus(\mc{U}(\Gamma(A)))^\perp=\mc{U}(\Gamma(A))\oplus\Gamma(A^\ast)\]
    as desired. For the second claim, since $\mc{U}(\Gamma(A))$ was found to be closed, we have
    \[\mc{U}(\Gamma(A))=((\mc{U}(\Gamma(A)))^\perp)^\perp=\Gamma(A^\ast)^\perp.\]
    Now, observe that $\mc{U}\circ\mc{U}\equiv-\mbb{I}$, where $\mbb{I}$ is the identity operator on $X^2$, since for any $(x,y)\in X^2$, $\mc{U}^2(x,y)=(-x,-y)=-(x,y)$.
    But $\Gamma(A)$ is a linear subspace of $X^2$, so $\mc{U}^2(\Gamma(A))=\Gamma(A)$. Thus,
    \[\Gamma(A)=\mc{U}(\Gamma(A^\ast)^\perp)=(\mc{U}(\Gamma(A^\ast)))^\perp\tag{4}\]
    where the last equality follows from lemma \ref{lem1}. Finally, observe that $\Gamma(A^\ast)$ was found to be the orthocomplement of $\mc{U}(\Gamma(A))$, and is thus closed. Just as we showed that $\Gamma(A)$ closed implies $\mc{U}(\Gamma(A))$ is too, we have
    $\mc{U}(\Gamma(A^\ast))$ is closed. Thus:
    \begin{align*}
        X^2=\mc{U}(\Gamma(A^\ast))\oplus(\mc{U}(\Gamma(A^\ast)))^\perp=\mc{U}(\Gamma(A^\ast))\oplus\Gamma(A)
    \end{align*}
    where the last equality follows from (4), and we are done.\hfill{$\qed$}\\[5pt]
    {\bf e)} Explain why parts (c) and (d) together imply that if $A$ is a densely defined and closed linear operator, then $\Gamma(A)$ is the graph of $A^{\ast\ast}$. In particular, this shows that under these hypotheses $A=A^{\ast\ast}$.\\[5pt]
    {\bf Proof}. In class we proved that when $A$ is densely defined and closed, $A^\ast$ is densely defined. Thus $A^{\ast\ast}$ is uniquely defined. From (b), we now have
    \begin{align*}
        \Gamma(A^{\ast\ast})=(\mc{U}(\Gamma(A^\ast)))^\perp.
    \end{align*}
    But from (d), the equality (4) says $\Gamma(A)=(\mc{U}(\Gamma(A^\ast)))^\perp$, and (c) thus guarantees that $\Gamma(A)=\Gamma(A^{\ast\ast})$, and further $A=A^{\ast\ast}$.\hfill{$\qed$}\\[5pt]
\end{document}