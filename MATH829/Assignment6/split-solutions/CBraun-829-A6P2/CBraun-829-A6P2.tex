\documentclass[10pt]{article}
\usepackage[margin=1.3cm]{geometry}

% Packages
\usepackage{amsmath, amsfonts, amssymb, amsthm}
\usepackage{bbm} 
\usepackage{dutchcal} % [dutchcal, calrsfs, pzzcal] calligraphic fonts
\usepackage{graphicx}
\usepackage[T1]{fontenc}
\usepackage[tracking]{microtype}

% Palatino for text goes well with Euler
\usepackage[sc,osf]{mathpazo}   % With old-style figures and real smallcaps.
\linespread{1.025}              % Palatino leads a little more leading

% Euler for math and numbers
\usepackage[euler-digits,small]{eulervm}

% Command initialization
\DeclareMathAlphabet{\pazocal}{OMS}{zplm}{m}{n} \graphicspath{{./images/}}

% Custom Commands
\newcommand{\bs}[1]{\boldsymbol{#1}} \newcommand{\E}{\mathbb{E}}
\newcommand{\var}[1]{\text{Var}\left(#1\right)}
\newcommand{\bp}[1]{\left({#1}\right)} \newcommand{\mbb}[1]{\mathbb{#1}}
\newcommand{\1}[1]{\mathbbm{1}_{#1}} \newcommand{\mc}[1]{\mathcal{#1}}
\newcommand{\nck}[2]{{#1\choose#2}} \newcommand{\pc}[1]{\pazocal{#1}}
\newcommand{\ra}[1]{\renewcommand{\arraystretch}{#1}}
\newcommand*{\floor}[1]{\left\lfloor#1\right\rfloor}
\newcommand*{\ceil}[1]{\left\lceil#1\right\rceil}
\newcommand{\ip}[2]{\left\langle#1,#2\right\rangle }

\DeclareMathOperator{\Var}{Var} \DeclareMathOperator{\Cov}{Cov}
\DeclareMathOperator{\diag}{diag}

\makeatletter
\def\Ddots{\mathinner{\mkern1mu\raise\p@
\vbox{\kern7\p@\hbox{.}}\mkern2mu
\raise4\p@\hbox{.}\mkern2mu\raise7\p@\hbox{.}\mkern1mu}}
\makeatother

\def\powertower#1#2{#1\ifnum#2>1 ^{\powertower{#1}{\numexpr#2-1\relax}}\fi}

\newtheorem{theorem}{Theorem}
\newtheorem{lemma}{Lemma}

\begin{document}

    \begin{center}
        {\bf\large{MATH 829: FUNCTIONAL ANALYSIS AND QUANTUM MECHANICS}}
        \smallskip
        \hrule
        \smallskip
        {\bf Assignment} 6\hfill {\bf Connor Braun} \hfill {\bf 2024-10-25}
    \end{center}
    \vspace{5pt}
    \noindent{\bf Problem 2} For each of the following linear functions, say whether it is a bounded linear functional and -- if so -- find an upper bound for it s$\|\cdot\|_\ast$-norm.\\[5pt]
    {\bf a)} $A:(\mc{C}([-1,1]), \|\cdot\|_\infty)\rightarrow\mbb{C}$, with $A(u)=u(0)$.\\[5pt]
    {\bf Solution}. The functional $A$ is bounded. Let $u\in \mc{C}([-1,1])$ with $\|u\|_\infty=1$. Then
    \[\|u\|_\infty=\max_{-1\leq t\leq 1}|u(t)|\geq |u(s)|,\quad s\in[-1,1].\]
    In particular, we get $|Au|=|u(0)|\leq 1$. Then
    \[\|A\|_\ast=\sup\{|Au|:\;u\in \mc{C}([-1,1]),\;\|u\|_\infty=1\}=\{|u(0)|:\;u\in \mc{C}([-1,1]),\;\|u\|_\infty=1\}\leq\sup\{1\}=1.\tag{1}\]
    Henceforth, we shall allude to the logic encapsulated in (1) to obtain a bound in $\|\cdot\|_\ast$ once we have bounded $\Gamma x$ for any vector $x$ with $\|x\|=1$, and $\Gamma$
    whatever linear functional that happens to be in question.\hfill{$\qed$}\\[5pt]
    {\bf b)} $B:(\mc{C}([-1,1]),\|\cdot\|_\infty)\rightarrow\mbb{C}$, $B(u)=\int_-1^1t^8u(t)dt$.\\[5pt]
    {\bf Solution}. The functional $B$ is bounded. Let $u\in \mc{C}([-1,1])$ be such that $\|u\|_\infty=1$. Then
    \begin{align*}
        |Bu|=\left|\int_{-1}^1t^8u(t)dt\right|\leq \int_{-1}^1|t^8||u(t)|dt\leq \int_{-1}^1t^8dt=\frac{1}{9}t^9\bigg|^{1}_{-1}=\frac{2}{9}.
    \end{align*}
    Referring back to (1), we get $\|B\|_\ast\leq\frac{2}{9}$.\hfill{$\qed$}\\[5pt]
    {\bf c)} $C:\ell_3\rightarrow\mbb{C}$, $C(x)=\sum_{n=1}^\infty\frac{x_n}{n^{3/4}}$, where $x=(x_{n})_{n\geq 1}$.\\[5pt]
    {\bf Solution}. The functional $C$ is bounded. We first observe that $\alpha=(1/n^{3/4})_{n\geq 1}\in\ell_{3/2}$, since
    \begin{align*}
        \|\alpha\|_{3/2}^{3/2}=\sum_{n=1}^\infty\left|\frac{1}{n^{3/4}}\right|^{3/2}=\sum_{n=1}^\infty\frac{1}{n^{9/8}}<\infty.
    \end{align*}
    With this, one can use H\"older's inequality to obtain a bound. Let $x\in\ell_3$. Then
    \begin{align*}
        |Cx|=\left|\sum_{n=1}^\infty\frac{x_n}{n^{3/4}}\right|\leq \sum_{n=1}^\infty\left|\frac{x_n}{n^{3/4}}\right|&\leq\left(\sum_{n=1}^\infty|x_n|^3\right)^{1/3}\left(\sum_{n=1}^\infty\|\frac{1}{n^{3/4}}\|^{3/2}\right)^{2/3}\\
        &=\|\alpha\|_{3/2}\\
        &<\infty
    \end{align*}
    where, by the logic in (1), $\|C\|_\ast\leq \|\alpha\|_{3/2}$.\hfill{$\qed$}\\[5pt]
    {\bf d)} $D:(L_2([0,\pi]),\|\cdot\|_2)\rightarrow\mbb{C}$, $D(u)=\int_0^\pi\cos(t)u(t)dt$.\\[5pt]
    {\bf Solution}. The functional $D$ is bounded. Let $u\in L_2([0,\pi])$, $\|u\|_2=1$. Then notice
    \[\|\cos\|_2^2=\int_0^\pi|cos(t)|^2dt\leq \int_0^\pi dt=\pi.\]
    Just as for (c), an application of H\"older's inequality yields
    \begin{align*}
        |D(u)|=\left|\int_0^\pi\cos(t)u(t)dt\right|\leq\int_0^\pi|cos(t)u(t)|dt&\leq\left(\int_0^\pi|\cos(t)|^2dt\right)^{1/2}\left(\int_0^\pi |u(t)|^2dt\right)^{1/2}\\
        &\leq\sqrt{\pi}\|u\|_2\\
        &=\sqrt{\pi}. 
    \end{align*}
    This implies $\|D\|_\ast\leq \sqrt{\pi}$ via (1).\hfill{$\qed$}\\[5pt]
    {\bf e)} $E:\ell_2\rightarrow\mbb{C}$, $E(x)=\sum_{n=1}^\infty\frac{x_n+x_{n+1}}{2^n}$, $x=(x_n)_{n\geq 1}$.\\[5pt]
    {\bf Solution}. The functional $E$ is bounded. We begin by computing bounds on some series to be used later. First, wit $\alpha=(\frac{1}{2^n})_{n\geq 1}$,
    \begin{align*}
        \|\alpha\|_2^2=\sum_{n=1}^\infty\left|\frac{1}{2^n}\right|^2=\sum_{n=1}^\infty\left(\frac{1}{4}\right)^n=-1+\sum_{n=0}^\infty\left(\frac{1}{4}\right)^n=\frac{1}{1-1/4}-1=\frac{1}{3}
    \end{align*}
    so $\|\alpha\|_2=\frac{1}{\sqrt{3}}$. Now, when $x\in\ell_2$ is such that $\|x\|_2=1$, we have
    \begin{align*}
        \sum_{n=1}^\infty\left|\frac{x_{n+1}}{2^n}\right|&\leq \left(\sum_{n=1}^\infty|x_{n+1}|^2\right)^{1/2}\left(\sum_{n=1}^\infty\left|\frac{1}{2^n}\right|^2\right)^{1/2}\\
        &\leq\left(\sum_{n=1}^\infty|x_n|^2\right)^{1/2}\frac{1}{\sqrt{3}}\\
        &=\|x\|_2\frac{1}{\sqrt{3}}\\
        &=\frac{1}{\sqrt{3}}\tag{2}.
    \end{align*}
    Quite clearly, an identical application of H\"older's inequality also yields $\sum_{n=1}^\infty|\frac{x_n}{2^n}|\leq\frac{1}{\sqrt{3}}$. The convergence of these two series
    justifies the convergence of $Ex$ and allows us to determine a bound on its norm as follows:
    \begin{align*}
        |Ex|=\left|\sum_{n=1}^\infty\frac{x_{n}+x_{n+1}}{2^n}\right|\leq \sum_{n=1}^\infty\left|\frac{x_n}{2^n}+\frac{x_{n+1}}{2^n}\right|&\leq \sum_{n=1}^\infty\left|\frac{x_{n}}{2^n}\right|+\left|\frac{x_{n+1}}{2^n}\right|\\
        &=\lim_{N\rightarrow\infty}\sum_{n=1}^N\left|\frac{x_{n}}{2^n}\right|+\left|\frac{x_{n+1}}{2^n}\right|\\
        &=\lim_{N\rightarrow\infty}\sum_{n=1}^N\left|\frac{x_{n}}{2^n}\right|+\lim_{N\rightarrow\infty}\sum_{n=1}^N\left|\frac{x_{n+1}}{2^n}\right|\tag{3}\\
        &\leq \frac{2}{\sqrt{3}}
    \end{align*}
    where step (3) is justified by the convergence of both limits, as found in (2). With this, we conclude $\|E\|_\ast\leq \frac{2}{\sqrt{3}}$, by the same calculation as in (1).\hfill{$\qed$}\\[5pt]

\end{document}