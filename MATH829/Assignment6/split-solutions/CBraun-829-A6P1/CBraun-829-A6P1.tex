\documentclass[10pt]{article}
\usepackage[margin=1.3cm]{geometry}

% Packages
\usepackage{amsmath, amsfonts, amssymb, amsthm}
\usepackage{bbm} 
\usepackage{dutchcal} % [dutchcal, calrsfs, pzzcal] calligraphic fonts
\usepackage{graphicx}
\usepackage[T1]{fontenc}
\usepackage[tracking]{microtype}

% Palatino for text goes well with Euler
\usepackage[sc,osf]{mathpazo}   % With old-style figures and real smallcaps.
\linespread{1.025}              % Palatino leads a little more leading

% Euler for math and numbers
\usepackage[euler-digits,small]{eulervm}

% Command initialization
\DeclareMathAlphabet{\pazocal}{OMS}{zplm}{m}{n} \graphicspath{{./images/}}

% Custom Commands
\newcommand{\bs}[1]{\boldsymbol{#1}} \newcommand{\E}{\mathbb{E}}
\newcommand{\var}[1]{\text{Var}\left(#1\right)}
\newcommand{\bp}[1]{\left({#1}\right)} \newcommand{\mbb}[1]{\mathbb{#1}}
\newcommand{\1}[1]{\mathbbm{1}_{#1}} \newcommand{\mc}[1]{\mathcal{#1}}
\newcommand{\nck}[2]{{#1\choose#2}} \newcommand{\pc}[1]{\pazocal{#1}}
\newcommand{\ra}[1]{\renewcommand{\arraystretch}{#1}}
\newcommand*{\floor}[1]{\left\lfloor#1\right\rfloor}
\newcommand*{\ceil}[1]{\left\lceil#1\right\rceil}
\newcommand{\ip}[2]{\left\langle#1,#2\right\rangle }

\DeclareMathOperator{\Var}{Var} \DeclareMathOperator{\Cov}{Cov}
\DeclareMathOperator{\diag}{diag}

\makeatletter
\def\Ddots{\mathinner{\mkern1mu\raise\p@
\vbox{\kern7\p@\hbox{.}}\mkern2mu
\raise4\p@\hbox{.}\mkern2mu\raise7\p@\hbox{.}\mkern1mu}}
\makeatother

\def\powertower#1#2{#1\ifnum#2>1 ^{\powertower{#1}{\numexpr#2-1\relax}}\fi}

\newtheorem{theorem}{Theorem}
\newtheorem{lemma}{Lemma}

\begin{document}

    \begin{center}
        {\bf\large{MATH 829: FUNCTIONAL ANALYSIS AND QUANTUM MECHANICS}}
        \smallskip
        \hrule
        \smallskip
        {\bf Assignment} 6\hfill {\bf Connor Braun} \hfill {\bf 2024-10-25}
    \end{center}
    \vspace{5pt}
    \begin{center}
        \begin{minipage}{\dimexpr\paperwidth-10cm}
            I did not work with anybody in completing this assignment.
        \end{minipage}
    \end{center}
    \noindent\underline{\textbf{Project Title, Abstract and Bibliography}}\\[5pt]
    \textbf{Title}: Mean Ergodicity and Harmonic Analysis of Weakly Stationary Processes\\[5pt]
    \textbf{Abstract}: Following the accounts of \cite[ch. 1,8]{Eisner_etal_2015},\cite[ch. II.5]{Reed_Simon_1980}, a brief historical perspective on the statistical mechanical impetus
    for ergodic theory is provided, along with the necessary background and some
    physical interpretations. The main result, owing to \cite{Neumann_1932} is
    then proven following the expositions of \cite{Eisner_etal_2015},\cite{Reed_Simon_1980}, \cite{Weber_2009} along with a
    functional-analytic interpretation of ergodicity. Then, to emphasize the
    wide applicability of ergodic theory, various sources of mean ergodic operators
    are reviewed, including weakly stationary processes, measure-preserving
    dynamics and irreducible stochastic matrices associated to finite-state Markov chains.
    Finally, a mean-type Wiener-Wintner theorem \cite{Wiener_Wintner_1941}, \cite{Assani_2003} is
    presented, following the presentation found in \cite{Weber_2009}. This yields an
    uncountable system of orthogonal elements, such that when the subjacent
    Hilbert space is separable nearly all of these are zero. Time (and space) permitting, we
    shall explore how these special nonzero elements can be used to obtain a Fourier
    decomposition for weakly stationary processes \cite[$\S$6]{Fan_1946}.
    \bibliography{assignment6}
    \bibliographystyle{ieeetr}
\end{document}