\documentclass[10pt]{article}
\usepackage[margin=1.3cm]{geometry}

% Packages
\usepackage{amsmath, amsfonts, amssymb, amsthm}
\usepackage{bbm} 
\usepackage{dutchcal} % [dutchcal, calrsfs, pzzcal] calligraphic fonts
\usepackage{graphicx}
\usepackage[T1]{fontenc}
\usepackage[tracking]{microtype}

% Palatino for text goes well with Euler
\usepackage[sc,osf]{mathpazo}   % With old-style figures and real smallcaps.
\linespread{1.025}              % Palatino leads a little more leading

% Euler for math and numbers
\usepackage[euler-digits,small]{eulervm}

% Command initialization
\DeclareMathAlphabet{\pazocal}{OMS}{zplm}{m}{n} \graphicspath{{./images/}}

% Custom Commands
\newcommand{\bs}[1]{\boldsymbol{#1}} \newcommand{\E}{\mathbb{E}}
\newcommand{\var}[1]{\text{Var}\left(#1\right)}
\newcommand{\bp}[1]{\left({#1}\right)} \newcommand{\mbb}[1]{\mathbb{#1}}
\newcommand{\1}[1]{\mathbbm{1}_{#1}} \newcommand{\mc}[1]{\mathcal{#1}}
\newcommand{\nck}[2]{{#1\choose#2}} \newcommand{\pc}[1]{\pazocal{#1}}
\newcommand{\ra}[1]{\renewcommand{\arraystretch}{#1}}
\newcommand*{\floor}[1]{\left\lfloor#1\right\rfloor}
\newcommand*{\ceil}[1]{\left\lceil#1\right\rceil}
\newcommand{\ip}[2]{\left\langle#1,#2\right\rangle }

\DeclareMathOperator{\Var}{Var} \DeclareMathOperator{\Cov}{Cov}
\DeclareMathOperator{\diag}{diag}

\makeatletter
\def\Ddots{\mathinner{\mkern1mu\raise\p@
\vbox{\kern7\p@\hbox{.}}\mkern2mu
\raise4\p@\hbox{.}\mkern2mu\raise7\p@\hbox{.}\mkern1mu}}
\makeatother

\def\powertower#1#2{#1\ifnum#2>1 ^{\powertower{#1}{\numexpr#2-1\relax}}\fi}

\newtheorem{theorem}{Theorem}
\newtheorem{lemma}{Lemma}

\begin{document}

    \begin{center}
        {\bf\large{MATH 829: FUNCTIONAL ANALYSIS AND QUANTUM MECHANICS}}
        \smallskip
        \hrule
        \smallskip
        {\bf Assignment} 6\hfill {\bf Connor Braun} \hfill {\bf 2024-10-25}
    \end{center}
    \vspace{5pt}
    \begin{center}
        \begin{minipage}{\dimexpr\paperwidth-10cm}
            I did not work with anybody in completing this assignment.
        \end{minipage}
    \end{center}
    \underline{\textbf{Project Title, Abstract and Bibliography}}\\[5pt]
    \textbf{Title}: Mean Ergodicity and Harmonic Analysis of Weakly Stationary Processes\\[5pt]
    \textbf{Abstract}: Following the accounts of \cite[ch. 1,8]{Eisner_etal_2015},\cite[ch. II.5]{Reed_Simon_1980}, a brief historical perspective on the statistical mechanical impetus
    for ergodic theory is provided, along with the necessary background and some
    physical interpretations. The main result, owing to \cite{Neumann_1932} is
    then proven following the expositions of \cite{Eisner_etal_2015},\cite{Reed_Simon_1980}, \cite{Weber_2009} along with a
    functional-analytic interpretation of ergodicity. Then, to emphasize the
    wide applicability of ergodic theory, various sources of mean ergodic operators
    are reviewed, including weakly stationary processes, measure-preserving
    dynamics and irreducible stochastic matrices associated to finite-state Markov chains.
    Finally, a mean-type Wiener-Wintner theorem \cite{Wiener_Wintner_1941}, \cite{Assani_2003} is
    presented, following the presentation found in \cite{Weber_2009}. This yields an
    uncountable system of orthogonal elements, such that when the subjacent
    Hilbert space is separable nearly all of these are zero. Time (and space) permitting, we
    shall explore how these special nonzero elements can be used to obtain a Fourier
    decomposition for weakly stationary processes \cite[$\S$6]{Fan_1946}.
    \bibliography{assignment6}
    \bibliographystyle{ieeetr}
    \newpage
    \noindent{\bf Problem 2} For each of the following linear functions, say whether it is a bounded linear functional and -- if so -- find an upper bound for it s$\|\cdot\|_\ast$-norm.\\[5pt]
    {\bf a)} $A:(\mc{C}([-1,1]), \|\cdot\|_\infty)\rightarrow\mbb{C}$, with $A(u)=u(0)$.\\[5pt]
    {\bf Solution}. The functional $A$ is bounded. Let $u\in \mc{C}([-1,1])$ with $\|u\|_\infty=1$. Then
    \[\|u\|_\infty=\max_{-1\leq t\leq 1}|u(t)|\geq |u(s)|,\quad s\in[-1,1].\]
    In particular, we get $|Au|=|u(0)|\leq 1$. Then
    \[\|A\|_\ast=\sup\{|Au|:\;u\in \mc{C}([-1,1]),\;\|u\|_\infty=1\}=\{|u(0)|:\;u\in \mc{C}([-1,1]),\;\|u\|_\infty=1\}\leq\sup\{1\}=1.\tag{1}\]
    Henceforth, we shall allude to the logic encapsulated in (1) to obtain a bound in $\|\cdot\|_\ast$ once we have bounded $\Gamma x$ for any vector $x$ with $\|x\|=1$, and $\Gamma$
    whatever linear functional that happens to be in question.\hfill{$\qed$}\\[5pt]
    {\bf b)} $B:(\mc{C}([-1,1]),\|\cdot\|_\infty)\rightarrow\mbb{C}$, $B(u)=\int_-1^1t^8u(t)dt$.\\[5pt]
    {\bf Solution}. The functional $B$ is bounded. Let $u\in \mc{C}([-1,1])$ be such that $\|u\|_\infty=1$. Then
    \begin{align*}
        |Bu|=\left|\int_{-1}^1t^8u(t)dt\right|\leq \int_{-1}^1|t^8||u(t)|dt\leq \int_{-1}^1t^8dt=\frac{1}{9}t^9\bigg|^{1}_{-1}=\frac{2}{9}.
    \end{align*}
    Referring back to (1), we get $\|B\|_\ast\leq\frac{2}{9}$.\hfill{$\qed$}\\[5pt]
    {\bf c)} $C:\ell_3\rightarrow\mbb{C}$, $C(x)=\sum_{n=1}^\infty\frac{x_n}{n^{3/4}}$, where $x=(x_{n})_{n\geq 1}$.\\[5pt]
    {\bf Solution}. The functional $C$ is bounded. We first observe that $\alpha=(1/n^{3/4})_{n\geq 1}\in\ell_{3/2}$, since
    \begin{align*}
        \|\alpha\|_{3/2}^{3/2}=\sum_{n=1}^\infty\left|\frac{1}{n^{3/4}}\right|^{3/2}=\sum_{n=1}^\infty\frac{1}{n^{9/8}}<\infty.
    \end{align*}
    With this, one can use H\"older's inequality to obtain a bound. Let $x\in\ell_3$. Then
    \begin{align*}
        |Cx|=\left|\sum_{n=1}^\infty\frac{x_n}{n^{3/4}}\right|\leq \sum_{n=1}^\infty\left|\frac{x_n}{n^{3/4}}\right|&\leq\left(\sum_{n=1}^\infty|x_n|^3\right)^{1/3}\left(\sum_{n=1}^\infty\|\frac{1}{n^{3/4}}\|^{3/2}\right)^{2/3}\\
        &=\|\alpha\|_{3/2}\\
        &<\infty
    \end{align*}
    where, by the logic in (1), $\|C\|_\ast\leq \|\alpha\|_{3/2}$.\hfill{$\qed$}\\[5pt]
    {\bf d)} $D:(L_2([0,\pi]),\|\cdot\|_2)\rightarrow\mbb{C}$, $D(u)=\int_0^\pi\cos(t)u(t)dt$.\\[5pt]
    {\bf Solution}. The functional $D$ is bounded. Let $u\in L_2([0,\pi])$, $\|u\|_2=1$. Then notice
    \[\|\cos\|_2^2=\int_0^\pi|cos(t)|^2dt\leq \int_0^\pi dt=\pi.\]
    Just as for (c), an application of H\"older's inequality yields
    \begin{align*}
        |D(u)|=\left|\int_0^\pi\cos(t)u(t)dt\right|\leq\int_0^\pi|cos(t)u(t)|dt&\leq\left(\int_0^\pi|\cos(t)|^2dt\right)^{1/2}\left(\int_0^\pi |u(t)|^2dt\right)^{1/2}\\
        &\leq\sqrt{\pi}\|u\|_2\\
        &=\sqrt{\pi}. 
    \end{align*}
    This implies $\|D\|_\ast\leq \sqrt{\pi}$ via (1).\hfill{$\qed$}\\[5pt]
    {\bf e)} $E:\ell_2\rightarrow\mbb{C}$, $E(x)=\sum_{n=1}^\infty\frac{x_n+x_{n+1}}{2^n}$, $x=(x_n)_{n\geq 1}$.\\[5pt]
    {\bf Solution}. The functional $E$ is bounded. We begin by computing bounds on some series to be used later. First, wit $\alpha=(\frac{1}{2^n})_{n\geq 1}$,
    \begin{align*}
        \|\alpha\|_2^2=\sum_{n=1}^\infty\left|\frac{1}{2^n}\right|^2=\sum_{n=1}^\infty\left(\frac{1}{4}\right)^n=-1+\sum_{n=0}^\infty\left(\frac{1}{4}\right)^n=\frac{1}{1-1/4}-1=\frac{1}{3}
    \end{align*}
    so $\|\alpha\|_2=\frac{1}{\sqrt{3}}$. Now, when $x\in\ell_2$ is such that $\|x\|_2=1$, we have
    \begin{align*}
        \sum_{n=1}^\infty\left|\frac{x_{n+1}}{2^n}\right|&\leq \left(\sum_{n=1}^\infty|x_{n+1}|^2\right)^{1/2}\left(\sum_{n=1}^\infty\left|\frac{1}{2^n}\right|^2\right)^{1/2}\\
        &\leq\left(\sum_{n=1}^\infty|x_n|^2\right)^{1/2}\frac{1}{\sqrt{3}}\\
        &=\|x\|_2\frac{1}{\sqrt{3}}\\
        &=\frac{1}{\sqrt{3}}\tag{2}.
    \end{align*}
    Quite clearly, an identical application of H\"older's inequality also yields $\sum_{n=1}^\infty|\frac{x_n}{2^n}|\leq\frac{1}{\sqrt{3}}$. The convergence of these two series
    justifies the convergence of $Ex$ and allows us to determine a bound on its norm as follows:
    \begin{align*}
        |Ex|=\left|\sum_{n=1}^\infty\frac{x_{n}+x_{n+1}}{2^n}\right|\leq \sum_{n=1}^\infty\left|\frac{x_n}{2^n}+\frac{x_{n+1}}{2^n}\right|&\leq \sum_{n=1}^\infty\left|\frac{x_{n}}{2^n}\right|+\left|\frac{x_{n+1}}{2^n}\right|\\
        &=\lim_{N\rightarrow\infty}\sum_{n=1}^N\left|\frac{x_{n}}{2^n}\right|+\left|\frac{x_{n+1}}{2^n}\right|\\
        &=\lim_{N\rightarrow\infty}\sum_{n=1}^N\left|\frac{x_{n}}{2^n}\right|+\lim_{N\rightarrow\infty}\sum_{n=1}^N\left|\frac{x_{n+1}}{2^n}\right|\tag{3}\\
        &\leq \frac{2}{\sqrt{3}}
    \end{align*}
    where step (3) is justified by the convergence of both limits, as found in (2). With this, we conclude $\|E\|_\ast\leq \frac{2}{\sqrt{3}}$, by the same calculation as in (1).\hfill{$\qed$}\\[5pt]
    {\bf Problem 3}. For each of the following complex-valued functions on a Hilbert space, verify that they are linear and continuous. Then find the vector in the space representing them (in the sense of Riesz representation).\\[5pt]
    {\bf a)} $\Phi:\ell_2\rightarrow\mbb{C}$, $\Phi(x)=\sum_{k=2}^\infty\frac{x_{k+1}-x_{k-1}}{k}$.\\[5pt]
    {\bf Solution}. We follow a similar program as in problem 1.e. First, note that $\alpha=(1/k)_{k\geq 1}\in\ell^2$, since
    \begin{align*}
        \|\alpha\|_2^2=\sum_{k=1}^\infty\frac{1}{k^2}=\frac{\pi^2}{6}
    \end{align*}
    as we have shown in problem 2.c of assignment 5. Then, via H\"older's inequality, fow any $x\in\ell_2$ with $\|x\|_2=1$,
    \begin{align*}
        \sum_{n=2}^\infty\left|\frac{x_{n-1}}{n}\right|=\sum_{n=1}^\infty\left|\frac{x_n}{(n+1)}\right|&\leq\left(\sum_{n=1}^\infty|x_n|^2\right)^{1/2}\left(\sum_{n=1}^\infty\left|\frac{1}{n+1}\right|^2\right)^{1/2}\\
        &\leq\|x\|_2\left(\sum_{n=1}^\infty\frac{1}{n^2}\right)^{1/2}\\
        &=\frac{\pi}{\sqrt{6}}\tag{4}
    \end{align*}
    where in the second last line we simply added $1$ inside the (increasing) square root to convert the factor to $\|\alpha\|_2$. In an extremely similar manner:
    \begin{align*}
        \sum_{n=2}^\infty\left|\frac{x_{n+1}}{n}\right|=\sum_{n=3}^\infty\left|\frac{x_n}{n-1}\right|&\leq\left(\sum_{n=3}^\infty|x_n|^2\right)^{1/2}\left(\sum_{n=3}^\infty\frac{1}{(n-1)^2}\right)^{1/2}\\
        &\leq \left(\sum_{n=1}^\infty|x_n|^2\right)^{1/2}\left(\sum_{n=1}^\infty\frac{1}{n^2}\right)^{1/2}\\
        &=\frac{\pi}{\sqrt{6}}\tag{5}
    \end{align*}
    where in the second line we have added $|x_1|^2+|x_2|^2$ to the first factor and $1$ to the second to obtain the norms of $x$ and $\alpha$, respectively. We may now compute a bound on $\|\Phi\|_\ast$:
    \begin{align*}
        |\Phi(x)|=\left|\sum_{n=2}^\infty\frac{x_{n+1}-x_{n-1}}{n}\right|&\leq\sum_{n=2}^\infty\left|\frac{x_{n+1}}{n}\right|+\left|\frac{x_{n-1}}{n}\right|\\
        &=\sum_{n=2}^\infty\left|\frac{x_{n+1}}{n}\right|+\sum_{n=2}^\infty\left|\frac{x_{n-1}}{n}\right|\tag{6}\\
        &\leq\frac{2\pi}{\sqrt{6}}\\
        &=\frac{\sqrt{2}\pi}{\sqrt{3}}
    \end{align*}
    and so $\Phi$ is bounded. To establish linearity $\alpha,\beta\in\mbb{C}$, and $z,y\in\ell_2$. Then:
    \begin{align*}
        \Phi(\alpha y+\beta z)=\sum_{n=2}^\infty\frac{(\alpha y_{n+1}+\beta z_{n+1})-(\alpha y_{n-1}-\beta z_{n-1})}{n}&=\sum_{n=2}^\infty\alpha\frac{y_{n+1}-y_{n-1}}{n}+\beta\frac{z_{n+1}-z_{n-1}}{n}\\
        &=\alpha\sum_{n=1}^\infty\frac{y_{n+1}-y_{n-1}}{n}+\beta\sum_{n=2}^\infty\frac{z_{n+1}-z_{n-1}}{n}\\
        &=\alpha\Phi(y)+\beta\Phi(z)
    \end{align*}
    which holds since all series involved here are absolutely convergent. That is, $\Phi$ is linear, which along with its boundedness implies it is also continuous. To find the representing vector, call it $\xi=(\xi_n)_{n\geq 1}$, we note that for any $z\in\ell_2$, $\Phi(z)$ is stable under rearrangements
    since it is absolutely convergent (see for example, \cite[p.78, theorem 3.55]{Rudin_1976}). Then we compute
    \begin{align*}
        \Phi(z)=\sum_{n=2}^\infty\frac{z_{n+1}-z_{n-1}}{n}&=-\frac{1}{2}z_1-\frac{1}{3}z_2+\left(\frac{1}{2}-\frac{1}{4}\right)z_3+\left(\frac{1}{3}-\frac{1}{5}\right)z_4+\left(\frac{1}{4}-\frac{1}{6}\right)z_5+\dots\\
        &=\sum_{n=1}^\infty z_1\overline{\xi_n}\\
        &=\ip{z}{\xi}
    \end{align*}
    provided $\xi_1=-\frac{1}{2}$, $\xi_2=-\frac{1}{3}$, and for $n\geq 3$, $\xi_n=\frac{2}{n^2-1}$. To verify that this makes sense, we need to confirm that $\xi\in\ell_2$ under this choice. But this is easily seen, since
    \begin{align*}
        \|\xi\|_2^2=\frac{1}{4}+\frac{1}{9}+\sum_{n=3}^\infty\left(\frac{2}{n^2-1}\right)^2\leq 1+\sum_{n=3}^\infty\left(\frac{2}{n^2}\right)^2&\leq4\sum_{n=1}^\infty\frac{1}{n^4}=\frac{2\pi^4}{45}<\infty
    \end{align*}
    where $\sum_{n=1}^\infty\frac{1}{n^4}=\frac{\pi^4}{90}$ is as per problem 3.c of assignment 5. Thus, $\xi\in\ell_2$, and we are able to conclude that $\Phi(\cdot)=\ip{\cdot}{\xi}$. \hfill{$\qed$}\\[5pt]
    {\bf b)} $\Psi:L_2([0,1])\rightarrow\mbb{C}$, $\Psi(u)=\int_0^1 tu(t^3)dt$.\\[5pt]
    {\bf Solution}. Take $u,v\in L_2([0,1])$ and $\alpha,\beta\in\mbb{C}$. Then
    \begin{align*}
        \Psi(\alpha u+\beta v)=\int_0^1t(\alpha u(t)+\beta v(t))dt=\alpha\int_0^1tu(t)dt+\beta\int_0^1tv(t)dt=\alpha\Psi(u)+\beta\Psi(v)
    \end{align*}
    so $\Psi$ is linear. As in 3.b, we will establish continuity of $\Psi$ by showing it is bounded. For this, let $f\in L_2([0,1])$ be such that $\|f\|_2=1$. Then
    \begin{align*}
        |\Psi(f)|=\left|\int_0^1tf(t^3)dt\right|\leq\int_0^1|tf(t^3)|dt&=\int_0^1|\mu^{1/3} f(\mu)|d\mu\tag{substituting $t=\mu^{1/3}$.}\\
        &\leq\left(\int_0^1\mu^{2/3}d\mu\right)^{1/2}\left(\int_0^1|f(\mu)|^2d\mu\right)^{1/2}\tag{7}\\
        &=\|f\|_2\left(\frac{3}{5}\mu^{5/3}\bigg|^1_0\right)^{1/2}\\
        &=\sqrt{3/5}
    \end{align*}
    so $\|\Psi\|_\ast<\infty$. Since $\Psi$ is a linear and bounded functional on $L_2([0,1])$, it is continuous. Further, for any $g\in L_2([0,1])$, using the same substitution as above we get
    \begin{align*}
        \Psi(g)=\int_0^1g(\mu)\mu^{1/3}d\mu=\int_{0}^{1}g(t)\overline{t^{1/3}}dt=\ip{g}{\xi}
    \end{align*}
    provided we set $\xi(t)=t^{1/3}$ for $t\in[0,1]$. The computation in (7) verifies that $\xi\in L_2([0,1])$, and since $g\in L_2([0,1])$ was arbitrary, we conclude that $\Psi(\cdot)=\ip{\cdot}{\xi}$.\hfill{$\qed$}\\[5pt]
    {\bf Problem 4}. Let $\Phi:\ell_\infty\rightarrow\mbb{C}$ be a positive linear functional on $\ell_\infty$. That is, for every $x=(x_n)_{n\geq 1}\in\ell_\infty$ satisfying $x_n\geq 0$ for $n\geq 1$ we have
    $\Phi(x)\geq 0$ (with the provisio that a number $y\in\mbb{C}$ satisfying $y\geq 0$ implies $y\in\mbb{R}$). Prove that $\Phi$ is a bounded linear functional.\\[5pt]
    {\bf Proof}. Let us first consider a sequence $x\in\ell_\infty$ so that $x_n\in\mbb{R}$ for $n\geq 1$, and $\|x\|_\infty=1$. Throughout, we shall denote $\mathbbm{1}:=(1,1,1,\dots)\in\ell_\infty$. Then
    \[1=\|x\|_\infty =\sup_{n\geq 1}|x_n|\geq |x_k|,\quad k\geq 1\tag{8}\]
    and so the sequences $\mathbbm{1}-x$ and $\mathbbm{1}+x$ have elements $1-x_n\geq 0$ and $1+x_n\geq 0$  for $n\geq 1$. Thus, via the positivity and linearity of $\Phi$ we find
    \begin{align*}
        0\leq \Phi(\mathbbm{1}-x)\quad&\Rightarrow\quad \Phi(x)\leq \Phi(\mathbbm{1}) \\
        0\leq \Phi(\mathbbm{1}+x)\quad&\Rightarrow\quad -\Phi(x)\leq \Phi(\mathbbm{1})
    \end{align*}
    which, taken together, imply that $0\leq |\Phi(x)|\leq \Phi(\mathbbm{1})$ . Referring back to (8), it is clear that this holds for any real sequence $(z_n)_{n\geq 1}$ with $|z_n|\leq 1$ for $n\geq 1$.\\[5pt]
    Now take $\xi=(\xi_n)_{n\geq 1}\in\ell_\infty$, satisfying $\|\xi\|_\infty=1$, and with $\xi_n=\alpha_n+i\beta_n$ for some $\alpha_n,\beta_n\in\mbb{R}$ $\forall n\geq 1$. Denote $\alpha:=(\alpha_n)_{n\geq 1}$ and $\beta=(\beta_n)_{n\geq 1}$. Then we have
    \[1=\sup_{n\geq 1}|\xi_n|=\sup_{n\geq 1}(|\alpha_n|^2+|\beta_n|^2)^{1/2}\]
    from which we can see that both $1\geq |\alpha_n|$ and $1\geq |\beta_n|$ for $n\geq 1$. Thus,
    \begin{align*}
        |\Phi(\xi)|=|\Phi(\alpha+i\beta)|=|\Phi(\alpha)+i\Phi(\beta)|\leq |\Phi(\alpha)|+|i||\Phi(\beta)|\leq 2\Phi(\mathbbm{1})<\infty
    \end{align*}
    which holds regardless of the $\xi$ chosen on the unit sphere in $\ell_\infty$. Thus, $\|\Phi\|_\ast<\infty$, so $\Phi$ is bounded.\hfill{$\qed$}
    \end{document}